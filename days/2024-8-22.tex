\title{\date[d=22,m=8,y=2024][year:cn-y,年,month:cn,day:cn,日,·,weekday]·七月十九 ·处暑}
我就是这最后一个夜晚最后一盏黑暗的灯是最后一个夜晚水面上爱情阴沉的旗帜在黑暗中鞭打着一颗干渴的心沿着先知的梯子上下爬行\footnote{\bi{戈麦的诗} \regular{戈麦 西渡}}

