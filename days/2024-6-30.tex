\title{\date[d=30,m=6,y=2024][year:cn-a,年,month:cn,day:cn,日,,weekday]}
这是一场关于人类生活方式的革命:农业革命。 从采集走向农业的转变,始于大约公元前9500年~公元前8500年,发源于土耳其东南部、伊朗西部和地中海东部的丘陵地带。这场改变一开始速度缓慢,地区也有限。小麦与山羊驯化成为农作物和家畜的时间大约是在公元前9000年,豌豆和小扁豆约在公元前8000年,橄榄树在公元前5000年,马在公元前4000年,葡萄则是在公元前3500年。至于骆驼和腰果等其他动植物驯化的时间还要更晚,但不论如何,到了公元前3500年,主要一波驯化的热潮已经结束。即使到了今天,虽然人类有着种种先进科技,但食物热量超过90%的来源仍然是来自人类祖先在公元前9500年到公元前3500年间驯化的植物:小麦、稻米、玉米、马铃薯、小米和大麦。在过去2000年间,人类并没有驯化什么特别值得一提的动植物。可以说,人到现代还有着远古狩猎采集者的心,以及远古农民的胃。\footnote{人类简史:从动物到上帝, 尤瓦尔·赫拉利}

