\title{\date[d=29,m=1,y=2024][year:cn-a,年,month:cn,day:cn,日,,weekday]}
想要了解人类的天性、历史和心理,就得想办法回到那些狩猎采集的祖先头脑里面,看看他们的想法。在智人的历史上,他们绝大多数的时间都是靠采集为生。在过去两百年间,有越来越多智人的谋生方式是在城市里面劳动,整天坐办公桌前;而再之前的1万年,多数的智人则是务农或畜牧;但不论如何,比起先前几万年都在狩猎或采集,现代的谋生方式在历史上都只像是一瞬间的事罢了。 演化心理学近来发展蓬勃,认为现在人类的各种社会和心理特征早从农业时代之前就已经开始形塑。这个领域的学者认为,即使到了现在,我们的大脑和心灵都还是以狩猎和采集的生活方式在思维。我们的饮食习惯、冲突和性欲之所以是现在的样貌,正是因为我们还保留着狩猎采集者的头脑,但所处的却是工业化之后的环境,像是有超级城市、飞机、电话和计算机。在这样的环境下,我们比前人享有更多物质资源,拥有更长的寿命,但又觉得疏离、沮丧而压力重重。\footnote{人类简史:从动物到上帝, 尤瓦尔·赫拉利}

