\title{\date[d=26,m=12,y=2024][year:cn-y,年,month:cn,day:cn,日,,weekday]}
农业革命可能是史上最具争议的事件。有些人认为这让人类迈向繁荣和进步,也有人认为这条路终将导致灭亡。对后者来说,农业革命是个转折点,让智人抛下了与自然紧紧相连的共生关系,大步走向贪婪,自外于这个世界。但不管这条路的尽头为何,现在都已经无法回头。进入农业社会之后,人口得以急遽增加,任何一个复杂的农业社会想回到狩猎和采集的生活,就只有崩溃一途。大约在公元前10000年、进入农业时代的前夕,地球上采集者的人口大约有500万到800万。而到了公元1世纪,这个人数只剩下一两百万(主要在澳大利亚、美洲和非洲),相较于农业人口已达2.5亿,无疑是远远瞠乎其后。[插图] 绝大多数的农民都是住在永久聚落里,只有少数是游牧民族。“定居”这件事,让大多数人的活动范围大幅缩小。远古狩猎采集者的活动范围可能有几十甚至上百平方公里。当时这片范围都是他们的“家”,有山丘、溪流、树林,还有开阔的天空。\footnote{\bi{人类简史:从动物到上帝, 尤瓦尔·赫拉利}}

