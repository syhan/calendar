\title{\date[d=1,m=1,y=2024][year:cn-y,年,month:cn,day:cn,日,·,weekday]·十一月二十 }
但总有一天,我的手会远离我,如果我让它写,它就写下非我所想的句子。别样阐释的时代即将来临,任何词都不会在另一个词上停留,每一个意义都像云一样散掉、像水一样流走\footnote{\bi{布里格手记} \regular{里尔克}}

\title{\date[d=2,m=1,y=2024][year:cn-y,年,month:cn,day:cn,日,·,weekday]·十一月廿一 }
仿佛所有的时间只是这一刻仿佛这一刻就是全部的世界\footnote{\bi{张枣的诗} \regular{张枣 颜炼军}}

\title{\date[d=3,m=1,y=2024][year:cn-y,年,month:cn,day:cn,日,·,weekday]·十一月廿二 }
呼吸月光呼吸鸟语呼吸花朵和它的暴力\footnote{\bi{于坚的诗} \regular{于坚}}

\title{\date[d=4,m=1,y=2024][year:cn-y,年,month:cn,day:cn,日,·,weekday]·十一月廿三 }
我经历的感觉比我感到的所有人还要多,因为无论感到多少,我都觉得从未感觉够,\footnote{\bi{宇宙重建了自身:佩索阿诗精选} \regular{费尔南多·佩索阿}}

\title{\date[d=5,m=1,y=2024][year:cn-y,年,month:cn,day:cn,日,·,weekday]·十一月廿四 }
爱人,爱人,我们像神:整个世界都为了我们!我们在人间到处是家,一切都属于我们。\footnote{\bi{茨维塔耶娃诗选} \regular{茨维塔耶娃}}

\title{\date[d=6,m=1,y=2024][year:cn-y,年,month:cn,day:cn,日,·,weekday]·十一月廿五 ·小寒}
最终我不知道自己是在读诗还是在生活,我不知道我真正的位置是在这个世界上还是在你的诗歌里。\footnote{\bi{宇宙重建了自身:佩索阿诗精选} \regular{费尔南多·佩索阿}}

\title{\date[d=7,m=1,y=2024][year:cn-y,年,month:cn,day:cn,日,·,weekday]·十一月廿六 }
在不倦的火车的世界性的铁器隆隆声中,\footnote{\bi{宇宙重建了自身:佩索阿诗精选} \regular{费尔南多·佩索阿}}

\title{\date[d=8,m=1,y=2024][year:cn-y,年,month:cn,day:cn,日,·,weekday]·十一月廿七 }
我悄悄享受着入眠的可能。\footnote{\bi{不安之书} \regular{费尔南多·佩索阿 热罗尼莫·皮萨罗 }}

\title{\date[d=9,m=1,y=2024][year:cn-y,年,month:cn,day:cn,日,·,weekday]·十一月廿八 }
用感觉替代思想,不仅将感觉作为灵感的基础(这一点可以理解),还当作表达的手段(如果我们可以这样说的话)。\footnote{\bi{自决之书} \regular{费尔南多·佩索阿}}

\title{\date[d=10,m=1,y=2024][year:cn-y,年,month:cn,day:cn,日,·,weekday]·十一月廿九 }
带着明信片式的表情,\footnote{\bi{不安之书} \regular{费尔南多·佩索阿 热罗尼莫·皮萨罗 }}

\title{\date[d=11,m=1,y=2024][year:cn-y,年,month:cn,day:cn,日,·,weekday]·腊月初一 }
春天,一道道晦暗的台阶\footnote{\bi{骆一禾的诗} \regular{骆一禾 西渡}}

\title{\date[d=12,m=1,y=2024][year:cn-y,年,month:cn,day:cn,日,·,weekday]·腊月初二 }
可是等咱们从风信子花园回家,时间已晚,你双臂满抱,你的头发都湿了,我一句话都说不出来,眼睛也看不清了,我既不是活的也不是死的,我什么都不知道,茫然谛视着那光芒的心,一片寂静。大海荒芜而空寂。[插图]\footnote{\bi{荒原:艾略特文集·诗歌} \regular{T.S.艾略特}}

\title{\date[d=13,m=1,y=2024][year:cn-y,年,month:cn,day:cn,日,·,weekday]·腊月初三 }
我不会后退着生活,\footnote{\bi{抒情诗的呼吸:一九二六年书信(帕斯捷尔纳克作品系列)} \regular{鲍·列·帕斯捷尔纳克 玛·伊·茨维塔耶娃 莱·马·里尔克}}

\title{\date[d=14,m=1,y=2024][year:cn-y,年,month:cn,day:cn,日,·,weekday]·腊月初四 }
在黑暗的空心波浪,他弯曲,伸展,没有痕迹和声响,像一艘沉没的航船。\footnote{\bi{茨维塔耶娃诗选} \regular{茨维塔耶娃}}

\title{\date[d=15,m=1,y=2024][year:cn-y,年,month:cn,day:cn,日,·,weekday]·腊月初五 }
夜光具有艺术家的想象力,意味着魔幻与真实的间离。但是,赤裸却是一种深潜的本愿。\footnote{\bi{昌耀的诗} \regular{昌耀}}

\title{\date[d=16,m=1,y=2024][year:cn-y,年,month:cn,day:cn,日,·,weekday]·腊月初六 }
人们在思考我。\footnote{\bi{彩画集:兰波散文诗全集(译文经典)} \regular{阿蒂尔·兰波}}

\title{\date[d=17,m=1,y=2024][year:cn-y,年,month:cn,day:cn,日,·,weekday]·腊月初七 }
凡是我认为象征沉睡的东西,都有一种万物终结的声音,黑暗中的风声,如果再仔细听,还有我心肺的声音。\footnote{\bi{不安之书} \regular{费尔南多·佩索阿 热罗尼莫·皮萨罗 }}

\title{\date[d=18,m=1,y=2024][year:cn-y,年,month:cn,day:cn,日,·,weekday]·腊月初八 ·腊八节}
我必须孤独早上十分钟晚上十分钟。——什么事也不做。\footnote{\bi{沉石与火舌:特朗斯特罗姆诗全集} \regular{托马斯·特朗斯特罗姆}}

\title{\date[d=19,m=1,y=2024][year:cn-y,年,month:cn,day:cn,日,·,weekday]·腊月初九 }
每次告别都是一次死亡。是的,每次告别都是一次死亡。\footnote{\bi{宇宙重建了自身:佩索阿诗精选} \regular{费尔南多·佩索阿}}

\title{\date[d=20,m=1,y=2024][year:cn-y,年,month:cn,day:cn,日,·,weekday]·腊月初十 ·大寒}
我若是有胃口,只想吃泥土和石头。\footnote{\bi{彩画集:兰波散文诗全集(译文经典)} \regular{阿蒂尔·兰波}}

\title{\date[d=21,m=1,y=2024][year:cn-y,年,month:cn,day:cn,日,·,weekday]·腊月十一 }
我盯着卧室的墙,好像那是整个宇宙\footnote{\bi{想象一朵未来的玫瑰:佩索阿诗选} \regular{费尔南多·佩索阿}}

\title{\date[d=22,m=1,y=2024][year:cn-y,年,month:cn,day:cn,日,·,weekday]·腊月十二 }
我看见几只黑鸟在飞,像是从一部史诗中遗漏下的细节……\footnote{\bi{王家新的诗} \regular{王家新}}

\title{\date[d=23,m=1,y=2024][year:cn-y,年,month:cn,day:cn,日,·,weekday]·腊月十三 }
夜晚的幻觉化为抽象情感的低语,带着一种顾影自怜的母性关怀,栖居在想象的隐秘角落里。\footnote{\bi{不安之书} \regular{费尔南多·佩索阿 热罗尼莫·皮萨罗 }}

\title{\date[d=24,m=1,y=2024][year:cn-y,年,month:cn,day:cn,日,·,weekday]·腊月十四 }
它会萌芽吗?\footnote{\bi{顾城的诗} \regular{顾城}}

\title{\date[d=25,m=1,y=2024][year:cn-y,年,month:cn,day:cn,日,·,weekday]·腊月十五 }
泪水的故乡,泪水之乡也是心愿之乡心愿在河上摆渡,不能说生活是妄想遗忘的摇篮,遗忘的谷仓\footnote{\bi{戈麦的诗} \regular{戈麦 西渡}}

\title{\date[d=26,m=1,y=2024][year:cn-y,年,month:cn,day:cn,日,·,weekday]·腊月十六 }
能否把手机蓝牙的风吹草动吹得焚香袅袅,如心碎,如玉碎?\footnote{\bi{欧阳江河的诗} \regular{欧阳江河}}

\title{\date[d=27,m=1,y=2024][year:cn-y,年,month:cn,day:cn,日,·,weekday]·腊月十七 }
无论什么事情,只要站在终点眺望,就会变得宽容\footnote{\bi{金阁寺} \regular{三岛由纪夫}}

\title{\date[d=28,m=1,y=2024][year:cn-y,年,month:cn,day:cn,日,·,weekday]·腊月十八 }
做梦蚕食世界之外的世界\footnote{\bi{雪是谁说的谎:倪湛舸诗集} \regular{倪湛舸}}

\title{\date[d=29,m=1,y=2024][year:cn-y,年,month:cn,day:cn,日,·,weekday]·腊月十九 }
迷津从反面转为正面我与我的现在相遇我与我的现在直面每一轮行止\footnote{\bi{骆一禾的诗} \regular{骆一禾 西渡}}

\title{\date[d=30,m=1,y=2024][year:cn-y,年,month:cn,day:cn,日,·,weekday]·腊月二十 }
“我想要眼泪”\footnote{\bi{不安之书} \regular{费尔南多·佩索阿 热罗尼莫·皮萨罗 }}

\title{\date[d=31,m=1,y=2024][year:cn-y,年,month:cn,day:cn,日,·,weekday]·腊月廿一 }
至少去创造一种新的悲观主义吧,一种新的否定,这样就能产生幻觉,以为我们身上的某些东西即使不好,也会留下!\footnote{\bi{不安之书} \regular{费尔南多·佩索阿 热罗尼莫·皮萨罗 }}

\title{\date[d=1,m=2,y=2024][year:cn-y,年,month:cn,day:cn,日,·,weekday]·腊月廿二 }
他的欲望已经从清澈的灵魂溶液中析出。他的祈祷已经枝残叶败,枯木般从口中伸出。他的心打翻流光,尽失于晦暝之中。鞭笞在身上无力得如同赶苍蝇的尾巴。\footnote{\bi{布里格手记} \regular{里尔克}}

\title{\date[d=2,m=2,y=2024][year:cn-y,年,month:cn,day:cn,日,·,weekday]·腊月廿三 }
没有说话,便已喘息。\footnote{\bi{茨维塔耶娃诗选} \regular{茨维塔耶娃}}

\title{\date[d=3,m=2,y=2024][year:cn-y,年,month:cn,day:cn,日,·,weekday]·腊月廿四 }
我的梦正梦见另一个梦呢\footnote{\bi{张枣的诗} \regular{张枣 颜炼军}}

\title{\date[d=4,m=2,y=2024][year:cn-y,年,month:cn,day:cn,日,·,weekday]·腊月廿五 ·立春}
去我们此刻的国度!去此刻的国度!去飞向火星的国度!去没有我们的国度!\footnote{\bi{茨维塔耶娃诗选} \regular{茨维塔耶娃}}

\title{\date[d=5,m=2,y=2024][year:cn-y,年,month:cn,day:cn,日,·,weekday]·腊月廿六 }
在观念中露宿\footnote{\bi{不安之书} \regular{费尔南多·佩索阿 热罗尼莫·皮萨罗 }}

\title{\date[d=6,m=2,y=2024][year:cn-y,年,month:cn,day:cn,日,·,weekday]·腊月廿七 }
我将把所有的日子都给你带去\footnote{\bi{芒克的诗} \regular{芒克}}

\title{\date[d=7,m=2,y=2024][year:cn-y,年,month:cn,day:cn,日,·,weekday]·腊月廿八 }
我看见你坐在一万双眼睛里抽泣,发愣\footnote{\bi{张枣的诗} \regular{张枣 颜炼军}}

\title{\date[d=8,m=2,y=2024][year:cn-y,年,month:cn,day:cn,日,·,weekday]·腊月廿九 }
我只不过是以飞行为生\footnote{\bi{骆一禾的诗} \regular{骆一禾 西渡}}

\title{\date[d=9,m=2,y=2024][year:cn-y,年,month:cn,day:cn,日,·,weekday]·腊月三十 ·除夕}
等着,我将被你们——梦见\footnote{\bi{茨维塔耶娃诗选} \regular{茨维塔耶娃}}

\title{\date[d=10,m=2,y=2024][year:cn-y,年,month:cn,day:cn,日,·,weekday]·正月初一 ·春节}
我们就是我们的黑暗\footnote{\bi{芒克的诗} \regular{芒克}}

\title{\date[d=11,m=2,y=2024][year:cn-y,年,month:cn,day:cn,日,·,weekday]·正月初二 }
我总感到有个张开了口的深渊在我的灵魂深处不断扩大;我的心就是这个深渊!\footnote{\bi{恶之花} \regular{波德莱尔}}

\title{\date[d=12,m=2,y=2024][year:cn-y,年,month:cn,day:cn,日,·,weekday]·正月初三 }
每行诗都是爱情的孩子,是乞讨的私生子。\footnote{\bi{茨维塔耶娃诗选} \regular{茨维塔耶娃}}

\title{\date[d=13,m=2,y=2024][year:cn-y,年,month:cn,day:cn,日,·,weekday]·正月初四 }
而我这整个世界,满是彼此相异的人,好像一个多样但更紧凑的群体,它投出唯一的一个影子——这安静的、书写着的身体,我站着它倾斜,对着博尔热斯的高写字台,我过来拿我借给他的吸墨水器。\footnote{\bi{不安之书} \regular{费尔南多·佩索阿 热罗尼莫·皮萨罗 }}

\title{\date[d=14,m=2,y=2024][year:cn-y,年,month:cn,day:cn,日,·,weekday]·正月初五 }
我想要暗淡、明朗、改观。想要别人心灵和自己心灵的远角。想要你永远也听不到、永远也不会说的那些话。想要不同寻常的东西。想要神奇的东西。想要奇迹。\footnote{\bi{抒情诗的呼吸:一九二六年书信(帕斯捷尔纳克作品系列)} \regular{鲍·列·帕斯捷尔纳克 玛·伊·茨维塔耶娃 莱·马·里尔克}}

\title{\date[d=15,m=2,y=2024][year:cn-y,年,month:cn,day:cn,日,·,weekday]·正月初六 }
容得下多少小溪的混浊\footnote{\bi{海子的诗} \regular{海子}}

\title{\date[d=16,m=2,y=2024][year:cn-y,年,month:cn,day:cn,日,·,weekday]·正月初七 }
让它们从今天消失\footnote{\bi{顾城的诗} \regular{顾城}}

\title{\date[d=17,m=2,y=2024][year:cn-y,年,month:cn,day:cn,日,·,weekday]·正月初八 }
空间无穷无尽地散发并在群山之间显现出来,\footnote{\bi{谁此时孤独:里尔克晚期书信选} \regular{里尔克}}

\title{\date[d=18,m=2,y=2024][year:cn-y,年,month:cn,day:cn,日,·,weekday]·正月初九 }
你的记忆和你的感觉将是你创造性冲动的食粮\footnote{\bi{彩画集:兰波散文诗全集(译文经典)} \regular{阿蒂尔·兰波}}

\title{\date[d=19,m=2,y=2024][year:cn-y,年,month:cn,day:cn,日,·,weekday]·正月初十 ·雨水}
一一成熟为芒种的句子秋分的句子和惊蛰的句子\footnote{\bi{骆一禾的诗} \regular{骆一禾 西渡}}

\title{\date[d=20,m=2,y=2024][year:cn-y,年,month:cn,day:cn,日,·,weekday]·正月十一 }
疲惫的人是逐步沉向孤独的动物。\footnote{\bi{昌耀的诗} \regular{昌耀}}

\title{\date[d=21,m=2,y=2024][year:cn-y,年,month:cn,day:cn,日,·,weekday]·正月十二 }
我还将在这同样的宿命指引下耗尽不长的余生\footnote{\bi{戈麦的诗} \regular{戈麦 西渡}}

\title{\date[d=22,m=2,y=2024][year:cn-y,年,month:cn,day:cn,日,·,weekday]·正月十三 }
成为自己身上的死者\footnote{\bi{欧阳江河的诗} \regular{欧阳江河}}

\title{\date[d=23,m=2,y=2024][year:cn-y,年,month:cn,day:cn,日,·,weekday]·正月十四 }
我在想所有的诗句,\footnote{\bi{茨维塔耶娃诗选} \regular{茨维塔耶娃}}

\title{\date[d=24,m=2,y=2024][year:cn-y,年,month:cn,day:cn,日,·,weekday]·正月十五 ·元宵节}
酒回到粮食,秋天,空的杯盏。\footnote{\bi{欧阳江河的诗} \regular{欧阳江河}}

\title{\date[d=25,m=2,y=2024][year:cn-y,年,month:cn,day:cn,日,·,weekday]·正月十六 }
是虚假生活的碎片,被遥远的死亡染成金色,带着它有着全部真理的悲伤微笑。\footnote{\bi{不安之书} \regular{费尔南多·佩索阿 热罗尼莫·皮萨罗 }}

\title{\date[d=26,m=2,y=2024][year:cn-y,年,month:cn,day:cn,日,·,weekday]·正月十七 }
有时候他们在书页间移动,就像睡着的人在两个梦境之间辗转。停留在阅读的人群中多好啊。\footnote{\bi{布里格手记} \regular{里尔克}}

\title{\date[d=27,m=2,y=2024][year:cn-y,年,month:cn,day:cn,日,·,weekday]·正月十八 }
但愿‘往昔’能够转为‘今日’,我即无需等待‘未来’。时光怎可顷刻而返,未来时日又会如何?\footnote{\bi{堂吉诃德(译文名著精选)} \regular{塞万提斯}}

\title{\date[d=28,m=2,y=2024][year:cn-y,年,month:cn,day:cn,日,·,weekday]·正月十九 }
它就像钟表那种稳定而不知疲倦的滴答声;[226]只要它响起来,静寂就似乎有了一种低沉的共鸣:听得到每一个音节的来和去\footnote{\bi{布里格手记} \regular{里尔克}}

\title{\date[d=29,m=2,y=2024][year:cn-y,年,month:cn,day:cn,日,·,weekday]·正月二十 }
我的诗歌仅剩下消匿之后的痕迹一行行隐去,透彻但不清晰\footnote{\bi{戈麦的诗} \regular{戈麦 西渡}}

\title{\date[d=1,m=3,y=2024][year:cn-y,年,month:cn,day:cn,日,·,weekday]·正月廿一 }
像一笔坚硬的债,我要用全部生命偿还\footnote{\bi{戈麦的诗} \regular{戈麦 西渡}}

\title{\date[d=2,m=3,y=2024][year:cn-y,年,month:cn,day:cn,日,·,weekday]·正月廿二 }
机器的脸\footnote{\bi{顾城的诗} \regular{顾城}}

\title{\date[d=3,m=3,y=2024][year:cn-y,年,month:cn,day:cn,日,·,weekday]·正月廿三 }
那风没有手却能将它们打散,是它们为我们、在我们身上形成了宇宙那被感知的体系。\footnote{\bi{不安之书} \regular{费尔南多·佩索阿 热罗尼莫·皮萨罗 }}

\title{\date[d=4,m=3,y=2024][year:cn-y,年,month:cn,day:cn,日,·,weekday]·正月廿四 }
黏滞的风径自穿堂而过\footnote{\bi{戈麦的诗} \regular{戈麦 西渡}}

\title{\date[d=5,m=3,y=2024][year:cn-y,年,month:cn,day:cn,日,·,weekday]·正月廿五 ·惊蛰}
这是夜雨的杯盏\footnote{\bi{西川的诗} \regular{西川}}

\title{\date[d=6,m=3,y=2024][year:cn-y,年,month:cn,day:cn,日,·,weekday]·正月廿六 }
时间的恐怖,因为它在流逝,生命的恐怖,因为它只会杀戮。\footnote{\bi{宇宙重建了自身:佩索阿诗精选} \regular{费尔南多·佩索阿}}

\title{\date[d=7,m=3,y=2024][year:cn-y,年,month:cn,day:cn,日,·,weekday]·正月廿七 }
最大的狂热者也是最大的怀疑者\footnote{\bi{沉石与火舌:特朗斯特罗姆诗全集} \regular{托马斯·特朗斯特罗姆}}

\title{\date[d=8,m=3,y=2024][year:cn-y,年,month:cn,day:cn,日,·,weekday]·正月廿八 }
事物的充盈贯穿我空气般透澈的心,仅仅观望就足够温情。我从来都只是无形的幻梦,褪去全部的灵魂,空留一缕倏来忽往的漂泊的微风。\footnote{\bi{不安之书} \regular{费尔南多·佩索阿 热罗尼莫·皮萨罗 }}

\title{\date[d=9,m=3,y=2024][year:cn-y,年,month:cn,day:cn,日,·,weekday]·正月廿九 }
因为现在是所有的过去和所有的未来:\footnote{\bi{宇宙重建了自身:佩索阿诗精选} \regular{费尔南多·佩索阿}}

\title{\date[d=10,m=3,y=2024][year:cn-y,年,month:cn,day:cn,日,·,weekday]·二月初一 }
风吹过我 吹过千千万万山岗太阳失色 鹰翻落 山不动\footnote{\bi{于坚的诗} \regular{于坚}}

\title{\date[d=11,m=3,y=2024][year:cn-y,年,month:cn,day:cn,日,·,weekday]·二月初二 ·龙抬头}
较之于一切虚构的和可虚构的事物,“真实的事物”恰恰更难描述\footnote{\bi{谁此时孤独:里尔克晚期书信选} \regular{里尔克}}

\title{\date[d=12,m=3,y=2024][year:cn-y,年,month:cn,day:cn,日,·,weekday]·二月初三 }
不管是在床铺上抑或人群中,我的灵魂始终如一,痛楚地对世界保持清醒的意识。白日如幸福般迟迟未至,那个时刻,天亮似乎遥遥无期。\footnote{\bi{不安之书} \regular{费尔南多·佩索阿 热罗尼莫·皮萨罗 }}

\title{\date[d=13,m=3,y=2024][year:cn-y,年,month:cn,day:cn,日,·,weekday]·二月初四 }
诗是对事物的感受,不是再认识,而是幻想。一首诗是我让它醒着的梦。诗最重要的任务是塑造精神生活,揭示神秘。\footnote{\bi{沉石与火舌:特朗斯特罗姆诗全集} \regular{托马斯·特朗斯特罗姆}}

\title{\date[d=14,m=3,y=2024][year:cn-y,年,month:cn,day:cn,日,·,weekday]·二月初五 }
你要坦白承认,万一你写不出来,是不是必得因此而死去\footnote{\bi{给青年诗人的信} \regular{莱内·马利亚·里尔克}}

\title{\date[d=15,m=3,y=2024][year:cn-y,年,month:cn,day:cn,日,·,weekday]·二月初六 }
一团水中的火焰在夜色中被点燃\footnote{\bi{戈麦的诗} \regular{戈麦 西渡}}

\title{\date[d=16,m=3,y=2024][year:cn-y,年,month:cn,day:cn,日,·,weekday]·二月初七 }
极乐即太多的痛苦。\footnote{\bi{荒原:艾略特文集·诗歌} \regular{T.S.艾略特}}

\title{\date[d=17,m=3,y=2024][year:cn-y,年,month:cn,day:cn,日,·,weekday]·二月初八 }
碎纸迷惘如抚摸\footnote{\bi{张枣的诗} \regular{张枣 颜炼军}}

\title{\date[d=18,m=3,y=2024][year:cn-y,年,month:cn,day:cn,日,·,weekday]·二月初九 }
给我狭窄的心/一个大的宇宙\footnote{\bi{给青年诗人的信} \regular{莱内·马利亚·里尔克}}

\title{\date[d=19,m=3,y=2024][year:cn-y,年,month:cn,day:cn,日,·,weekday]·二月初十 }
幕的后面,神在打牌\footnote{\bi{戈麦的诗} \regular{戈麦 西渡}}

\title{\date[d=20,m=3,y=2024][year:cn-y,年,month:cn,day:cn,日,·,weekday]·二月十一 ·春分}
我就将是抵达那里的第一封信\footnote{\bi{抒情诗的呼吸:一九二六年书信(帕斯捷尔纳克作品系列)} \regular{鲍·列·帕斯捷尔纳克 玛·伊·茨维塔耶娃 莱·马·里尔克}}

\title{\date[d=21,m=3,y=2024][year:cn-y,年,month:cn,day:cn,日,·,weekday]·二月十二 }
我大概还有一段路程要跋涉\footnote{\bi{彩画集:兰波散文诗全集(译文经典)} \regular{阿蒂尔·兰波}}

\title{\date[d=22,m=3,y=2024][year:cn-y,年,month:cn,day:cn,日,·,weekday]·二月十三 }
向你注入我的毒液,啊,我的妹妹!\footnote{\bi{恶之花} \regular{波德莱尔}}

\title{\date[d=23,m=3,y=2024][year:cn-y,年,month:cn,day:cn,日,·,weekday]·二月十四 }
黑夜修女熬制的硫酸\footnote{\bi{于坚的诗} \regular{于坚}}

\title{\date[d=24,m=3,y=2024][year:cn-y,年,month:cn,day:cn,日,·,weekday]·二月十五 }
那里海浪相遇海浪。\footnote{\bi{荒原:艾略特文集·诗歌} \regular{T.S.艾略特}}

\title{\date[d=25,m=3,y=2024][year:cn-y,年,month:cn,day:cn,日,·,weekday]·二月十六 }
我痛恨挪动线条人造的栩栩如生,我永远也不会满面泪痕,永远也不会满面笑容。\footnote{\bi{恶之花} \regular{波德莱尔}}

\title{\date[d=26,m=3,y=2024][year:cn-y,年,month:cn,day:cn,日,·,weekday]·二月十七 }
取消了时间\footnote{\bi{王家新的诗} \regular{王家新}}

\title{\date[d=27,m=3,y=2024][year:cn-y,年,month:cn,day:cn,日,·,weekday]·二月十八 }
给风的预言,只给风,因为只有风会倾听\footnote{\bi{荒原:艾略特文集·诗歌} \regular{T.S.艾略特}}

\title{\date[d=28,m=3,y=2024][year:cn-y,年,month:cn,day:cn,日,·,weekday]·二月十九 }
现在(是的,我应如何描述?),现在一片静寂。静得仿佛疼痛停止了。一种感觉特殊的、发痒的静寂,仿佛伤口在愈合。\footnote{\bi{布里格手记} \regular{里尔克}}

\title{\date[d=29,m=3,y=2024][year:cn-y,年,month:cn,day:cn,日,·,weekday]·二月二十 }
要实现梦,就得将它忘记,转移注意力。所以,实现的方法就是不去实现。生活总是自相矛盾,\footnote{\bi{不安之书} \regular{费尔南多·佩索阿 热罗尼莫·皮萨罗 }}

\title{\date[d=30,m=3,y=2024][year:cn-y,年,month:cn,day:cn,日,·,weekday]·二月廿一 }
但我想和你相会,我想和你一起说说话。\footnote{\bi{想象一朵未来的玫瑰:佩索阿诗选} \regular{费尔南多·佩索阿}}

\title{\date[d=31,m=3,y=2024][year:cn-y,年,month:cn,day:cn,日,·,weekday]·二月廿二 }
一本黑夜的小说,白天的读者翻得很慢\footnote{\bi{戈麦的诗} \regular{戈麦 西渡}}

\title{\date[d=1,m=4,y=2024][year:cn-y,年,month:cn,day:cn,日,·,weekday]·二月廿三 }
蜃气飘摇的地表\footnote{\bi{昌耀的诗} \regular{昌耀}}

\title{\date[d=2,m=4,y=2024][year:cn-y,年,month:cn,day:cn,日,·,weekday]·二月廿四 }
横渡时间之海的美丽航船。\footnote{\bi{金阁寺} \regular{三岛由纪夫}}

\title{\date[d=3,m=4,y=2024][year:cn-y,年,month:cn,day:cn,日,·,weekday]·二月廿五 }
他们求助于自然,通过寻找自然,他们找寻着自己。\footnote{\bi{里尔克全集 第九卷 沃普斯韦德、奥古斯特·罗丹} \regular{莱纳.马利亚.里克尔 叶廷芳}}

\title{\date[d=4,m=4,y=2024][year:cn-y,年,month:cn,day:cn,日,·,weekday]·二月廿六 ·清明}
啊,而夜,夜,当盈满了世界空间的风削损着我们的脸庞——,这思慕着的,柔顺地幻灭的夜,在孤单的心怀之前艰苦地现身的夜,不为谁而伫留吧。\footnote{\bi{杜英诺悲歌:里尔克诗选(文学馆系列)} \regular{里尔克}}

\title{\date[d=5,m=4,y=2024][year:cn-y,年,month:cn,day:cn,日,·,weekday]·二月廿七 }
于是你的苦痛与辛酸终成眷属\footnote{\bi{不安之书} \regular{费尔南多·佩索阿 热罗尼莫·皮萨罗 }}

\title{\date[d=6,m=4,y=2024][year:cn-y,年,month:cn,day:cn,日,·,weekday]·二月廿八 }
我多么熟悉那个世界啊!是根据梦,根据梦的空气,根据梦的紊乱性和迫切性熟悉它的。我对此岂能不知道,对此岂能不喜爱,我在其中又遭受了多少委屈!那个世界,你只要知道:亮光,照明,被你我之光异样照亮的事物。\footnote{\bi{抒情诗的呼吸:一九二六年书信(帕斯捷尔纳克作品系列)} \regular{鲍·列·帕斯捷尔纳克 玛·伊·茨维塔耶娃 莱·马·里尔克}}

\title{\date[d=7,m=4,y=2024][year:cn-y,年,month:cn,day:cn,日,·,weekday]·二月廿九 }
少女无邪的手心在苍龙湮灭的寰球上平伸\footnote{\bi{骆一禾的诗} \regular{骆一禾 西渡}}

\title{\date[d=8,m=4,y=2024][year:cn-y,年,month:cn,day:cn,日,·,weekday]·二月三十 }
但是在我们感情中有一种提前悲伤的痕迹,一种穿上旅服的伤痛,我们模糊地注意到万物缤纷地展开,风有着另一种声调,如果夜晚降临,古老的安宁会沿着宇宙那不可回避的存在而延伸。\footnote{\bi{不安之书} \regular{费尔南多·佩索阿 热罗尼莫·皮萨罗 }}

\title{\date[d=9,m=4,y=2024][year:cn-y,年,month:cn,day:cn,日,·,weekday]·三月初一 }
半坡的鸟道、\footnote{\bi{昌耀的诗} \regular{昌耀}}

\title{\date[d=10,m=4,y=2024][year:cn-y,年,month:cn,day:cn,日,·,weekday]·三月初二 }
你发现了吗,我是在零星地把自己给你?\footnote{\bi{抒情诗的呼吸:一九二六年书信(帕斯捷尔纳克作品系列)} \regular{鲍·列·帕斯捷尔纳克 玛·伊·茨维塔耶娃 莱·马·里尔克}}

\title{\date[d=11,m=4,y=2024][year:cn-y,年,month:cn,day:cn,日,·,weekday]·三月初三 }
可大地仍是宇宙娇娆而失手的镜子。\footnote{\bi{张枣的诗} \regular{张枣 颜炼军}}

\title{\date[d=12,m=4,y=2024][year:cn-y,年,month:cn,day:cn,日,·,weekday]·三月初四 }
那个夏天还在拖延那个声音已经停止\footnote{\bi{顾城的诗} \regular{顾城}}

\title{\date[d=13,m=4,y=2024][year:cn-y,年,month:cn,day:cn,日,·,weekday]·三月初五 }
会写作的人知道怎样透彻地看清自己的梦(的确是这样),或在梦中观望生活,把生活视为虚无缥缈的存在,用白日梦的照相机为其摄影\footnote{\bi{不安之书} \regular{费尔南多·佩索阿 热罗尼莫·皮萨罗 }}

\title{\date[d=14,m=4,y=2024][year:cn-y,年,month:cn,day:cn,日,·,weekday]·三月初六 }
演奏者成就的短时间的美,把时间变成了纯粹的延续,无法重复不能回头,生命像蜉蝣一样短暂的同时,也幻化为完完全全的抽象和创造。没有什么比音乐更像生命\footnote{\bi{金阁寺} \regular{三岛由纪夫}}

\title{\date[d=15,m=4,y=2024][year:cn-y,年,month:cn,day:cn,日,·,weekday]·三月初七 }
如果我们无法通过自己敏感的想象力成为他者的话,我们永远无法到达另一个人。\footnote{\bi{不安之书} \regular{费尔南多·佩索阿 热罗尼莫·皮萨罗 }}

\title{\date[d=16,m=4,y=2024][year:cn-y,年,month:cn,day:cn,日,·,weekday]·三月初八 }
大太阳\footnote{\bi{骆一禾的诗} \regular{骆一禾 西渡}}

\title{\date[d=17,m=4,y=2024][year:cn-y,年,month:cn,day:cn,日,·,weekday]·三月初九 }
如果回忆很多,我们必须能够忘记\footnote{\bi{给青年诗人的信} \regular{莱内·马利亚·里尔克}}

\title{\date[d=18,m=4,y=2024][year:cn-y,年,month:cn,day:cn,日,·,weekday]·三月初十 }
就让上帝送给你百合,但别送给你此刻,\footnote{\bi{我的心迟到了:佩索阿情诗} \regular{费尔南多·佩索阿}}

\title{\date[d=19,m=4,y=2024][year:cn-y,年,month:cn,day:cn,日,·,weekday]·三月十一 ·谷雨}
至善需要耐心,旷远依赖于时间的丈量\footnote{\bi{戈麦的诗} \regular{戈麦 西渡}}

\title{\date[d=20,m=4,y=2024][year:cn-y,年,month:cn,day:cn,日,·,weekday]·三月十二 }
丑是命运刻好的印章,把你的灵魂抵押给孤寂。\footnote{\bi{我的心迟到了:佩索阿情诗} \regular{费尔南多·佩索阿}}

\title{\date[d=21,m=4,y=2024][year:cn-y,年,month:cn,day:cn,日,·,weekday]·三月十三 }
我过去的生活混杂了将来的生活\footnote{\bi{想象一朵未来的玫瑰:佩索阿诗选} \regular{费尔南多·佩索阿}}

\title{\date[d=22,m=4,y=2024][year:cn-y,年,month:cn,day:cn,日,·,weekday]·三月十四 }
以画笔或铅笔去拥抱,去痴情占有\footnote{\bi{谁此时孤独:里尔克晚期书信选} \regular{里尔克}}

\title{\date[d=23,m=4,y=2024][year:cn-y,年,month:cn,day:cn,日,·,weekday]·三月十五 }
她不考虑过去,她在创造自己心目中的过去,\footnote{\bi{抒情诗的呼吸:一九二六年书信(帕斯捷尔纳克作品系列)} \regular{鲍·列·帕斯捷尔纳克 玛·伊·茨维塔耶娃 莱·马·里尔克}}

\title{\date[d=24,m=4,y=2024][year:cn-y,年,month:cn,day:cn,日,·,weekday]·三月十六 }
最重要的是梦见一切,以便将其转化为我们最深处的本质\footnote{\bi{不安之书} \regular{费尔南多·佩索阿 热罗尼莫·皮萨罗 }}

\title{\date[d=25,m=4,y=2024][year:cn-y,年,month:cn,day:cn,日,·,weekday]·三月十七 }
它们聚集在体内,成为一种没有生活过、被摈斥、被遗弃的生命,能以使我们死去\footnote{\bi{给青年诗人的信} \regular{莱内·马利亚·里尔克}}

\title{\date[d=26,m=4,y=2024][year:cn-y,年,month:cn,day:cn,日,·,weekday]·三月十八 }
我之愀然是为心作,声闻旷远。\footnote{\bi{昌耀的诗} \regular{昌耀}}

\title{\date[d=27,m=4,y=2024][year:cn-y,年,month:cn,day:cn,日,·,weekday]·三月十九 }
锯齿形的烟\footnote{\bi{顾城的诗} \regular{顾城}}

\title{\date[d=28,m=4,y=2024][year:cn-y,年,month:cn,day:cn,日,·,weekday]·三月二十 }
谁与我同享暮色的金黄然后一起退入月亮宝石?\footnote{\bi{昌耀的诗} \regular{昌耀}}

\title{\date[d=29,m=4,y=2024][year:cn-y,年,month:cn,day:cn,日,·,weekday]·三月廿一 }
她望着我 永远也不离开永远也不走近\footnote{\bi{于坚的诗} \regular{于坚}}

\title{\date[d=30,m=4,y=2024][year:cn-y,年,month:cn,day:cn,日,·,weekday]·三月廿二 }
我将成为鹿,或指鹿为马\footnote{\bi{戈麦的诗} \regular{戈麦 西渡}}

\title{\date[d=1,m=5,y=2024][year:cn-y,年,month:cn,day:cn,日,·,weekday]·三月廿三 }
因为整个天空都是泪水\footnote{\bi{欧阳江河的诗} \regular{欧阳江河}}

\title{\date[d=2,m=5,y=2024][year:cn-y,年,month:cn,day:cn,日,·,weekday]·三月廿四 }
欲留在鸡巴之前\footnote{\bi{西川的诗} \regular{西川}}

\title{\date[d=3,m=5,y=2024][year:cn-y,年,month:cn,day:cn,日,·,weekday]·三月廿五 }
生活中没有初学者的班级,它总是要求马上就开始最难的。\footnote{\bi{布里格手记} \regular{里尔克}}

\title{\date[d=4,m=5,y=2024][year:cn-y,年,month:cn,day:cn,日,·,weekday]·三月廿六 }
我正推着太阳车轮\footnote{\bi{骆一禾的诗} \regular{骆一禾 西渡}}

\title{\date[d=5,m=5,y=2024][year:cn-y,年,month:cn,day:cn,日,·,weekday]·三月廿七 ·立夏}
最好的箭,全都是盲目的!\footnote{\bi{抒情诗的呼吸:一九二六年书信(帕斯捷尔纳克作品系列)} \regular{鲍·列·帕斯捷尔纳克 玛·伊·茨维塔耶娃 莱·马·里尔克}}

\title{\date[d=6,m=5,y=2024][year:cn-y,年,month:cn,day:cn,日,·,weekday]·三月廿八 }
而我们无有归去的路。而我们只可前行。而我们无可回归。\footnote{\bi{昌耀的诗} \regular{昌耀}}

\title{\date[d=7,m=5,y=2024][year:cn-y,年,month:cn,day:cn,日,·,weekday]·三月廿九 }
。如果有一种方法可以活过所有的生命和所有的时代所有形式的形式与所有姿态的姿态就好了!\footnote{\bi{宇宙重建了自身:佩索阿诗精选} \regular{费尔南多·佩索阿}}

\title{\date[d=8,m=5,y=2024][year:cn-y,年,month:cn,day:cn,日,·,weekday]·四月初一 }
我学着看。\footnote{\bi{布里格手记} \regular{里尔克}}

\title{\date[d=9,m=5,y=2024][year:cn-y,年,month:cn,day:cn,日,·,weekday]·四月初二 }
我在绝望中期待,我的离去就是留下\footnote{\bi{堂吉诃德(译文名著精选)} \regular{塞万提斯}}

\title{\date[d=10,m=5,y=2024][year:cn-y,年,month:cn,day:cn,日,·,weekday]·四月初三 }
而风景中的人,对我才产生意义\footnote{\bi{沉石与火舌:特朗斯特罗姆诗全集} \regular{托马斯·特朗斯特罗姆}}

\title{\date[d=11,m=5,y=2024][year:cn-y,年,month:cn,day:cn,日,·,weekday]·四月初四 }
一切都和我们不同,这就是万物存在的理由。\footnote{\bi{自决之书} \regular{费尔南多·佩索阿}}

\title{\date[d=12,m=5,y=2024][year:cn-y,年,month:cn,day:cn,日,·,weekday]·四月初五 }
一种对虚无的乡愁,对某种模糊事物的渴望。\footnote{\bi{宇宙重建了自身:佩索阿诗精选} \regular{费尔南多·佩索阿}}

\title{\date[d=13,m=5,y=2024][year:cn-y,年,month:cn,day:cn,日,·,weekday]·四月初六 }
在雹霰雷殛灾变的绝域长不高大的乔木屈曲天边\footnote{\bi{昌耀的诗} \regular{昌耀}}

\title{\date[d=14,m=5,y=2024][year:cn-y,年,month:cn,day:cn,日,·,weekday]·四月初七 }
雕像披着黄昏像披着自己的肺腑\footnote{\bi{张枣的诗} \regular{张枣 颜炼军}}

\title{\date[d=15,m=5,y=2024][year:cn-y,年,month:cn,day:cn,日,·,weekday]·四月初八 }
诗人必须有一种超自然的清醒明悟,并系统地有目的地培养他的特殊感觉(Sensations),通过打乱他的感觉意识,以求发现人类的命运。求助于毒品(drogue),疾病(maladie),罪恶(crime),目的是培育自身所有珍奇的感觉和幻觉,即不曾想象得到的那种形象。\footnote{\bi{彩画集:兰波散文诗全集(译文经典)} \regular{阿蒂尔·兰波}}

\title{\date[d=16,m=5,y=2024][year:cn-y,年,month:cn,day:cn,日,·,weekday]·四月初九 }
人们看到的我,和我自己认为的我,哪个更长久呢?\footnote{\bi{金阁寺} \regular{三岛由纪夫}}

\title{\date[d=17,m=5,y=2024][year:cn-y,年,month:cn,day:cn,日,·,weekday]·四月初十 }
缱绻于忧怀\footnote{\bi{昌耀的诗} \regular{昌耀}}

\title{\date[d=18,m=5,y=2024][year:cn-y,年,month:cn,day:cn,日,·,weekday]·四月十一 }
你的两只翅膀向往天空,因为世界是你的摇篮,坟墓是世界!\footnote{\bi{茨维塔耶娃诗选} \regular{茨维塔耶娃}}

\title{\date[d=19,m=5,y=2024][year:cn-y,年,month:cn,day:cn,日,·,weekday]·四月十二 }
原来我的手臂是凉的\footnote{\bi{西川的诗} \regular{西川}}

\title{\date[d=20,m=5,y=2024][year:cn-y,年,month:cn,day:cn,日,·,weekday]·四月十三 ·小满}
我全然是一种模糊的怀想,不是对过去,也不是对未来:我是一种对当前的怀想,匿名、冗长而又不被理解。\footnote{\bi{不安之书} \regular{费尔南多·佩索阿 热罗尼莫·皮萨罗 }}

\title{\date[d=21,m=5,y=2024][year:cn-y,年,month:cn,day:cn,日,·,weekday]·四月十四 }
我的诗源于我的生活不可遏止的爆裂。\footnote{\bi{宇宙重建了自身:佩索阿诗精选} \regular{费尔南多·佩索阿}}

\title{\date[d=22,m=5,y=2024][year:cn-y,年,month:cn,day:cn,日,·,weekday]·四月十五 }
因为村庄很小,所以在那里能比在城市看到更多的世界;因而村庄比城市更大……因为我不是和我的身量,而是和我看到的事物一样大。\footnote{\bi{不安之书} \regular{费尔南多·佩索阿 热罗尼莫·皮萨罗 }}

\title{\date[d=23,m=5,y=2024][year:cn-y,年,month:cn,day:cn,日,·,weekday]·四月十六 }
我全身心地投入所有的生活,\footnote{\bi{宇宙重建了自身:佩索阿诗精选} \regular{费尔南多·佩索阿}}

\title{\date[d=24,m=5,y=2024][year:cn-y,年,month:cn,day:cn,日,·,weekday]·四月十七 }
假如雨是朝天空那个方向下就好啦 意味着一种拯救\footnote{\bi{于坚的诗} \regular{于坚}}

\title{\date[d=25,m=5,y=2024][year:cn-y,年,month:cn,day:cn,日,·,weekday]·四月十八 }
学诗的尽头是火红的窑火\footnote{\bi{骆一禾的诗} \regular{骆一禾 西渡}}

\title{\date[d=26,m=5,y=2024][year:cn-y,年,month:cn,day:cn,日,·,weekday]·四月十九 }
他身后一无所有,万事万物都在他的前方:“世界很辽阔。”\footnote{\bi{里尔克全集 第九卷 沃普斯韦德、奥古斯特·罗丹} \regular{莱纳.马利亚.里克尔 叶廷芳}}

\title{\date[d=27,m=5,y=2024][year:cn-y,年,month:cn,day:cn,日,·,weekday]·四月二十 }
我想,我得开始做点什么了[插图]。现在,我学着看。我28岁,几乎一事无成。再说一次:我写过一篇关于卡巴乔[插图]的糟糕论文,一部叫做《婚姻》的戏,想用双关法表现某种虚伪的东西,还有诗。可太早动笔,写不出什么诗来。[20]得等,一辈子都要去搜集意义和甜蜜,也许那会是漫长的一生,然后,在尽头,或许能写出十行好诗。诗并不像人们所想的那样,不是感觉[插图](很早的时候感觉就够多了)——而是经验。为写一句诗,得见过许多城市,许多人和物,要认识动物,要感受鸟儿如何飞翔,要知道小小的花朵以怎样的姿态在清晨开放。要能想起无名之地的路,想起未料到的相遇和眼见其缓缓而至的离别——要想起尚混沌的童年,想起受伤害的父母,他们想让你快乐,你却不理解他们(那是另一个人的快乐)[插图],想起孩子的病,它莫名地出现,有过那么多次深重而艰难的转变,想起那些静寂、压抑的小屋里的日子和海边的清晨,尤其是那片海[插图]。要想起海,想起低啸而过、随繁星飞走的旅夜——想起这一切,却还不够。\footnote{\bi{布里格手记} \regular{里尔克}}

\title{\date[d=28,m=5,y=2024][year:cn-y,年,month:cn,day:cn,日,·,weekday]·四月廿一 }
它们不是我的想法,而是从我脑海中经过的想法。我没有沉思;我是在做梦;我没有灵感;我只是胡言乱语。我会画画,可我从没画过;我会谱曲,可我从未谱过曲\footnote{\bi{自决之书} \regular{费尔南多·佩索阿}}

\title{\date[d=29,m=5,y=2024][year:cn-y,年,month:cn,day:cn,日,·,weekday]·四月廿二 }
您不是我最喜爱的诗人(“最喜爱的”是一个程度),您是大自然的一个现象,这一现象不可能是我的,也无法去爱它,而只能用全部身心去感觉它,或者(还没完呀!)您就是第五元素的化身:即诗的本身,或者(还没完)您就是诗从其中诞生出来的那种东西,是大于诗歌本身——即您大于自身的那种东西。\footnote{\bi{抒情诗的呼吸:一九二六年书信(帕斯捷尔纳克作品系列)} \regular{鲍·列·帕斯捷尔纳克 玛·伊·茨维塔耶娃 莱·马·里尔克}}

\title{\date[d=30,m=5,y=2024][year:cn-y,年,month:cn,day:cn,日,·,weekday]·四月廿三 }
“这世界上所有的诗行都是同一只手写出来的!”\footnote{\bi{多多的诗} \regular{多多}}

\title{\date[d=31,m=5,y=2024][year:cn-y,年,month:cn,day:cn,日,·,weekday]·四月廿四 }
大水\footnote{\bi{骆一禾的诗} \regular{骆一禾 西渡}}

\title{\date[d=1,m=6,y=2024][year:cn-y,年,month:cn,day:cn,日,·,weekday]·四月廿五 }
现在是明天的前夜。\footnote{\bi{彩画集:兰波散文诗全集(译文经典)} \regular{阿蒂尔·兰波}}

\title{\date[d=2,m=6,y=2024][year:cn-y,年,month:cn,day:cn,日,·,weekday]·四月廿六 }
除了那种虚无,我更喜欢成为虚无的虚无。\footnote{\bi{宇宙重建了自身:佩索阿诗精选} \regular{费尔南多·佩索阿}}

\title{\date[d=3,m=6,y=2024][year:cn-y,年,month:cn,day:cn,日,·,weekday]·四月廿七 }
突然间我独立于世。我从精神的屋顶高处看到这一切。我独立于世。看见就是身在远处。看清就是静止。分析就是成为异国人。所有人都从我身边经过,却又不碰擦到我。只有空气在我周围。我感到自己是那么孤立,以致几乎可以擦到我和西装之间的距离。我是一个孩子,拿着一盏几乎不亮的油灯,穿着睡衣走过一间巨大而荒凉的房子。围绕着我的是活的阴影——只有阴影,它们是死物和陪伴我的光明的女儿。它们环绕着我,在这里,在阳光下,却是人。\footnote{\bi{不安之书} \regular{费尔南多·佩索阿 热罗尼莫·皮萨罗 }}

\title{\date[d=4,m=6,y=2024][year:cn-y,年,month:cn,day:cn,日,·,weekday]·四月廿八 }
如果我必须做梦,那为什么不做我自己的梦呢?\footnote{\bi{自决之书} \regular{费尔南多·佩索阿}}

\title{\date[d=5,m=6,y=2024][year:cn-y,年,month:cn,day:cn,日,·,weekday]·四月廿九 ·芒种}
也许留在黑暗里更好,你那尚无界限的心会试着成为万物之心,无别而沉重\footnote{\bi{布里格手记} \regular{里尔克}}

\title{\date[d=6,m=6,y=2024][year:cn-y,年,month:cn,day:cn,日,·,weekday]·五月初一 }
我是一场壮烈的溃败\footnote{\bi{不安之书} \regular{费尔南多·佩索阿 热罗尼莫·皮萨罗 }}

\title{\date[d=7,m=6,y=2024][year:cn-y,年,month:cn,day:cn,日,·,weekday]·五月初二 }
那山是多个世界!\footnote{\bi{茨维塔耶娃诗选} \regular{茨维塔耶娃}}

\title{\date[d=8,m=6,y=2024][year:cn-y,年,month:cn,day:cn,日,·,weekday]·五月初三 }
雨像稠密的马鬃抽打眼睛。\footnote{\bi{茨维塔耶娃诗选} \regular{茨维塔耶娃}}

\title{\date[d=9,m=6,y=2024][year:cn-y,年,month:cn,day:cn,日,·,weekday]·五月初四 }
现在他太沉重,没人爱得动\footnote{\bi{布里格手记} \regular{里尔克}}

\title{\date[d=10,m=6,y=2024][year:cn-y,年,month:cn,day:cn,日,·,weekday]·五月初五 ·端午节}
我悄悄地像天体一样建构自己,并拥有我的无限。\footnote{\bi{不安之书} \regular{费尔南多·佩索阿 热罗尼莫·皮萨罗 }}

\title{\date[d=11,m=6,y=2024][year:cn-y,年,month:cn,day:cn,日,·,weekday]·五月初六 }
却宁愿期待着渐增的隔阂会在未来和解,而和解愈是遥遥无期,愈是令人着迷。\footnote{\bi{布里格手记} \regular{里尔克}}

\title{\date[d=12,m=6,y=2024][year:cn-y,年,month:cn,day:cn,日,·,weekday]·五月初七 }
落入那个世界的时光,来自另一个更为骄傲的世界,因为有更多愁苦被消解……\footnote{\bi{不安之书} \regular{费尔南多·佩索阿 热罗尼莫·皮萨罗 }}

\title{\date[d=13,m=6,y=2024][year:cn-y,年,month:cn,day:cn,日,·,weekday]·五月初八 }
我的开始之日便是我的结束之时\footnote{\bi{荒原:艾略特文集·诗歌} \regular{T.S.艾略特}}

\title{\date[d=14,m=6,y=2024][year:cn-y,年,month:cn,day:cn,日,·,weekday]·五月初九 }
永远是涵和忽 永远是光和无的通明\footnote{\bi{骆一禾的诗} \regular{骆一禾 西渡}}

\title{\date[d=15,m=6,y=2024][year:cn-y,年,month:cn,day:cn,日,·,weekday]·五月初十 }
自身\footnote{\bi{人生的智慧} \regular{叔本华}}

\title{\date[d=16,m=6,y=2024][year:cn-y,年,month:cn,day:cn,日,·,weekday]·五月十一 }
您将词的始初本义还给了词,将表达事物的始初单词(价值)还给了事物\footnote{\bi{抒情诗的呼吸:一九二六年书信(帕斯捷尔纳克作品系列)} \regular{鲍·列·帕斯捷尔纳克 玛·伊·茨维塔耶娃 莱·马·里尔克}}

\title{\date[d=17,m=6,y=2024][year:cn-y,年,month:cn,day:cn,日,·,weekday]·五月十二 }
透明者纷然破裂 但在后面 镜子立即弥合又在前方敞开 侵入者不会被划破\footnote{\bi{于坚的诗} \regular{于坚}}

\title{\date[d=18,m=6,y=2024][year:cn-y,年,month:cn,day:cn,日,·,weekday]·五月十三 }
那是一个沉重、湿润、仿佛在不断陷落的晚上;\footnote{\bi{布里格手记} \regular{里尔克}}

\title{\date[d=19,m=6,y=2024][year:cn-y,年,month:cn,day:cn,日,·,weekday]·五月十四 }
目击众神死亡的草原上野花一片远在远方的风比远方更远\footnote{\bi{海子的诗} \regular{海子}}

\title{\date[d=20,m=6,y=2024][year:cn-y,年,month:cn,day:cn,日,·,weekday]·五月十五 }
有些感受就是睡意,好像雾一样笼罩着精神的全部延展,不允许思考,也不允许行动,甚至不允许明确地存在。\footnote{\bi{不安之书} \regular{费尔南多·佩索阿 热罗尼莫·皮萨罗 }}

\title{\date[d=21,m=6,y=2024][year:cn-y,年,month:cn,day:cn,日,·,weekday]·五月十六 ·夏至}
我是一串永恒展开的图像,连贯或不连贯的\footnote{\bi{不安之书} \regular{费尔南多·佩索阿 热罗尼莫·皮萨罗 }}

\title{\date[d=22,m=6,y=2024][year:cn-y,年,month:cn,day:cn,日,·,weekday]·五月十七 }
当您爱一个人的时候,您总是想让他离开,以便去思念他。[插图]\footnote{\bi{抒情诗的呼吸:一九二六年书信(帕斯捷尔纳克作品系列)} \regular{鲍·列·帕斯捷尔纳克 玛·伊·茨维塔耶娃 莱·马·里尔克}}

\title{\date[d=23,m=6,y=2024][year:cn-y,年,month:cn,day:cn,日,·,weekday]·五月十八 }
所有一切都付托于你。可是你能胜任吗\footnote{\bi{杜英诺悲歌:里尔克诗选(文学馆系列)} \regular{里尔克}}

\title{\date[d=24,m=6,y=2024][year:cn-y,年,month:cn,day:cn,日,·,weekday]·五月十九 }
我睡,而又不眠。\footnote{\bi{不安之书} \regular{费尔南多·佩索阿 热罗尼莫·皮萨罗 }}

\title{\date[d=25,m=6,y=2024][year:cn-y,年,month:cn,day:cn,日,·,weekday]·五月二十 }
因为诗歌就是惊异、欣赏\footnote{\bi{自决之书} \regular{费尔南多·佩索阿}}

\title{\date[d=26,m=6,y=2024][year:cn-y,年,month:cn,day:cn,日,·,weekday]·五月廿一 }
我怕沉睡,好像人们怕一个不知通向什么地方、充满了隐隐约约的恐怖的幽深黑洞一样;我只看见无限展现在所有的窗外,我的灵魂,始终被眩晕所折磨,不禁嫉妒虚无的冷漠。\footnote{\bi{恶之花} \regular{波德莱尔}}

\title{\date[d=27,m=6,y=2024][year:cn-y,年,month:cn,day:cn,日,·,weekday]·五月廿二 }
你要迎着黄昏歌唱迎着黄昏歌唱你便走到黑夜的那边\footnote{\bi{骆一禾的诗} \regular{骆一禾 西渡}}

\title{\date[d=28,m=6,y=2024][year:cn-y,年,month:cn,day:cn,日,·,weekday]·五月廿三 }
这也寓意着从世界既定的秩序通往自由甚至通往官能感受的桥。\footnote{\bi{金阁寺} \regular{三岛由纪夫}}

\title{\date[d=29,m=6,y=2024][year:cn-y,年,month:cn,day:cn,日,·,weekday]·五月廿四 }
我失败于向之所是,向之所欲,向之所知。\footnote{\bi{想象一朵未来的玫瑰:佩索阿诗选} \regular{费尔南多·佩索阿}}

\title{\date[d=30,m=6,y=2024][year:cn-y,年,month:cn,day:cn,日,·,weekday]·五月廿五 }
在我的心里,你想种下什么\footnote{\bi{芒克的诗} \regular{芒克}}

\title{\date[d=1,m=7,y=2024][year:cn-y,年,month:cn,day:cn,日,·,weekday]·五月廿六 }
我喜欢沉重的地球从未在我们脚下漂移。\footnote{\bi{茨维塔耶娃诗选} \regular{茨维塔耶娃}}

\title{\date[d=2,m=7,y=2024][year:cn-y,年,month:cn,day:cn,日,·,weekday]·五月廿七 }
连接我内心和外界的这把生锈的锁,即将华丽打开。风儿将自在吹拂,内外畅行无阻;水桶如同长了翅膀一样轻盈上升;一切将像原野一样在我面前展开,尽收眼底;密室将荡然无存\footnote{\bi{金阁寺} \regular{三岛由纪夫}}

\title{\date[d=3,m=7,y=2024][year:cn-y,年,month:cn,day:cn,日,·,weekday]·五月廿八 }
追求科学需要特殊的勇敢。\footnote{\bi{伽利略传} \regular{贝托尔特·布莱希特}}

\title{\date[d=4,m=7,y=2024][year:cn-y,年,month:cn,day:cn,日,·,weekday]·五月廿九 }
,我踏上旅途的冲动一部分源自大海的暗示\footnote{\bi{金阁寺} \regular{三岛由纪夫}}

\title{\date[d=5,m=7,y=2024][year:cn-y,年,month:cn,day:cn,日,·,weekday]·五月三十 }
是对一切事物不加选择的予以观察的态度,以及对一切都发出赞叹的坚定决心\footnote{\bi{里尔克全集 第九卷 沃普斯韦德、奥古斯特·罗丹} \regular{莱纳.马利亚.里克尔 叶廷芳}}

\title{\date[d=6,m=7,y=2024][year:cn-y,年,month:cn,day:cn,日,·,weekday]·六月初一 ·小暑}
人类灵魂是一个幽深而又黏稠的深渊,一种不在世界表面使用的井\footnote{\bi{不安之书} \regular{费尔南多·佩索阿 热罗尼莫·皮萨罗 }}

\title{\date[d=7,m=7,y=2024][year:cn-y,年,month:cn,day:cn,日,·,weekday]·六月初二 }
我的声音世界无法听懂 呢喃燕语 在不可言说之中在这儿干什么我都赤身裸体 我不是希腊的男神体态臃肿 其貌不扬 可它是我的身体 惟一的身体没有镜子的房间 我是我的镜子 我是我的语法和词我是我的神 我是我梦中的野兽 布满镜子的房间人的形象就是我的形象 神的样子就是我的样子活动就是我在活动 看见就是被我看见 我看见我\footnote{\bi{于坚的诗} \regular{于坚}}

\title{\date[d=8,m=7,y=2024][year:cn-y,年,month:cn,day:cn,日,·,weekday]·六月初三 }
你听你听,我的弦断了。\footnote{\bi{何其芳散文} \regular{何其芳}}

\title{\date[d=9,m=7,y=2024][year:cn-y,年,month:cn,day:cn,日,·,weekday]·六月初四 }
天下烈酒我们对饮各半,\footnote{\bi{欧阳江河的诗} \regular{欧阳江河}}

\title{\date[d=10,m=7,y=2024][year:cn-y,年,month:cn,day:cn,日,·,weekday]·六月初五 }
你应拥有一切,但只当一无所有\footnote{\bi{堂吉诃德(译文名著精选)} \regular{塞万提斯}}

\title{\date[d=11,m=7,y=2024][year:cn-y,年,month:cn,day:cn,日,·,weekday]·六月初六 }
那么现在,我对自己说的话,你能够听到了吗?\footnote{\bi{多多的诗} \regular{多多}}

\title{\date[d=12,m=7,y=2024][year:cn-y,年,month:cn,day:cn,日,·,weekday]·六月初七 }
我为爱你而心痛过,那么不再爱你的痛也含有亲密……\footnote{\bi{我的心迟到了:佩索阿情诗} \regular{费尔南多·佩索阿}}

\title{\date[d=13,m=7,y=2024][year:cn-y,年,month:cn,day:cn,日,·,weekday]·六月初八 }
但风景在呈现动态时,并无所意欲。\footnote{\bi{里尔克全集 第九卷 沃普斯韦德、奥古斯特·罗丹} \regular{莱纳.马利亚.里克尔 叶廷芳}}

\title{\date[d=14,m=7,y=2024][year:cn-y,年,month:cn,day:cn,日,·,weekday]·六月初九 }
假如夜不仅仅是光的缺失,假如夜确实是某样东西。那么夜就是这声音\footnote{\bi{沉石与火舌:特朗斯特罗姆诗全集} \regular{托马斯·特朗斯特罗姆}}

\title{\date[d=15,m=7,y=2024][year:cn-y,年,month:cn,day:cn,日,·,weekday]·六月初十 }
我的过去也没有什么经历能引发那无用的、再来一次的渴望。我一直都是我自己的一种痕迹和模拟。我的过去就是我无法成为的一切。甚至对那些逝去时光的感受在我看来也没什么值得怀想的:能被感受的事物都要求特定的时刻;过了这个时刻,就是翻页和故事再续,但是文本并不继续。\footnote{\bi{不安之书} \regular{费尔南多·佩索阿 热罗尼莫·皮萨罗 }}

\title{\date[d=16,m=7,y=2024][year:cn-y,年,month:cn,day:cn,日,·,weekday]·六月十一 }
在两处同时放映我正在广场看上集,你却在幕间休息\footnote{\bi{顾城的诗} \regular{顾城}}

\title{\date[d=17,m=7,y=2024][year:cn-y,年,month:cn,day:cn,日,·,weekday]·六月十二 }
这便是无限的成就,而此成就可立即克服黏附于痛苦之上的一切否定因素和始终构成部分痛苦的一切惰性与屈服,这才是有为的在内部起作用的痛苦,具有意义并与我们相称的唯一痛苦。\footnote{\bi{谁此时孤独:里尔克晚期书信选} \regular{里尔克}}

\title{\date[d=18,m=7,y=2024][year:cn-y,年,month:cn,day:cn,日,·,weekday]·六月十三 }
他们只是世界的假象,他们跟我不能相比,他们更孤独,他们是不可挽救的狡猾者呵!\footnote{\bi{多多的诗} \regular{多多}}

\title{\date[d=19,m=7,y=2024][year:cn-y,年,month:cn,day:cn,日,·,weekday]·六月十四 }
我可以成为后来人的一个样子:我缺乏的正是一切失意者身上有的。那就是,他们在找不到安慰的时候,就自己安慰自己;而对我而言,找不到安慰,就会产生更大的悲痛和苦难,至死也难以消除。\footnote{\bi{堂吉诃德(译文名著精选)} \regular{塞万提斯}}

\title{\date[d=20,m=7,y=2024][year:cn-y,年,month:cn,day:cn,日,·,weekday]·六月十五 }
爬行的阴影\footnote{\bi{顾城的诗} \regular{顾城}}

\title{\date[d=21,m=7,y=2024][year:cn-y,年,month:cn,day:cn,日,·,weekday]·六月十六 }
他们精准地预感到写作永远不过是情感与梦的片刻。凡无意识写下的东西,就会接近可能的完美\footnote{\bi{不安之书} \regular{费尔南多·佩索阿 热罗尼莫·皮萨罗 }}

\title{\date[d=22,m=7,y=2024][year:cn-y,年,month:cn,day:cn,日,·,weekday]·六月十七 ·大暑}
因为语言是真正的、最有效的“鸦片”,只有借助鸦片所造成的迷幻,用词汇来固定感觉和想象的诗人才是真正的诗人。动词和想象的结合可以产生无穷的力量,这种想象不是使现实具象化,而是集合起散乱的断片,重新建立秩序并赋之予新意。它是创造性的,也是综合性的。音乐由于单凭音响就能发出暗示,激起联想,创造出幻境,所以深得波德莱尔的青睐。他在诗中屏弃客观描写和逻辑演绎,抓住某种特殊的感觉并据此和谐地组织意象,以获得一种内在的音乐性。\footnote{\bi{恶之花} \regular{波德莱尔}}

\title{\date[d=23,m=7,y=2024][year:cn-y,年,month:cn,day:cn,日,·,weekday]·六月十八 }
夜深如深井,语言就像吊在深井里的沉重水桶,一边嘎吱作响一边往上。\footnote{\bi{金阁寺} \regular{三岛由纪夫}}

\title{\date[d=24,m=7,y=2024][year:cn-y,年,month:cn,day:cn,日,·,weekday]·六月十九 }
有着正确的疲劳,我是不安的,被困住的,充满思念的。\footnote{\bi{不安之书} \regular{费尔南多·佩索阿 热罗尼莫·皮萨罗 }}

\title{\date[d=25,m=7,y=2024][year:cn-y,年,month:cn,day:cn,日,·,weekday]·六月二十 }
为拉曼恰扬名,我听说您的祖籍和出生地是拉曼恰,对吧\footnote{\bi{堂吉诃德(译文名著精选)} \regular{塞万提斯}}

\title{\date[d=26,m=7,y=2024][year:cn-y,年,month:cn,day:cn,日,·,weekday]·六月廿一 }
生活是虚无的螺旋,无限向往着它不能拥有的事物。\footnote{\bi{不安之书} \regular{费尔南多·佩索阿 热罗尼莫·皮萨罗 }}

\title{\date[d=27,m=7,y=2024][year:cn-y,年,month:cn,day:cn,日,·,weekday]·六月廿二 }
千百个太阳从缝隙飞入。被倒置的重量引力主宰着光的游戏\footnote{\bi{沉石与火舌:特朗斯特罗姆诗全集} \regular{托马斯·特朗斯特罗姆}}

\title{\date[d=28,m=7,y=2024][year:cn-y,年,month:cn,day:cn,日,·,weekday]·六月廿三 }
精神上的搏斗和人与人之间的战斗一样激烈残酷;\footnote{\bi{彩画集:兰波散文诗全集(译文经典)} \regular{阿蒂尔·兰波}}

\title{\date[d=29,m=7,y=2024][year:cn-y,年,month:cn,day:cn,日,·,weekday]·六月廿四 }
我把全部的风都揽进了自己的怀抱,——不对!是抱住了整个北方。\footnote{\bi{抒情诗的呼吸:一九二六年书信(帕斯捷尔纳克作品系列)} \regular{鲍·列·帕斯捷尔纳克 玛·伊·茨维塔耶娃 莱·马·里尔克}}

\title{\date[d=30,m=7,y=2024][year:cn-y,年,month:cn,day:cn,日,·,weekday]·六月廿五 }
我用词语幻觉解释我各种像中了魔法那样的诡论!\footnote{\bi{彩画集:兰波散文诗全集(译文经典)} \regular{阿蒂尔·兰波}}

\title{\date[d=31,m=7,y=2024][year:cn-y,年,month:cn,day:cn,日,·,weekday]·六月廿六 }
我正死着自己的死和后人的死。\footnote{\bi{荒原:艾略特文集·诗歌} \regular{T.S.艾略特}}

\title{\date[d=1,m=8,y=2024][year:cn-y,年,month:cn,day:cn,日,·,weekday]·六月廿七 }
我的命运贴地而行。\footnote{\bi{堂吉诃德(译文名著精选)} \regular{塞万提斯}}

\title{\date[d=2,m=8,y=2024][year:cn-y,年,month:cn,day:cn,日,·,weekday]·六月廿八 }
组成世界的无数人与物就是一条看不见尽头的画廊,其内在枯燥乏味\footnote{\bi{不安之书} \regular{费尔南多·佩索阿 热罗尼莫·皮萨罗 }}

\title{\date[d=3,m=8,y=2024][year:cn-y,年,month:cn,day:cn,日,·,weekday]·六月廿九 }
为什么万年又生一天?\footnote{\bi{骆一禾的诗} \regular{骆一禾 西渡}}

\title{\date[d=4,m=8,y=2024][year:cn-y,年,month:cn,day:cn,日,·,weekday]·七月初一 }
灵魂的渴求只有溺水者的感受可为比拟\footnote{\bi{昌耀的诗} \regular{昌耀}}

\title{\date[d=5,m=8,y=2024][year:cn-y,年,month:cn,day:cn,日,·,weekday]·七月初二 }
只要沉默与深渊还在拖延,我想独处!\footnote{\bi{宇宙重建了自身:佩索阿诗精选} \regular{费尔南多·佩索阿}}

\title{\date[d=6,m=8,y=2024][year:cn-y,年,month:cn,day:cn,日,·,weekday]·七月初三 }
我是我所成为的和我所不成为的事物之间的空隔,是我梦到的事物和生活使我成为的事物之间的空隔,是抽象和有肉体的、一无是处的东西之间的平均数,而我也是虚无\footnote{\bi{不安之书} \regular{费尔南多·佩索阿 热罗尼莫·皮萨罗 }}

\title{\date[d=7,m=8,y=2024][year:cn-y,年,month:cn,day:cn,日,·,weekday]·七月初四 ·立秋}
众鸟的合唱降为低语,\footnote{\bi{黄灿然的诗} \regular{黄灿然}}

\title{\date[d=8,m=8,y=2024][year:cn-y,年,month:cn,day:cn,日,·,weekday]·七月初五 }
我把手插入我的海顿口袋模仿一个人从容地观望世界。\footnote{\bi{沉石与火舌:特朗斯特罗姆诗全集} \regular{托马斯·特朗斯特罗姆}}

\title{\date[d=9,m=8,y=2024][year:cn-y,年,month:cn,day:cn,日,·,weekday]·七月初六 }
也可能我全部的过去存在于别处\footnote{\bi{想象一朵未来的玫瑰:佩索阿诗选} \regular{费尔南多·佩索阿}}

\title{\date[d=10,m=8,y=2024][year:cn-y,年,month:cn,day:cn,日,·,weekday]·七月初七 ·七夕}
在夜的黑暗中碾着虚无\footnote{\bi{沉石与火舌:特朗斯特罗姆诗全集} \regular{托马斯·特朗斯特罗姆}}

\title{\date[d=11,m=8,y=2024][year:cn-y,年,month:cn,day:cn,日,·,weekday]·七月初八 }
这赋予我的脸一种甚至比我的童年还老的老态,让我的凝视在幸福中流露出一丝焦虑。\footnote{\bi{宇宙重建了自身:佩索阿诗精选} \regular{费尔南多·佩索阿}}

\title{\date[d=12,m=8,y=2024][year:cn-y,年,month:cn,day:cn,日,·,weekday]·七月初九 }
她与世界同眠,\footnote{\bi{最好的里尔克} \regular{赖纳·马利亚·里尔克}}

\title{\date[d=13,m=8,y=2024][year:cn-y,年,month:cn,day:cn,日,·,weekday]·七月初十 }
我等待真理退去,好将自己再次设定为无用和虚构,聪明和自然。\footnote{\bi{不安之书} \regular{费尔南多·佩索阿 热罗尼莫·皮萨罗 }}

\title{\date[d=14,m=8,y=2024][year:cn-y,年,month:cn,day:cn,日,·,weekday]·七月十一 }
太阳升起来天空血淋淋的犹如一块盾\footnote{\bi{芒克的诗} \regular{芒克}}

\title{\date[d=15,m=8,y=2024][year:cn-y,年,month:cn,day:cn,日,·,weekday]·七月十二 }
啊,如果我能成为所有人和所有地方就好了!\footnote{\bi{宇宙重建了自身:佩索阿诗精选} \regular{费尔南多·佩索阿}}

\title{\date[d=16,m=8,y=2024][year:cn-y,年,month:cn,day:cn,日,·,weekday]·七月十三 }
你去思考你自身负担着的世界;\footnote{\bi{给青年诗人的信} \regular{莱内·马利亚·里尔克}}

\title{\date[d=17,m=8,y=2024][year:cn-y,年,month:cn,day:cn,日,·,weekday]·七月十四 }
随后宇宙在我眼里已重建了自身,却仍不理想也不符合希望,\footnote{\bi{宇宙重建了自身:佩索阿诗精选} \regular{费尔南多·佩索阿}}

\title{\date[d=18,m=8,y=2024][year:cn-y,年,month:cn,day:cn,日,·,weekday]·七月十五 ·中元节}
太阳出来了吗?我们这里,——太阳一刻也不曾有过。我想把整个太阳都寄给你,把它钉在你头顶上的那爿天空上。\footnote{\bi{抒情诗的呼吸:一九二六年书信(帕斯捷尔纳克作品系列)} \regular{鲍·列·帕斯捷尔纳克 玛·伊·茨维塔耶娃 莱·马·里尔克}}

\title{\date[d=19,m=8,y=2024][year:cn-y,年,month:cn,day:cn,日,·,weekday]·七月十六 }
风一直在领航,指引的是海上的波浪波浪一直在荡,海面上延伸的钟磬一直在响\footnote{\bi{戈麦的诗} \regular{戈麦 西渡}}

\title{\date[d=20,m=8,y=2024][year:cn-y,年,month:cn,day:cn,日,·,weekday]·七月十七 }
别像提笔写诗一样写信,不要只把生活当成情感喜怒无常和不可信赖之诱因加以忍受。生活比这个多得多\footnote{\bi{谁此时孤独:里尔克晚期书信选} \regular{里尔克}}

\title{\date[d=21,m=8,y=2024][year:cn-y,年,month:cn,day:cn,日,·,weekday]·七月十八 }
十只海鸥就可以造就一个抒情诗人一万只海鸥之下 必有一个诗人之城\footnote{\bi{于坚的诗} \regular{于坚}}

\title{\date[d=22,m=8,y=2024][year:cn-y,年,month:cn,day:cn,日,·,weekday]·七月十九 ·处暑}
我学着看[插图]。不知为何,一切都更深地进入了我,并不在它们素来终止的地方停留。\footnote{\bi{布里格手记} \regular{里尔克}}

\title{\date[d=23,m=8,y=2024][year:cn-y,年,month:cn,day:cn,日,·,weekday]·七月二十 }
反时间也在转动。\footnote{\bi{欧阳江河的诗} \regular{欧阳江河}}

\title{\date[d=24,m=8,y=2024][year:cn-y,年,month:cn,day:cn,日,·,weekday]·七月廿一 }
我一遍又一遍挥霍你的形象,只企盼有一天把你用完耗毁——可那与我相似的,皆与你相反。\footnote{\bi{张枣的诗} \regular{张枣 颜炼军}}

\title{\date[d=25,m=8,y=2024][year:cn-y,年,month:cn,day:cn,日,·,weekday]·七月廿二 }
这质朴、广大、急速旋转的原野,只是为了让你看它最后一眼才出现的,只是一闪,看了你一眼,永远消逝了……\footnote{\bi{王家新的诗} \regular{王家新}}

\title{\date[d=26,m=8,y=2024][year:cn-y,年,month:cn,day:cn,日,·,weekday]·七月廿三 }
你心中的时间越多, 你就变得越老\footnote{\bi{毛毛:时间窃贼和一个小女孩的不可思议的故事} \regular{米切尔·恩德}}

\title{\date[d=27,m=8,y=2024][year:cn-y,年,month:cn,day:cn,日,·,weekday]·七月廿四 }
玻璃窗上写字\footnote{\bi{多多的诗} \regular{多多}}

\title{\date[d=28,m=8,y=2024][year:cn-y,年,month:cn,day:cn,日,·,weekday]·七月廿五 }
诗人就是那种超越(本应当超越)生命的人。\footnote{\bi{抒情诗的呼吸:一九二六年书信(帕斯捷尔纳克作品系列)} \regular{鲍·列·帕斯捷尔纳克 玛·伊·茨维塔耶娃 莱·马·里尔克}}

\title{\date[d=29,m=8,y=2024][year:cn-y,年,month:cn,day:cn,日,·,weekday]·七月廿六 }
形象的出现往往先于意念\footnote{\bi{戈麦的诗} \regular{戈麦 西渡}}

\title{\date[d=30,m=8,y=2024][year:cn-y,年,month:cn,day:cn,日,·,weekday]·七月廿七 }
意志想要实现的事情,身体诚实地展现了出来\footnote{\bi{金阁寺} \regular{三岛由纪夫}}

\title{\date[d=31,m=8,y=2024][year:cn-y,年,month:cn,day:cn,日,·,weekday]·七月廿八 }
她像炊烟忠实于天空\footnote{\bi{于坚的诗} \regular{于坚}}

\title{\date[d=1,m=9,y=2024][year:cn-y,年,month:cn,day:cn,日,·,weekday]·七月廿九 }
我们的声音是夜晚、月光和森林的一部分。\footnote{\bi{不安之书} \regular{费尔南多·佩索阿 热罗尼莫·皮萨罗 }}

\title{\date[d=2,m=9,y=2024][year:cn-y,年,month:cn,day:cn,日,·,weekday]·七月三十 }
偶然形成的景色所具有的神秘魅力。但这种渴望又总是使我们感到悲哀!\footnote{\bi{恶之花} \regular{波德莱尔}}

\title{\date[d=3,m=9,y=2024][year:cn-y,年,month:cn,day:cn,日,·,weekday]·八月初一 }
用一切方式感觉一切事物\footnote{\bi{自决之书} \regular{费尔南多·佩索阿}}

\title{\date[d=4,m=9,y=2024][year:cn-y,年,month:cn,day:cn,日,·,weekday]·八月初二 }
在俄语诗人中,茨维塔耶娃是最受作曲家青睐的诗人之一,肖斯塔科维奇等著名音乐家曾将她的许多诗作谱成歌曲,这并非偶然。\footnote{\bi{茨维塔耶娃诗选} \regular{茨维塔耶娃}}

\title{\date[d=5,m=9,y=2024][year:cn-y,年,month:cn,day:cn,日,·,weekday]·八月初三 }
随后,会有这方土地承受哭泣。是无名氏的哭泣。是情有所自的语言的哭泣。\footnote{\bi{昌耀的诗} \regular{昌耀}}

\title{\date[d=6,m=9,y=2024][year:cn-y,年,month:cn,day:cn,日,·,weekday]·八月初四 }
要想走遍世界就别长大走遍世界别长大\footnote{\bi{骆一禾的诗} \regular{骆一禾 西渡}}

\title{\date[d=7,m=9,y=2024][year:cn-y,年,month:cn,day:cn,日,·,weekday]·八月初五 ·白露}
从一种陌生到另一种陌生\footnote{\bi{戈麦的诗} \regular{戈麦 西渡}}

\title{\date[d=8,m=9,y=2024][year:cn-y,年,month:cn,day:cn,日,·,weekday]·八月初六 }
因为所有这一切——天,地,世界,——所有这一切中的能有的只是我自己!\footnote{\bi{不安之书} \regular{费尔南多·佩索阿 热罗尼莫·皮萨罗 }}

\title{\date[d=9,m=9,y=2024][year:cn-y,年,month:cn,day:cn,日,·,weekday]·八月初七 }
与人的每一次相处都是一个岛,并且是一个永远沉没的岛——完全沉没,无踪无影。\footnote{\bi{抒情诗的呼吸:一九二六年书信(帕斯捷尔纳克作品系列)} \regular{鲍·列·帕斯捷尔纳克 玛·伊·茨维塔耶娃 莱·马·里尔克}}

\title{\date[d=10,m=9,y=2024][year:cn-y,年,month:cn,day:cn,日,·,weekday]·八月初八 }
死也是一种享乐死也是一次选择\footnote{\bi{芒克的诗} \regular{芒克}}

\title{\date[d=11,m=9,y=2024][year:cn-y,年,month:cn,day:cn,日,·,weekday]·八月初九 }
我就是这最后一个夜晚最后一盏黑暗的灯是最后一个夜晚水面上爱情阴沉的旗帜在黑暗中鞭打着一颗干渴的心沿着先知的梯子上下爬行\footnote{\bi{戈麦的诗} \regular{戈麦 西渡}}

\title{\date[d=12,m=9,y=2024][year:cn-y,年,month:cn,day:cn,日,·,weekday]·八月初十 }
用静止计算时间\footnote{\bi{戈麦的诗} \regular{戈麦 西渡}}

\title{\date[d=13,m=9,y=2024][year:cn-y,年,month:cn,day:cn,日,·,weekday]·八月十一 }
琴声如诉,耳朵里空有一颗心\footnote{\bi{欧阳江河的诗} \regular{欧阳江河}}

\title{\date[d=14,m=9,y=2024][year:cn-y,年,month:cn,day:cn,日,·,weekday]·八月十二 }
我思念的片断是一只在雨后的田野里爬满露水的南瓜这思念在夏日的流水中与女人的体温交谈\footnote{\bi{于坚的诗} \regular{于坚}}

\title{\date[d=15,m=9,y=2024][year:cn-y,年,month:cn,day:cn,日,·,weekday]·八月十三 }
金币般的叶子!\footnote{\bi{王家新的诗} \regular{王家新}}

\title{\date[d=16,m=9,y=2024][year:cn-y,年,month:cn,day:cn,日,·,weekday]·八月十四 }
但是现在一切都完了\footnote{\bi{彩画集:兰波散文诗全集(译文经典)} \regular{阿蒂尔·兰波}}

\title{\date[d=17,m=9,y=2024][year:cn-y,年,month:cn,day:cn,日,·,weekday]·八月十五 ·中秋节}
一切重来,真正把它承担起来,这就是远行者回家的原因\footnote{\bi{布里格手记} \regular{里尔克}}

\title{\date[d=18,m=9,y=2024][year:cn-y,年,month:cn,day:cn,日,·,weekday]·八月十六 }
橘红飘动的睡袍\footnote{\bi{顾城的诗} \regular{顾城}}

\title{\date[d=19,m=9,y=2024][year:cn-y,年,month:cn,day:cn,日,·,weekday]·八月十七 }
我让一首旧诗写我。我已让它写我了很久很久。\footnote{\bi{王家新的诗} \regular{王家新}}

\title{\date[d=20,m=9,y=2024][year:cn-y,年,month:cn,day:cn,日,·,weekday]·八月十八 }
多汁而又黑暗的土地\footnote{\bi{骆一禾的诗} \regular{骆一禾 西渡}}

\title{\date[d=21,m=9,y=2024][year:cn-y,年,month:cn,day:cn,日,·,weekday]·八月十九 }
生命中最黑暗的事件 “写”永远不会抵达 所谓写作就是逃跑的马拉松\footnote{\bi{于坚的诗} \regular{于坚}}

\title{\date[d=22,m=9,y=2024][year:cn-y,年,month:cn,day:cn,日,·,weekday]·八月二十 ·秋分}
喝吧,我的小燕子!杯底是溶解的珍珠……”你在畅饮大海,你在畅饮霞光。\footnote{\bi{茨维塔耶娃诗选} \regular{茨维塔耶娃}}

\title{\date[d=23,m=9,y=2024][year:cn-y,年,month:cn,day:cn,日,·,weekday]·八月廿一 }
体验新感觉的唯一方法就是建立一个新的灵魂\footnote{\bi{不安之书} \regular{费尔南多·佩索阿 热罗尼莫·皮萨罗 }}

\title{\date[d=24,m=9,y=2024][year:cn-y,年,month:cn,day:cn,日,·,weekday]·八月廿二 }
我不能把我模仿水的涟漪的韵律换成水在我手上的真正凉意,\footnote{\bi{宇宙重建了自身:佩索阿诗精选} \regular{费尔南多·佩索阿}}

\title{\date[d=25,m=9,y=2024][year:cn-y,年,month:cn,day:cn,日,·,weekday]·八月廿三 }
如果你在人我之间没有和谐,你就试行与物接近,它们不会遗弃你;还有夜,还有风——那吹过树林、掠过田野的风;在物中间和动物那里,一切都充满了你可以分担的事;还有儿童,他们同你在儿时所经验过的一样,又悲哀,又幸福,——如果你想起你的童年,你就又在那些寂寞的儿童中间\footnote{\bi{给青年诗人的信} \regular{莱内·马利亚·里尔克}}

\title{\date[d=26,m=9,y=2024][year:cn-y,年,month:cn,day:cn,日,·,weekday]·八月廿四 }
所有的单调也就是我本身的单调\footnote{\bi{不安之书} \regular{费尔南多·佩索阿 热罗尼莫·皮萨罗 }}

\title{\date[d=27,m=9,y=2024][year:cn-y,年,month:cn,day:cn,日,·,weekday]·八月廿五 }
太阳在你的金发上提炼黄金。\footnote{\bi{我的心迟到了:佩索阿情诗} \regular{费尔南多·佩索阿}}

\title{\date[d=28,m=9,y=2024][year:cn-y,年,month:cn,day:cn,日,·,weekday]·八月廿六 }
她笑的时候我感到卷入了她的笑声并成了笑声的一部分,最后的她的牙齿成了仅仅偶然出现的星星,仿佛富有班组训练才能一般地偶然出现的星星。我被一次次短暂的喘气吸进,在每一个短暂的恢复中吸下,终于消失在她咽喉的漆黑的洞穴中,在那看不到的肌肤的波纹中擦得遍体鳞伤\footnote{\bi{荒原:艾略特文集·诗歌} \regular{T.S.艾略特}}

\title{\date[d=29,m=9,y=2024][year:cn-y,年,month:cn,day:cn,日,·,weekday]·八月廿七 }
我像日暮的结尾一样漫游在风景发生之间。眼皮重重地压到我拖着的脚上。\footnote{\bi{不安之书} \regular{费尔南多·佩索阿 热罗尼莫·皮萨罗 }}

\title{\date[d=30,m=9,y=2024][year:cn-y,年,month:cn,day:cn,日,·,weekday]·八月廿八 }
世界就是另外一种看法 另外一类词汇\footnote{\bi{于坚的诗} \regular{于坚}}

\title{\date[d=1,m=10,y=2024][year:cn-y,年,month:cn,day:cn,日,·,weekday]·八月廿九 }
打伤我,撕开我,杀了我!我想要的一切是将一颗溢出大海的灵魂带给死神,\footnote{\bi{宇宙重建了自身:佩索阿诗精选} \regular{费尔南多·佩索阿}}

\title{\date[d=2,m=10,y=2024][year:cn-y,年,month:cn,day:cn,日,·,weekday]·八月三十 }
高山的脸\footnote{\bi{于坚的诗} \regular{于坚}}

\title{\date[d=3,m=10,y=2024][year:cn-y,年,month:cn,day:cn,日,·,weekday]·九月初一 }
说出那些最剧烈的苦痛,也就说出我的痛苦。\footnote{\bi{秋日:冯至译诗选} \regular{歌德 海涅 尼采 荷尔德林 布莱希特 里尔克 格奥尔格}}

\title{\date[d=4,m=10,y=2024][year:cn-y,年,month:cn,day:cn,日,·,weekday]·九月初二 }
我的手燃烧着去舔她的手但她一疼就缩开了\footnote{\bi{于坚的诗} \regular{于坚}}

\title{\date[d=5,m=10,y=2024][year:cn-y,年,month:cn,day:cn,日,·,weekday]·九月初三 }
我只不过是一种有颜色、有形状、有影像的客观结构,我是一面将被售出的、摇摆不定的镜子。\footnote{\bi{不安之书} \regular{费尔南多·佩索阿 热罗尼莫·皮萨罗 }}

\title{\date[d=6,m=10,y=2024][year:cn-y,年,month:cn,day:cn,日,·,weekday]·九月初四 }
爱情是一种受难!\footnote{\bi{秋日:冯至译诗选} \regular{歌德 海涅 尼采 荷尔德林 布莱希特 里尔克 格奥尔格}}

\title{\date[d=7,m=10,y=2024][year:cn-y,年,month:cn,day:cn,日,·,weekday]·九月初五 }
我们就是历史\footnote{\bi{西川的诗} \regular{西川}}

\title{\date[d=8,m=10,y=2024][year:cn-y,年,month:cn,day:cn,日,·,weekday]·九月初六 ·寒露}
努力是一种荒谬的浪费,生命是一场空,因为幻灭总是紧随在幻想之后而死亡似乎是生命的意义……\footnote{\bi{宇宙重建了自身:佩索阿诗精选} \regular{费尔南多·佩索阿}}

\title{\date[d=9,m=10,y=2024][year:cn-y,年,month:cn,day:cn,日,·,weekday]·九月初七 }
成一种它是不可思议的感觉,在我写下这些文字的时候,这种感觉实在难以形诸笔墨。\footnote{\bi{抒情诗的呼吸:一九二六年书信(帕斯捷尔纳克作品系列)} \regular{鲍·列·帕斯捷尔纳克 玛·伊·茨维塔耶娃 莱·马·里尔克}}

\title{\date[d=10,m=10,y=2024][year:cn-y,年,month:cn,day:cn,日,·,weekday]·九月初八 }
诗是我让它醒着的梦\footnote{\bi{沉石与火舌:特朗斯特罗姆诗全集} \regular{托马斯·特朗斯特罗姆}}

\title{\date[d=11,m=10,y=2024][year:cn-y,年,month:cn,day:cn,日,·,weekday]·九月初九 ·重阳节}
恐怖在情人极乐的乳房上追到他们,让他们瑟瑟发抖,灰心丧气\footnote{\bi{布里格手记} \regular{里尔克}}

\title{\date[d=12,m=10,y=2024][year:cn-y,年,month:cn,day:cn,日,·,weekday]·九月初十 }
一大团酷似我的黑暗\footnote{\bi{骆一禾的诗} \regular{骆一禾 西渡}}

\title{\date[d=13,m=10,y=2024][year:cn-y,年,month:cn,day:cn,日,·,weekday]·九月十一 }
一个这样的图像不必拘泥于言辞,它活着是靠自身飘忽不定,它由此更新自己,并非它仿佛不确切并意欲始终如此。\footnote{\bi{谁此时孤独:里尔克晚期书信选} \regular{里尔克}}

\title{\date[d=14,m=10,y=2024][year:cn-y,年,month:cn,day:cn,日,·,weekday]·九月十二 }
言在此而意在彼,泯端倪而离形象,绝议论而穷思维,引人于冥漠恍惚之境,所以为至也。\footnote{\bi{沉石与火舌:特朗斯特罗姆诗全集} \regular{托马斯·特朗斯特罗姆}}

\title{\date[d=15,m=10,y=2024][year:cn-y,年,month:cn,day:cn,日,·,weekday]·九月十三 }
手插在遗物一般的大衣口袋里,我以宽大而坚决的脚步将我短短的房间走成大道,以无用的狂想做完与众人并无二致的梦。\footnote{\bi{不安之书} \regular{费尔南多·佩索阿 热罗尼莫·皮萨罗 }}

\title{\date[d=16,m=10,y=2024][year:cn-y,年,month:cn,day:cn,日,·,weekday]·九月十四 }
这种语言,综合了芳香、音响、色彩,概括一切,可以把思想与思想连结起来,又引出思想,这种语言将使心灵与心灵呼应相通。\footnote{\bi{彩画集:兰波散文诗全集(译文经典)} \regular{阿蒂尔·兰波}}

\title{\date[d=17,m=10,y=2024][year:cn-y,年,month:cn,day:cn,日,·,weekday]·九月十五 }
那些颜色杂乱的烟被风抽成细丝\footnote{\bi{顾城的诗} \regular{顾城}}

\title{\date[d=18,m=10,y=2024][year:cn-y,年,month:cn,day:cn,日,·,weekday]·九月十六 }
我什么都不是……我是一个虚构……我想从世上的一切或自身之中获得什么?\footnote{\bi{想象一朵未来的玫瑰:佩索阿诗选} \regular{费尔南多·佩索阿}}

\title{\date[d=19,m=10,y=2024][year:cn-y,年,month:cn,day:cn,日,·,weekday]·九月十七 }
河水,我饮过苍凉\footnote{\bi{戈麦的诗} \regular{戈麦 西渡}}

\title{\date[d=20,m=10,y=2024][year:cn-y,年,month:cn,day:cn,日,·,weekday]·九月十八 }
大爱人\footnote{\bi{骆一禾的诗} \regular{骆一禾 西渡}}

\title{\date[d=21,m=10,y=2024][year:cn-y,年,month:cn,day:cn,日,·,weekday]·九月十九 }
无垠向我们敞开…\footnote{\bi{沉石与火舌:特朗斯特罗姆诗全集} \regular{托马斯·特朗斯特罗姆}}

\title{\date[d=22,m=10,y=2024][year:cn-y,年,month:cn,day:cn,日,·,weekday]·九月二十 }
每个夜晚所有房间都黑暗,每个声音都暗淡。\footnote{\bi{茨维塔耶娃诗选} \regular{茨维塔耶娃}}

\title{\date[d=23,m=10,y=2024][year:cn-y,年,month:cn,day:cn,日,·,weekday]·九月廿一 ·霜降}
不用色彩,不用画笔!光是它的王国,因为它白。\footnote{\bi{茨维塔耶娃诗选} \regular{茨维塔耶娃}}

\title{\date[d=24,m=10,y=2024][year:cn-y,年,month:cn,day:cn,日,·,weekday]·九月廿二 }
夜,你这黑太阳,请把我烧成灰!\footnote{\bi{茨维塔耶娃诗选} \regular{茨维塔耶娃}}

\title{\date[d=25,m=10,y=2024][year:cn-y,年,month:cn,day:cn,日,·,weekday]·九月廿三 }
虽然真正伸出的双手从来不是空的\footnote{\bi{谁此时孤独:里尔克晚期书信选} \regular{里尔克}}

\title{\date[d=26,m=10,y=2024][year:cn-y,年,month:cn,day:cn,日,·,weekday]·九月廿四 }
诺瓦利斯的告白“生命就是精神的一种疾病”,或兰波的绝望之语“真实的生命已然缺席,我们并不存于此世”\footnote{\bi{西方正典} \regular{哈罗德·布鲁姆}}

\title{\date[d=27,m=10,y=2024][year:cn-y,年,month:cn,day:cn,日,·,weekday]·九月廿五 }
只有贫瘠不育才是高贵和值得尊敬的。唯有杀死从未存在的东西,才是罕见、崇高和荒谬的。\footnote{\bi{不安之书} \regular{费尔南多·佩索阿 热罗尼莫·皮萨罗 }}

\title{\date[d=28,m=10,y=2024][year:cn-y,年,month:cn,day:cn,日,·,weekday]·九月廿六 }
每一个人都是很多人。对我来说,我是我想成为的那个人,而对他人来说,却是巨大的错谬—\footnote{\bi{我的心迟到了:佩索阿情诗} \regular{费尔南多·佩索阿}}

\title{\date[d=29,m=10,y=2024][year:cn-y,年,month:cn,day:cn,日,·,weekday]·九月廿七 }
我那巨大的怀旧来自空无,是空无,就像我看不见却在不具人格中凝视着的高天。\footnote{\bi{不安之书} \regular{费尔南多·佩索阿 热罗尼莫·皮萨罗 }}

\title{\date[d=30,m=10,y=2024][year:cn-y,年,month:cn,day:cn,日,·,weekday]·九月廿八 }
起风了,新鲜,温和,一切都升起来:气味,呼唤,钟声。\footnote{\bi{布里格手记} \regular{里尔克}}

\title{\date[d=31,m=10,y=2024][year:cn-y,年,month:cn,day:cn,日,·,weekday]·九月廿九 }
我听说睡眠只有一样不好:它跟死亡差不多,一个沉睡的人和一个死人没什么两样\footnote{\bi{堂吉诃德(译文名著精选)} \regular{塞万提斯}}

\title{\date[d=1,m=11,y=2024][year:cn-y,年,month:cn,day:cn,日,·,weekday]·十月初一 }
也许是由于遥远,已把你变为一种文体\footnote{\bi{多多的诗} \regular{多多}}

\title{\date[d=2,m=11,y=2024][year:cn-y,年,month:cn,day:cn,日,·,weekday]·十月初二 }
在拥抱中一切尽失,比起这浓稠的悲伤,所有这些阴郁又算得了什么\footnote{\bi{布里格手记} \regular{里尔克}}

\title{\date[d=3,m=11,y=2024][year:cn-y,年,month:cn,day:cn,日,·,weekday]·十月初三 }
毫无起伏,毫无意义\footnote{\bi{荒原:艾略特文集·诗歌} \regular{T.S.艾略特}}

\title{\date[d=4,m=11,y=2024][year:cn-y,年,month:cn,day:cn,日,·,weekday]·十月初四 }
嘿,所有过去都在现在里!嘿,所有未来已经在我们身体里了!嘿!\footnote{\bi{宇宙重建了自身:佩索阿诗精选} \regular{费尔南多·佩索阿}}

\title{\date[d=5,m=11,y=2024][year:cn-y,年,month:cn,day:cn,日,·,weekday]·十月初五 }
我易于感伤。而对于泪水,人们总是讳莫如深\footnote{\bi{昌耀的诗} \regular{昌耀}}

\title{\date[d=6,m=11,y=2024][year:cn-y,年,month:cn,day:cn,日,·,weekday]·十月初六 }
我借此火得度一生的茫茫黑夜\footnote{\bi{海子的诗} \regular{海子}}

\title{\date[d=7,m=11,y=2024][year:cn-y,年,month:cn,day:cn,日,·,weekday]·十月初七 ·立冬}
这一切以及许多更细微的事情,似乎偶然的事情,都能引起并强化一种自我的发现或重新认识\footnote{\bi{谁此时孤独:里尔克晚期书信选} \regular{里尔克}}

\title{\date[d=8,m=11,y=2024][year:cn-y,年,month:cn,day:cn,日,·,weekday]·十月初八 }
两个人的孤独只是孤独的一半。\footnote{\bi{欧阳江河的诗} \regular{欧阳江河}}

\title{\date[d=9,m=11,y=2024][year:cn-y,年,month:cn,day:cn,日,·,weekday]·十月初九 }
我知道我曾是错误和歪路,知道自己从未生活过,而仅仅只是存在过而已,因为我只是用意识和思考填充了时间。\footnote{\bi{不安之书} \regular{费尔南多·佩索阿 热罗尼莫·皮萨罗 }}

\title{\date[d=10,m=11,y=2024][year:cn-y,年,month:cn,day:cn,日,·,weekday]·十月初十 }
一切都那么多,一切都那么深,一切都那么黑而又那么冷!\footnote{\bi{不安之书} \regular{费尔南多·佩索阿 热罗尼莫·皮萨罗 }}

\title{\date[d=11,m=11,y=2024][year:cn-y,年,month:cn,day:cn,日,·,weekday]·十月十一 }
…我不再像以往那样热衷于旅行。但旅行却登门拜访\footnote{\bi{沉石与火舌:特朗斯特罗姆诗全集} \regular{托马斯·特朗斯特罗姆}}

\title{\date[d=12,m=11,y=2024][year:cn-y,年,month:cn,day:cn,日,·,weekday]·十月十二 }
不仅仅成为你们在我自己身体里的抽象的狂欢行为,不仅如此,我还想成为比这更多的人——这一切的上帝!\footnote{\bi{宇宙重建了自身:佩索阿诗精选} \regular{费尔南多·佩索阿}}

\title{\date[d=13,m=11,y=2024][year:cn-y,年,month:cn,day:cn,日,·,weekday]·十月十三 }
我这显然微乎其微的声音是否也体现了千万个声音的本质、千万个生命倾诉自我的饥渴以及无数灵魂的忍耐,它们与我的灵魂一样屈从于日常命运、徒劳之梦和无影无踪的希望。\footnote{\bi{不安之书} \regular{费尔南多·佩索阿 热罗尼莫·皮萨罗 }}

\title{\date[d=14,m=11,y=2024][year:cn-y,年,month:cn,day:cn,日,·,weekday]·十月十四 }
你只是一张车票到时候检票的女郎会把你轻轻地一撕 一点也不疼\footnote{\bi{于坚的诗} \regular{于坚}}

\title{\date[d=15,m=11,y=2024][year:cn-y,年,month:cn,day:cn,日,·,weekday]·十月十五 }
那种精密也许无限接近恶的运转精密\footnote{\bi{金阁寺} \regular{三岛由纪夫}}

\title{\date[d=16,m=11,y=2024][year:cn-y,年,month:cn,day:cn,日,·,weekday]·十月十六 }
雾,缓缓化开像糯米纸一样\footnote{\bi{顾城的诗} \regular{顾城}}

\title{\date[d=17,m=11,y=2024][year:cn-y,年,month:cn,day:cn,日,·,weekday]·十月十七 }
黑暗是怎样地在你身上掠夺\footnote{\bi{芒克的诗} \regular{芒克}}

\title{\date[d=18,m=11,y=2024][year:cn-y,年,month:cn,day:cn,日,·,weekday]·十月十八 }
你对着透明处呵气;你把自己变小;你像孩子那样藏起来,发出短促、快乐的呼声,最多只有天使才能找到你。\footnote{\bi{布里格手记} \regular{里尔克}}

\title{\date[d=19,m=11,y=2024][year:cn-y,年,month:cn,day:cn,日,·,weekday]·十月十九 }
诗不是表达“瞬息情绪”就完了。更真实的世界是在瞬息消失后的那种持续性和整体性,对立物的结合\footnote{\bi{沉石与火舌:特朗斯特罗姆诗全集} \regular{托马斯·特朗斯特罗姆}}

\title{\date[d=20,m=11,y=2024][year:cn-y,年,month:cn,day:cn,日,·,weekday]·十月二十 }
莫让死亡的欢快,再次给我以生命。\footnote{\bi{堂吉诃德(译文名著精选)} \regular{塞万提斯}}

\title{\date[d=21,m=11,y=2024][year:cn-y,年,month:cn,day:cn,日,·,weekday]·十月廿一 }
不自主的人性动物或自然人的表达,抛却了一切学识、记忆和过去对感官的表现。这种诗存在过吗?\footnote{\bi{西方正典} \regular{哈罗德·布鲁姆}}

\title{\date[d=22,m=11,y=2024][year:cn-y,年,month:cn,day:cn,日,·,weekday]·十月廿二 ·小雪}
我有比我好像活了一千岁还要多的记忆。\footnote{\bi{恶之花} \regular{波德莱尔}}

\title{\date[d=23,m=11,y=2024][year:cn-y,年,month:cn,day:cn,日,·,weekday]·十月廿三 }
我们真想潜入深渊深处,潜入未知世界的深处\footnote{\bi{恶之花} \regular{波德莱尔}}

\title{\date[d=24,m=11,y=2024][year:cn-y,年,month:cn,day:cn,日,·,weekday]·十月廿四 }
我又想喝酒,喝醉我现在端起的是盛满感情的酒杯\footnote{\bi{芒克的诗} \regular{芒克}}

\title{\date[d=25,m=11,y=2024][year:cn-y,年,month:cn,day:cn,日,·,weekday]·十月廿五 }
我是一口井,有许多举动不曾在我心里勾勒,有许多话我从没想过抵上双唇的弓弦,有许多梦我忘记梦见它们的结局。\footnote{\bi{不安之书} \regular{费尔南多·佩索阿 热罗尼莫·皮萨罗 }}

\title{\date[d=26,m=11,y=2024][year:cn-y,年,month:cn,day:cn,日,·,weekday]·十月廿六 }
也生下空气 水 癌症孤独感和快乐的眼泪\footnote{\bi{于坚的诗} \regular{于坚}}

\title{\date[d=27,m=11,y=2024][year:cn-y,年,month:cn,day:cn,日,·,weekday]·十月廿七 }
你在夜间用梳子低语,你在夜间打磨箭头。\footnote{\bi{茨维塔耶娃诗选} \regular{茨维塔耶娃}}

\title{\date[d=28,m=11,y=2024][year:cn-y,年,month:cn,day:cn,日,·,weekday]·十月廿八 }
在梦里过虚幻的人生也依然是在生活\footnote{\bi{不安之书} \regular{费尔南多·佩索阿 热罗尼莫·皮萨罗 }}

\title{\date[d=29,m=11,y=2024][year:cn-y,年,month:cn,day:cn,日,·,weekday]·十月廿九 }
当我歌唱起来这街道就是属于我的我把它称作六弦琴\footnote{\bi{骆一禾的诗} \regular{骆一禾 西渡}}

\title{\date[d=30,m=11,y=2024][year:cn-y,年,month:cn,day:cn,日,·,weekday]·十月三十 }
爱情是关联,而非寻求。\footnote{\bi{茨维塔耶娃诗选} \regular{茨维塔耶娃}}

\title{\date[d=1,m=12,y=2024][year:cn-y,年,month:cn,day:cn,日,·,weekday]·十一月初一 }
天上的热有一种死去的冷,\footnote{\bi{不安之书} \regular{费尔南多·佩索阿 热罗尼莫·皮萨罗 }}

\title{\date[d=2,m=12,y=2024][year:cn-y,年,month:cn,day:cn,日,·,weekday]·十一月初二 }
最终,我们应该无法确定自己是真的跟人交谈过,还是单纯想象出一场对话…\footnote{\bi{不安之书} \regular{费尔南多·佩索阿 热罗尼莫·皮萨罗 }}

\title{\date[d=3,m=12,y=2024][year:cn-y,年,month:cn,day:cn,日,·,weekday]·十一月初三 }
宇宙的一切世界正在使自己坠入不可见之物,即坠入自己下一种更深的真实;某些星辰使自己直接上升,消逝在天使无限的意识中;另一些则依附于缓慢而艰难地转化着它们的存在者,在这些存在者的惊惧和狂喜中达到自己下一个不可见的实现\footnote{\bi{谁此时孤独:里尔克晚期书信选} \regular{里尔克}}

\title{\date[d=4,m=12,y=2024][year:cn-y,年,month:cn,day:cn,日,·,weekday]·十一月初四 }
我们向死而生,因为只有死于昨日,才能活于今日。我们期待死亡,因为只有确信今日已死,才能希冀明日。做梦时我们为死而活,因为做梦就是否定生命。活着时我们也在为死而消亡,因为活着就是拒绝永恒!死亡指引我们,寻找我们,陪伴我们。我们拥有的一切都是死亡,我们想要的一切也是死亡,死亡就是我们期待自己渴求的全部。\footnote{\bi{不安之书} \regular{费尔南多·佩索阿 热罗尼莫·皮萨罗 }}

\title{\date[d=5,m=12,y=2024][year:cn-y,年,month:cn,day:cn,日,·,weekday]·十一月初五 }
我公然不模糊眼睛。我透过大雨凝望。\footnote{\bi{茨维塔耶娃诗选} \regular{茨维塔耶娃}}

\title{\date[d=6,m=12,y=2024][year:cn-y,年,month:cn,day:cn,日,·,weekday]·十一月初六 ·大雪}
我将全世界的梦想都集中在我的内心。\footnote{\bi{自决之书} \regular{费尔南多·佩索阿}}

\title{\date[d=7,m=12,y=2024][year:cn-y,年,month:cn,day:cn,日,·,weekday]·十一月初七 }
梦是一种惩罚。我从梦境中获得如此清明的理智,以至于我把所有梦见的事物都视为真实。因此,一切曾经入梦的东西也都失去了价值。\footnote{\bi{不安之书} \regular{费尔南多·佩索阿 热罗尼莫·皮萨罗 }}

\title{\date[d=8,m=12,y=2024][year:cn-y,年,month:cn,day:cn,日,·,weekday]·十一月初八 }
诗句生长,像星星像玫瑰,像家中不需要的美。\footnote{\bi{茨维塔耶娃诗选} \regular{茨维塔耶娃}}

\title{\date[d=9,m=12,y=2024][year:cn-y,年,month:cn,day:cn,日,·,weekday]·十一月初九 }
我像一把梯子斜着,\footnote{\bi{沉石与火舌:特朗斯特罗姆诗全集} \regular{托马斯·特朗斯特罗姆}}

\title{\date[d=10,m=12,y=2024][year:cn-y,年,month:cn,day:cn,日,·,weekday]·十一月初十 }
而是出于可能\footnote{\bi{西川的诗} \regular{西川}}

\title{\date[d=11,m=12,y=2024][year:cn-y,年,month:cn,day:cn,日,·,weekday]·十一月十一 }
愁容骑士\footnote{\bi{堂吉诃德(译文名著精选)} \regular{塞万提斯}}

\title{\date[d=12,m=12,y=2024][year:cn-y,年,month:cn,day:cn,日,·,weekday]·十一月十二 }
时间停下脚步让你路过,我却将你记错了,当我试图把你放进生活抑或相似的表象之中。\footnote{\bi{不安之书} \regular{费尔南多·佩索阿 热罗尼莫·皮萨罗 }}

\title{\date[d=13,m=12,y=2024][year:cn-y,年,month:cn,day:cn,日,·,weekday]·十一月十三 }
你已占有它,它已可口地化身成你。\footnote{\bi{最好的里尔克} \regular{赖纳·马利亚·里尔克}}

\title{\date[d=14,m=12,y=2024][year:cn-y,年,month:cn,day:cn,日,·,weekday]·十一月十四 }
我们彼此相爱,而谎言是我们交换的吻。\footnote{\bi{不安之书} \regular{费尔南多·佩索阿 热罗尼莫·皮萨罗 }}

\title{\date[d=15,m=12,y=2024][year:cn-y,年,month:cn,day:cn,日,·,weekday]·十一月十五 }
是谁设计的这种折磨?爱。\footnote{\bi{荒原:艾略特文集·诗歌} \regular{T.S.艾略特}}

\title{\date[d=16,m=12,y=2024][year:cn-y,年,month:cn,day:cn,日,·,weekday]·十一月十六 }
让世界抽象的一切做法都注定会失败,  就如同给风暴画脸\footnote{\bi{沉石与火舌:特朗斯特罗姆诗全集} \regular{托马斯·特朗斯特罗姆}}

\title{\date[d=17,m=12,y=2024][year:cn-y,年,month:cn,day:cn,日,·,weekday]·十一月十七 }
我在死亡中生活,在冰雪中燃烧,在烈火中发抖\footnote{\bi{堂吉诃德(译文名著精选)} \regular{塞万提斯}}

\title{\date[d=18,m=12,y=2024][year:cn-y,年,month:cn,day:cn,日,·,weekday]·十一月十八 }
不会由此而逃入黄昏与肩头沉重的人兀自相遇\footnote{\bi{戈麦的诗} \regular{戈麦 西渡}}

\title{\date[d=19,m=12,y=2024][year:cn-y,年,month:cn,day:cn,日,·,weekday]·十一月十九 }
我让我身体里所有的泪水连同这一夜倾泻一尽。——我的衰竭由此永远滞留不去。\footnote{\bi{彩画集:兰波散文诗全集(译文经典)} \regular{阿蒂尔·兰波}}

\title{\date[d=20,m=12,y=2024][year:cn-y,年,month:cn,day:cn,日,·,weekday]·十一月二十 }
她在她内里的某处;只是偶尔才回到空空的知觉里来,而且只是一瞬间,她已经不在其中了\footnote{\bi{布里格手记} \regular{里尔克}}

\title{\date[d=21,m=12,y=2024][year:cn-y,年,month:cn,day:cn,日,·,weekday]·十一月廿一 ·冬至}
我可以将自我交给生活了。我可以睡着,可以忽视我自己\footnote{\bi{不安之书} \regular{费尔南多·佩索阿 热罗尼莫·皮萨罗 }}

\title{\date[d=22,m=12,y=2024][year:cn-y,年,month:cn,day:cn,日,·,weekday]·十一月廿二 }
大海是冰冷的,汹涌的,隐秘的,不爱的,充盈自我的,—\footnote{\bi{抒情诗的呼吸:一九二六年书信(帕斯捷尔纳克作品系列)} \regular{鲍·列·帕斯捷尔纳克 玛·伊·茨维塔耶娃 莱·马·里尔克}}

\title{\date[d=23,m=12,y=2024][year:cn-y,年,month:cn,day:cn,日,·,weekday]·十一月廿三 }
大部分信件都有种强烈的令人信服的气味,它扑面而来,似乎也想唤起我心中的回忆。可我没有回忆\footnote{\bi{布里格手记} \regular{里尔克}}

\title{\date[d=24,m=12,y=2024][year:cn-y,年,month:cn,day:cn,日,·,weekday]·十一月廿四 }
唯有无声息的感官为他输入世界,[68]万籁俱寂,一个紧张等待的世界,它尚未完成,在音调被创造之前。\footnote{\bi{布里格手记} \regular{里尔克}}

\title{\date[d=25,m=12,y=2024][year:cn-y,年,month:cn,day:cn,日,·,weekday]·十一月廿五 }
我想,我就是万物,死过了,但还活着。\footnote{\bi{昌耀的诗} \regular{昌耀}}

\title{\date[d=26,m=12,y=2024][year:cn-y,年,month:cn,day:cn,日,·,weekday]·十一月廿六 }
上升的路和下降的路是同一条路。\footnote{\bi{荒原:艾略特文集·诗歌} \regular{T.S.艾略特}}

\title{\date[d=27,m=12,y=2024][year:cn-y,年,month:cn,day:cn,日,·,weekday]·十一月廿七 }
阳光纷然坠下 世界一阵晕眩\footnote{\bi{于坚的诗} \regular{于坚}}

\title{\date[d=28,m=12,y=2024][year:cn-y,年,month:cn,day:cn,日,·,weekday]·十一月廿八 }
它掌控了我;它已事先规定好我的动作,我脸上的表情,甚至是我的想法\footnote{\bi{布里格手记} \regular{里尔克}}

\title{\date[d=29,m=12,y=2024][year:cn-y,年,month:cn,day:cn,日,·,weekday]·十一月廿九 }
古代蒙昧时代的孤独可否体谅?他们在守望,他们在歌唱。\footnote{\bi{彩画集:兰波散文诗全集(译文经典)} \regular{阿蒂尔·兰波}}

\title{\date[d=30,m=12,y=2024][year:cn-y,年,month:cn,day:cn,日,·,weekday]·十一月三十 }
红水晶如雨一样洒落\footnote{\bi{骆一禾的诗} \regular{骆一禾 西渡}}

\title{\date[d=31,m=12,y=2024][year:cn-y,年,month:cn,day:cn,日,·,weekday]·腊月初一 }
有的是一种巨大的消除,它抹去一切做成的动作,而不是那种潜在的疲倦,来自从未做过的动作。\footnote{\bi{不安之书} \regular{费尔南多·佩索阿 热罗尼莫·皮萨罗 }}

