\title{\date[d=1,m=1,y=2024][year:cn-y,年,month:cn,day:cn,日,·,weekday]·十一月二十 }
身体对我们而言就像一丝缥缈的香气,而我们的生活就是泉水叮咚的回响。\footnote{\bi{不安之书} \regular{费尔南多·佩索阿 热罗尼莫·皮萨罗 }}

\title{\date[d=2,m=1,y=2024][year:cn-y,年,month:cn,day:cn,日,·,weekday]·十一月廿一 }
黑色的天空写满词句,漂亮的眼睛失去视力……\footnote{\bi{茨维塔耶娃诗选} \regular{茨维塔耶娃}}

\title{\date[d=3,m=1,y=2024][year:cn-y,年,month:cn,day:cn,日,·,weekday]·十一月廿二 }
我爱过,都是假装爱过,而我的假装,是向着我自己的。\footnote{\bi{不安之书} \regular{费尔南多·佩索阿 热罗尼莫·皮萨罗 }}

\title{\date[d=4,m=1,y=2024][year:cn-y,年,month:cn,day:cn,日,·,weekday]·十一月廿三 }
谁这时孤独,就永远孤独,\footnote{\bi{秋日:冯至译诗选} \regular{歌德 海涅 尼采 荷尔德林 布莱希特 里尔克 格奥尔格}}

\title{\date[d=5,m=1,y=2024][year:cn-y,年,month:cn,day:cn,日,·,weekday]·十一月廿四 }
嘿,所有过去都在现在里!嘿,所有未来已经在我们身体里了!嘿!\footnote{\bi{宇宙重建了自身:佩索阿诗精选} \regular{费尔南多·佩索阿}}

\title{\date[d=6,m=1,y=2024][year:cn-y,年,month:cn,day:cn,日,·,weekday]·十一月廿五 ·小寒}
像它会善待宇宙,给它合乎舞台的衣裙宇宙也会善待圣者,给他一颗奥妙的内心\footnote{\bi{张枣的诗} \regular{张枣 颜炼军}}

\title{\date[d=7,m=1,y=2024][year:cn-y,年,month:cn,day:cn,日,·,weekday]·十一月廿六 }
让树叶永远沙沙作响 也不生出鸟的翅膀\footnote{\bi{顾城的诗} \regular{顾城}}

\title{\date[d=8,m=1,y=2024][year:cn-y,年,month:cn,day:cn,日,·,weekday]·十一月廿七 }
不会由此而逃入黄昏与肩头沉重的人兀自相遇\footnote{\bi{戈麦的诗} \regular{戈麦 西渡}}

\title{\date[d=9,m=1,y=2024][year:cn-y,年,month:cn,day:cn,日,·,weekday]·十一月廿八 }
死者用火交流思想,远远超过生者的语言。\footnote{\bi{荒原:艾略特文集·诗歌} \regular{T.S.艾略特}}

\title{\date[d=10,m=1,y=2024][year:cn-y,年,month:cn,day:cn,日,·,weekday]·十一月廿九 }
我几乎相信我永远没有醒着。我不知道我是在活着时做梦,还是在做梦时活着,或者梦和生活无非是在我身上混合、交叉的东西,在相互渗透中,形成了我有意识的存在。\footnote{\bi{不安之书} \regular{费尔南多·佩索阿 热罗尼莫·皮萨罗 }}

\title{\date[d=11,m=1,y=2024][year:cn-y,年,month:cn,day:cn,日,·,weekday]·腊月初一 }
而且把千万个忍辱负重的人藏在心窝里\footnote{\bi{黄灿然的诗} \regular{黄灿然}}

\title{\date[d=12,m=1,y=2024][year:cn-y,年,month:cn,day:cn,日,·,weekday]·腊月初二 }
每个人也同样囿于自己的意识。\footnote{\bi{人生的智慧} \regular{叔本华}}

\title{\date[d=13,m=1,y=2024][year:cn-y,年,month:cn,day:cn,日,·,weekday]·腊月初三 }
当月亮怀着闲愁偶尔向地球悄悄地洒下一颗泪珠的时候,有位虔诚的诗人,偏偏不能入梦,赶紧用自己的手心接住\footnote{\bi{恶之花} \regular{波德莱尔}}

\title{\date[d=14,m=1,y=2024][year:cn-y,年,month:cn,day:cn,日,·,weekday]·腊月初四 }
你让我受难吧!我无处不在:\footnote{\bi{茨维塔耶娃诗选} \regular{茨维塔耶娃}}

\title{\date[d=15,m=1,y=2024][year:cn-y,年,month:cn,day:cn,日,·,weekday]·腊月初五 }
无数的我急于与无数的我告别继而消失\footnote{\bi{雪是谁说的谎:倪湛舸诗集} \regular{倪湛舸}}

\title{\date[d=16,m=1,y=2024][year:cn-y,年,month:cn,day:cn,日,·,weekday]·腊月初六 }
世界不停拆除自己的营帐。风在夏日攥住橡树的船帆,把地球扔向前去。\footnote{\bi{沉石与火舌:特朗斯特罗姆诗全集} \regular{托马斯·特朗斯特罗姆}}

\title{\date[d=17,m=1,y=2024][year:cn-y,年,month:cn,day:cn,日,·,weekday]·腊月初七 }
不忍对别人开枪,就对自己开枪吧\footnote{\bi{欧阳江河的诗} \regular{欧阳江河}}

\title{\date[d=18,m=1,y=2024][year:cn-y,年,month:cn,day:cn,日,·,weekday]·腊月初八 ·腊八节}
但是在我们感情中有一种提前悲伤的痕迹,一种穿上旅服的伤痛,我们模糊地注意到万物缤纷地展开,风有着另一种声调,如果夜晚降临,古老的安宁会沿着宇宙那不可回避的存在而延伸。\footnote{\bi{不安之书} \regular{费尔南多·佩索阿 热罗尼莫·皮萨罗 }}

\title{\date[d=19,m=1,y=2024][year:cn-y,年,month:cn,day:cn,日,·,weekday]·腊月初九 }
技巧不能超过天赋,只能使天赋更加完美。为此,天赋加技巧,或技巧加天赋,方能造就完美之诗人。\footnote{\bi{堂吉诃德(译文名著精选)} \regular{塞万提斯}}

\title{\date[d=20,m=1,y=2024][year:cn-y,年,month:cn,day:cn,日,·,weekday]·腊月初十 ·大寒}
时间在我的体内静立,它拥有无穷的时间,拥有忘掉所有语言,发明“生生不息”所需要的时间。\footnote{\bi{沉石与火舌:特朗斯特罗姆诗全集} \regular{托马斯·特朗斯特罗姆}}

\title{\date[d=21,m=1,y=2024][year:cn-y,年,month:cn,day:cn,日,·,weekday]·腊月十一 }
恨使四海咸,\footnote{\bi{顾城的诗} \regular{顾城}}

\title{\date[d=22,m=1,y=2024][year:cn-y,年,month:cn,day:cn,日,·,weekday]·腊月十二 }
这种语言,综合了芳香、音响、色彩,概括一切,可以把思想与思想连结起来,又引出思想,这种语言将使心灵与心灵呼应相通。\footnote{\bi{彩画集:兰波散文诗全集(译文经典)} \regular{阿蒂尔·兰波}}

\title{\date[d=23,m=1,y=2024][year:cn-y,年,month:cn,day:cn,日,·,weekday]·腊月十三 }
反时间也在转动。\footnote{\bi{欧阳江河的诗} \regular{欧阳江河}}

\title{\date[d=24,m=1,y=2024][year:cn-y,年,month:cn,day:cn,日,·,weekday]·腊月十四 }
谁,倘若我叫喊,可以从天使的序列中听见我?\footnote{\bi{杜英诺悲歌:里尔克诗选(文学馆系列)} \regular{里尔克}}

\title{\date[d=25,m=1,y=2024][year:cn-y,年,month:cn,day:cn,日,·,weekday]·腊月十五 }
夜,你这黑太阳,请把我烧成灰!\footnote{\bi{茨维塔耶娃诗选} \regular{茨维塔耶娃}}

\title{\date[d=26,m=1,y=2024][year:cn-y,年,month:cn,day:cn,日,·,weekday]·腊月十六 }
是痛心的快乐\footnote{\bi{昌耀的诗} \regular{昌耀}}

\title{\date[d=27,m=1,y=2024][year:cn-y,年,month:cn,day:cn,日,·,weekday]·腊月十七 }
今天的钥匙在昨天很快地锈去\footnote{\bi{骆一禾的诗} \regular{骆一禾 西渡}}

\title{\date[d=28,m=1,y=2024][year:cn-y,年,month:cn,day:cn,日,·,weekday]·腊月十八 }
做梦蚕食世界之外的世界\footnote{\bi{雪是谁说的谎:倪湛舸诗集} \regular{倪湛舸}}

\title{\date[d=29,m=1,y=2024][year:cn-y,年,month:cn,day:cn,日,·,weekday]·腊月十九 }
从外面的世界带回来的就只一些梦,如一些饮空了的酒瓶,与他久别的乡土是应该给他一瓶未开封的新酿了\footnote{\bi{何其芳散文} \regular{何其芳}}

\title{\date[d=30,m=1,y=2024][year:cn-y,年,month:cn,day:cn,日,·,weekday]·腊月二十 }
我看到眼睛,但未看到泪水这是我的苦难。\footnote{\bi{荒原:艾略特文集·诗歌} \regular{T.S.艾略特}}

\title{\date[d=31,m=1,y=2024][year:cn-y,年,month:cn,day:cn,日,·,weekday]·腊月廿一 }
一种厌倦,只包含着对更多厌倦的预感;一种遗憾,明天我将遗憾于今天有过遗憾——全是混乱的纠葛,没有用途,没有真相,混乱的纠葛……\footnote{\bi{不安之书} \regular{费尔南多·佩索阿 热罗尼莫·皮萨罗 }}

\title{\date[d=1,m=2,y=2024][year:cn-y,年,month:cn,day:cn,日,·,weekday]·腊月廿二 }
你总是问:“你究竟从哪儿来的这奇怪的忧愁,如海潮涌向那光秃秃的幽暗岩石?”\footnote{\bi{恶之花} \regular{波德莱尔}}

\title{\date[d=2,m=2,y=2024][year:cn-y,年,month:cn,day:cn,日,·,weekday]·腊月廿三 }
我曾是您的青春, 青春在一旁溜走。\footnote{\bi{茨维塔耶娃诗选} \regular{茨维塔耶娃}}

\title{\date[d=3,m=2,y=2024][year:cn-y,年,month:cn,day:cn,日,·,weekday]·腊月廿四 }
努力是一种荒谬的浪费,生命是一场空,因为幻灭总是紧随在幻想之后而死亡似乎是生命的意义……\footnote{\bi{宇宙重建了自身:佩索阿诗精选} \regular{费尔南多·佩索阿}}

\title{\date[d=4,m=2,y=2024][year:cn-y,年,month:cn,day:cn,日,·,weekday]·腊月廿五 ·立春}
纵然只有黑夜,白昼也如火炽。\footnote{\bi{彩画集:兰波散文诗全集(译文经典)} \regular{阿蒂尔·兰波}}

\title{\date[d=5,m=2,y=2024][year:cn-y,年,month:cn,day:cn,日,·,weekday]·腊月廿六 }
天空收下了鸟群泥土保存着树根\footnote{\bi{顾城的诗} \regular{顾城}}

\title{\date[d=6,m=2,y=2024][year:cn-y,年,month:cn,day:cn,日,·,weekday]·腊月廿七 }
最伟大的爱就是死亡、遗忘或者放弃——所有的爱都是对爱的深恶痛绝。\footnote{\bi{不安之书} \regular{费尔南多·佩索阿 热罗尼莫·皮萨罗 }}

\title{\date[d=7,m=2,y=2024][year:cn-y,年,month:cn,day:cn,日,·,weekday]·腊月廿八 }
手来回搬弄发声的重量,仿佛我们在触摸轻重,试图打破秤杆可怕的平衡:痛苦与欢乐半斤八两\footnote{\bi{沉石与火舌:特朗斯特罗姆诗全集} \regular{托马斯·特朗斯特罗姆}}

\title{\date[d=8,m=2,y=2024][year:cn-y,年,month:cn,day:cn,日,·,weekday]·腊月廿九 }
外界时而冰凉,时而火热。或者说,外面的世界时而是圆点状,时而又变成条纹状。我的内部世界跟外界进行着平缓无序的交换。就像周围毫无意义的风景映入眼帘,我也同样进入风景,没进入的部分则在远方兀自绽放和闪耀\footnote{\bi{金阁寺} \regular{三岛由纪夫}}

\title{\date[d=9,m=2,y=2024][year:cn-y,年,month:cn,day:cn,日,·,weekday]·腊月三十 ·除夕}
而那无限者从四面八方如此亲密地渐渐化为他,使他得以相信自己感觉到在此期间潜入的星辰轻轻靠在他胸中。\footnote{\bi{谁此时孤独:里尔克晚期书信选} \regular{里尔克}}

\title{\date[d=10,m=2,y=2024][year:cn-y,年,month:cn,day:cn,日,·,weekday]·正月初一 ·春节}
这样,在连续的图像中描述自己——不无真理,但也有谎言——我在图像中比在我自身停留得更久,自我讲述着,直到不存在,以灵魂作为墨水来书写,这灵魂除了被用来书写,别无他用\footnote{\bi{不安之书} \regular{费尔南多·佩索阿 热罗尼莫·皮萨罗 }}

\title{\date[d=11,m=2,y=2024][year:cn-y,年,month:cn,day:cn,日,·,weekday]·正月初二 }
我是建筑的废墟,从未比这更多,有人施工到一半,就懒得去想他要建造什么了。\footnote{\bi{不安之书} \regular{费尔南多·佩索阿 热罗尼莫·皮萨罗 }}

\title{\date[d=12,m=2,y=2024][year:cn-y,年,month:cn,day:cn,日,·,weekday]·正月初三 }
她不是在拥抱的黑暗中掘取满足,而是翻找渴望。\footnote{\bi{布里格手记} \regular{里尔克}}

\title{\date[d=13,m=2,y=2024][year:cn-y,年,month:cn,day:cn,日,·,weekday]·正月初四 }
去与所有地方相反的地方!去那里是向前,也是退后,\footnote{\bi{茨维塔耶娃诗选} \regular{茨维塔耶娃}}

\title{\date[d=14,m=2,y=2024][year:cn-y,年,month:cn,day:cn,日,·,weekday]·正月初五 }
永远是涵和忽 永远是光和无的通明\footnote{\bi{骆一禾的诗} \regular{骆一禾 西渡}}

\title{\date[d=15,m=2,y=2024][year:cn-y,年,month:cn,day:cn,日,·,weekday]·正月初六 }
我全然是一种模糊的怀想,不是对过去,也不是对未来:我是一种对当前的怀想,匿名、冗长而又不被理解。\footnote{\bi{不安之书} \regular{费尔南多·佩索阿 热罗尼莫·皮萨罗 }}

\title{\date[d=16,m=2,y=2024][year:cn-y,年,month:cn,day:cn,日,·,weekday]·正月初七 }
黑夜哭泣时,月亮周围的晕。\footnote{\bi{宇宙重建了自身:佩索阿诗精选} \regular{费尔南多·佩索阿}}

\title{\date[d=17,m=2,y=2024][year:cn-y,年,month:cn,day:cn,日,·,weekday]·正月初八 }
蜃气飘摇的地表\footnote{\bi{昌耀的诗} \regular{昌耀}}

\title{\date[d=18,m=2,y=2024][year:cn-y,年,month:cn,day:cn,日,·,weekday]·正月初九 }
云,正在聚集正在无声无息地哭咸咸的,哭\footnote{\bi{顾城的诗} \regular{顾城}}

\title{\date[d=19,m=2,y=2024][year:cn-y,年,month:cn,day:cn,日,·,weekday]·正月初十 ·雨水}
人总是觉得自己要比实际年轻。我的内心带着自己早期的面孔,就像树带着自己的年轮。它们的总和就是“自我”。镜子只看到我后来的面孔,我熟悉我早年所有的脸。\footnote{\bi{沉石与火舌:特朗斯特罗姆诗全集} \regular{托马斯·特朗斯特罗姆}}

\title{\date[d=20,m=2,y=2024][year:cn-y,年,month:cn,day:cn,日,·,weekday]·正月十一 }
我睡,而又不眠。\footnote{\bi{不安之书} \regular{费尔南多·佩索阿 热罗尼莫·皮萨罗 }}

\title{\date[d=21,m=2,y=2024][year:cn-y,年,month:cn,day:cn,日,·,weekday]·正月十二 }
给风的预言,只给风,因为只有风会倾听\footnote{\bi{荒原:艾略特文集·诗歌} \regular{T.S.艾略特}}

\title{\date[d=22,m=2,y=2024][year:cn-y,年,month:cn,day:cn,日,·,weekday]·正月十三 }
我的梦是一个愚蠢的避难所,就像一把遮挡闪电的雨伞。\footnote{\bi{不安之书} \regular{费尔南多·佩索阿 热罗尼莫·皮萨罗 }}

\title{\date[d=23,m=2,y=2024][year:cn-y,年,month:cn,day:cn,日,·,weekday]·正月十四 }
灯,从门窗向外生活\footnote{\bi{海子的诗} \regular{海子}}

\title{\date[d=24,m=2,y=2024][year:cn-y,年,month:cn,day:cn,日,·,weekday]·正月十五 ·元宵节}
梦想那妖异的爱情和奇幻的宇宙\footnote{\bi{彩画集:兰波散文诗全集(译文经典)} \regular{阿蒂尔·兰波}}

\title{\date[d=25,m=2,y=2024][year:cn-y,年,month:cn,day:cn,日,·,weekday]·正月十六 }
心最后总要滚动一下才能变成石子\footnote{\bi{顾城的诗} \regular{顾城}}

\title{\date[d=26,m=2,y=2024][year:cn-y,年,month:cn,day:cn,日,·,weekday]·正月十七 }
凡是我认为象征沉睡的东西,都有一种万物终结的声音,黑暗中的风声,如果再仔细听,还有我心肺的声音。\footnote{\bi{不安之书} \regular{费尔南多·佩索阿 热罗尼莫·皮萨罗 }}

\title{\date[d=27,m=2,y=2024][year:cn-y,年,month:cn,day:cn,日,·,weekday]·正月十八 }
我在你身上失去从来不曾有过的所有人!怎样的渴望,当畅饮被你充斥的空气!既然纳克索斯岛[6]是我的骨!既然我皮肤下的血是冥河!徒劳充满我的身体\footnote{\bi{茨维塔耶娃诗选} \regular{茨维塔耶娃}}

\title{\date[d=28,m=2,y=2024][year:cn-y,年,month:cn,day:cn,日,·,weekday]·正月十九 }
那里海浪相遇海浪。\footnote{\bi{荒原:艾略特文集·诗歌} \regular{T.S.艾略特}}

\title{\date[d=29,m=2,y=2024][year:cn-y,年,month:cn,day:cn,日,·,weekday]·正月二十 }
“外在”虽然如此广延,虽有恒星的一切距离,但它很难与这些维度、与我们内心的深层维度相比拟,内心根本无需宇宙之广袤,亦可在自身之中几乎了无止境。\footnote{\bi{谁此时孤独:里尔克晚期书信选} \regular{里尔克}}

\title{\date[d=1,m=3,y=2024][year:cn-y,年,month:cn,day:cn,日,·,weekday]·正月廿一 }
孤僻把我塑造成与它相似的形象。\footnote{\bi{不安之书} \regular{费尔南多·佩索阿 热罗尼莫·皮萨罗 }}

\title{\date[d=2,m=3,y=2024][year:cn-y,年,month:cn,day:cn,日,·,weekday]·正月廿二 }
假如夜不仅仅是光的缺失,假如夜确实是某样东西。那么夜就是这声音\footnote{\bi{沉石与火舌:特朗斯特罗姆诗全集} \regular{托马斯·特朗斯特罗姆}}

\title{\date[d=3,m=3,y=2024][year:cn-y,年,month:cn,day:cn,日,·,weekday]·正月廿三 }
而我们把一生的大部分时光用在醒的边缘上。\footnote{\bi{黄灿然的诗} \regular{黄灿然}}

\title{\date[d=4,m=3,y=2024][year:cn-y,年,month:cn,day:cn,日,·,weekday]·正月廿四 }
假如雨是朝天空那个方向下就好啦 意味着一种拯救\footnote{\bi{于坚的诗} \regular{于坚}}

\title{\date[d=5,m=3,y=2024][year:cn-y,年,month:cn,day:cn,日,·,weekday]·正月廿五 ·惊蛰}
没有人能猜想到,在我旁边,永远有另一个人,但说到底还是我。人们总是以为我和我完全一致。\footnote{\bi{不安之书} \regular{费尔南多·佩索阿 热罗尼莫·皮萨罗 }}

\title{\date[d=6,m=3,y=2024][year:cn-y,年,month:cn,day:cn,日,·,weekday]·正月廿六 }
我湮没在沉沉无声的夜和幸福遗失之中\footnote{\bi{彩画集:兰波散文诗全集(译文经典)} \regular{阿蒂尔·兰波}}

\title{\date[d=7,m=3,y=2024][year:cn-y,年,month:cn,day:cn,日,·,weekday]·正月廿七 }
只有大海没入大海,大海变得更深万物之流一片轰轰作响另一个世界正在轰鸣\footnote{\bi{骆一禾的诗} \regular{骆一禾 西渡}}

\title{\date[d=8,m=3,y=2024][year:cn-y,年,month:cn,day:cn,日,·,weekday]·正月廿八 }
嗅觉是一种奇怪的视野。它以一种潜意识的、突然的描绘,勾起多愁善感的风景。\footnote{\bi{不安之书} \regular{费尔南多·佩索阿 热罗尼莫·皮萨罗 }}

\title{\date[d=9,m=3,y=2024][year:cn-y,年,month:cn,day:cn,日,·,weekday]·正月廿九 }
问天,不如问山鬼。\footnote{\bi{欧阳江河的诗} \regular{欧阳江河}}

\title{\date[d=10,m=3,y=2024][year:cn-y,年,month:cn,day:cn,日,·,weekday]·二月初一 }
黑暗是怎样地在你身上掠夺\footnote{\bi{芒克的诗} \regular{芒克}}

\title{\date[d=11,m=3,y=2024][year:cn-y,年,month:cn,day:cn,日,·,weekday]·二月初二 ·龙抬头}
黑磁铁之夜有如沉思者吸紧空旷。钥匙吮着世界。\footnote{\bi{张枣的诗} \regular{张枣 颜炼军}}

\title{\date[d=12,m=3,y=2024][year:cn-y,年,month:cn,day:cn,日,·,weekday]·二月初三 }
当我歌唱起来这街道就是属于我的我把它称作六弦琴\footnote{\bi{骆一禾的诗} \regular{骆一禾 西渡}}

\title{\date[d=13,m=3,y=2024][year:cn-y,年,month:cn,day:cn,日,·,weekday]·二月初四 }
我来到这里,却谁也不等,只观察所有的别人的等待,成为等待着的所有的别人,成为所有别人的焦灼的等待。\footnote{\bi{想象一朵未来的玫瑰:佩索阿诗选} \regular{费尔南多·佩索阿}}

\title{\date[d=14,m=3,y=2024][year:cn-y,年,month:cn,day:cn,日,·,weekday]·二月初五 }
远方就是这样的,就是我站立的地方\footnote{\bi{海子的诗} \regular{海子}}

\title{\date[d=15,m=3,y=2024][year:cn-y,年,month:cn,day:cn,日,·,weekday]·二月初六 }
瞻望永恒的梦抵达以太之上以太之上,大质量的烟,大质量的柱子,棋局缜密而清晰,什么样的数学,什么样的对弈者\footnote{\bi{戈麦的诗} \regular{戈麦 西渡}}

\title{\date[d=16,m=3,y=2024][year:cn-y,年,month:cn,day:cn,日,·,weekday]·二月初七 }
谁与我同享暮色的金黄然后一起退入月亮宝石?\footnote{\bi{昌耀的诗} \regular{昌耀}}

\title{\date[d=17,m=3,y=2024][year:cn-y,年,month:cn,day:cn,日,·,weekday]·二月初八 }
一切皆属过程,凡应发生者皆不可避免。凡已发生者仍将如是\footnote{\bi{昌耀的诗} \regular{昌耀}}

\title{\date[d=18,m=3,y=2024][year:cn-y,年,month:cn,day:cn,日,·,weekday]·二月初九 }
因为所有这一切——天,地,世界,——所有这一切中的能有的只是我自己!\footnote{\bi{不安之书} \regular{费尔南多·佩索阿 热罗尼莫·皮萨罗 }}

\title{\date[d=19,m=3,y=2024][year:cn-y,年,month:cn,day:cn,日,·,weekday]·二月初十 }
在我的心里,你想种下什么\footnote{\bi{芒克的诗} \regular{芒克}}

\title{\date[d=20,m=3,y=2024][year:cn-y,年,month:cn,day:cn,日,·,weekday]·二月十一 ·春分}
我知道了我有两次生命一次还没结束一次刚刚开始\footnote{\bi{顾城的诗} \regular{顾城}}

\title{\date[d=21,m=3,y=2024][year:cn-y,年,month:cn,day:cn,日,·,weekday]·二月十二 }
这得等那些话在你心里成熟才行。\footnote{\bi{毛毛:时间窃贼和一个小女孩的不可思议的故事} \regular{米切尔·恩德}}

\title{\date[d=22,m=3,y=2024][year:cn-y,年,month:cn,day:cn,日,·,weekday]·二月十三 }
但凡进入他的记忆,人就存在着,死亡也改变不了什么。\footnote{\bi{布里格手记} \regular{里尔克}}

\title{\date[d=23,m=3,y=2024][year:cn-y,年,month:cn,day:cn,日,·,weekday]·二月十四 }
成为自己身上的死者\footnote{\bi{欧阳江河的诗} \regular{欧阳江河}}

\title{\date[d=24,m=3,y=2024][year:cn-y,年,month:cn,day:cn,日,·,weekday]·二月十五 }
所以, 人们节省的时间越多, 属于他们自己的时间就越少。\footnote{\bi{毛毛:时间窃贼和一个小女孩的不可思议的故事} \regular{米切尔·恩德}}

\title{\date[d=25,m=3,y=2024][year:cn-y,年,month:cn,day:cn,日,·,weekday]·二月十六 }
静得像擦过一样\footnote{\bi{多多的诗} \regular{多多}}

\title{\date[d=26,m=3,y=2024][year:cn-y,年,month:cn,day:cn,日,·,weekday]·二月十七 }
我思念的片断是一只在雨后的田野里爬满露水的南瓜这思念在夏日的流水中与女人的体温交谈\footnote{\bi{于坚的诗} \regular{于坚}}

\title{\date[d=27,m=3,y=2024][year:cn-y,年,month:cn,day:cn,日,·,weekday]·二月十八 }
命运的洪流带来苦难,直泻而下,来势凶猛。世上没有力量可以阻止,人间没有办法能够预防,这难道不是千真万确的吗?\footnote{\bi{堂吉诃德(译文名著精选)} \regular{塞万提斯}}

\title{\date[d=28,m=3,y=2024][year:cn-y,年,month:cn,day:cn,日,·,weekday]·二月十九 }
那些颜色杂乱的烟被风抽成细丝\footnote{\bi{顾城的诗} \regular{顾城}}

\title{\date[d=29,m=3,y=2024][year:cn-y,年,month:cn,day:cn,日,·,weekday]·二月二十 }
风一直在领航,指引的是海上的波浪波浪一直在荡,海面上延伸的钟磬一直在响\footnote{\bi{戈麦的诗} \regular{戈麦 西渡}}

\title{\date[d=30,m=3,y=2024][year:cn-y,年,month:cn,day:cn,日,·,weekday]·二月廿一 }
悠缓的黄昏仿佛自事物深处而来,向远方的悲伤伸出双手,与之达成精神的和解,又从天空高阔的寂静俯身,贴近灵魂;\footnote{\bi{不安之书} \regular{费尔南多·佩索阿 热罗尼莫·皮萨罗 }}

\title{\date[d=31,m=3,y=2024][year:cn-y,年,month:cn,day:cn,日,·,weekday]·二月廿二 }
特定的启示好像正是从自己闻所未闻的一次性中,在伟大、悲情和人性上取得最难以置信的收获。与唯一之物以及不可挽回的消逝之物打交道,这也还是(除了苦难)我们的灵魂的一种强项和一种骄傲。\footnote{\bi{谁此时孤独:里尔克晚期书信选} \regular{里尔克}}

\title{\date[d=1,m=4,y=2024][year:cn-y,年,month:cn,day:cn,日,·,weekday]·二月廿三 }
一个倾听着宇宙万物的大耳轮中间\footnote{\bi{毛毛:时间窃贼和一个小女孩的不可思议的故事} \regular{米切尔·恩德}}

\title{\date[d=2,m=4,y=2024][year:cn-y,年,month:cn,day:cn,日,·,weekday]·二月廿四 }
两千只眼睛同时醒来 是我的幸福\footnote{\bi{骆一禾的诗} \regular{骆一禾 西渡}}

\title{\date[d=3,m=4,y=2024][year:cn-y,年,month:cn,day:cn,日,·,weekday]·二月廿五 }
打伤我,撕开我,杀了我!我想要的一切是将一颗溢出大海的灵魂带给死神,\footnote{\bi{宇宙重建了自身:佩索阿诗精选} \regular{费尔南多·佩索阿}}

\title{\date[d=4,m=4,y=2024][year:cn-y,年,month:cn,day:cn,日,·,weekday]·二月廿六 ·清明}
你这被欢乐那灼热的闪电所燃烧的灵魂就这样迅速又勇敢地冲向无边无际令人喜悦的穹苍。\footnote{\bi{恶之花} \regular{波德莱尔}}

\title{\date[d=5,m=4,y=2024][year:cn-y,年,month:cn,day:cn,日,·,weekday]·二月廿七 }
芳香、色彩、声响纷纷相互呼应。\footnote{\bi{恶之花} \regular{波德莱尔}}

\title{\date[d=6,m=4,y=2024][year:cn-y,年,month:cn,day:cn,日,·,weekday]·二月廿八 }
死亡是地球的一只眼睛\footnote{\bi{骆一禾的诗} \regular{骆一禾 西渡}}

\title{\date[d=7,m=4,y=2024][year:cn-y,年,month:cn,day:cn,日,·,weekday]·二月廿九 }
倘若我因属于你而将驻留与永生,倘若失去你就是找到你?\footnote{\bi{不安之书} \regular{费尔南多·佩索阿 热罗尼莫·皮萨罗 }}

\title{\date[d=8,m=4,y=2024][year:cn-y,年,month:cn,day:cn,日,·,weekday]·二月三十 }
死深藏于爱的本质之中\footnote{\bi{谁此时孤独:里尔克晚期书信选} \regular{里尔克}}

\title{\date[d=9,m=4,y=2024][year:cn-y,年,month:cn,day:cn,日,·,weekday]·三月初一 }
天空珍贵得好像珠子\footnote{\bi{骆一禾的诗} \regular{骆一禾 西渡}}

\title{\date[d=10,m=4,y=2024][year:cn-y,年,month:cn,day:cn,日,·,weekday]·三月初二 }
也许是为了以一种恍如隔世的目光看生活?或是在回头的一刻再次产生“我是否就在那里”的无端追问?\footnote{\bi{王家新的诗} \regular{王家新}}

\title{\date[d=11,m=4,y=2024][year:cn-y,年,month:cn,day:cn,日,·,weekday]·三月初三 }
能不能把那个好时辰还给我,能不能把那善意的手臂伸给我?\footnote{\bi{彩画集:兰波散文诗全集(译文经典)} \regular{阿蒂尔·兰波}}

\title{\date[d=12,m=4,y=2024][year:cn-y,年,month:cn,day:cn,日,·,weekday]·三月初四 }
原来我的手臂是凉的\footnote{\bi{西川的诗} \regular{西川}}

\title{\date[d=13,m=4,y=2024][year:cn-y,年,month:cn,day:cn,日,·,weekday]·三月初五 }
你呀这红红绿绿的夜又不知该怎样地把我折磨\footnote{\bi{芒克的诗} \regular{芒克}}

\title{\date[d=14,m=4,y=2024][year:cn-y,年,month:cn,day:cn,日,·,weekday]·三月初六 }
风时时刻刻在雕刻、比盾牌更坚硬的那些人。\footnote{\bi{雪是谁说的谎:倪湛舸诗集} \regular{倪湛舸}}

\title{\date[d=15,m=4,y=2024][year:cn-y,年,month:cn,day:cn,日,·,weekday]·三月初七 }
完整定义的艺术即和谐表达出我们所意识到的感觉”\footnote{\bi{自决之书} \regular{费尔南多·佩索阿}}

\title{\date[d=16,m=4,y=2024][year:cn-y,年,month:cn,day:cn,日,·,weekday]·三月初八 }
欢乐与痛苦在露珠的放大镜里膨胀。我们其实并不知道这一点,只是感到:我们的生活有一条姐妹船,在一条截然不同的航道上行驶。当太阳在群岛的背后燃烧\footnote{\bi{沉石与火舌:特朗斯特罗姆诗全集} \regular{托马斯·特朗斯特罗姆}}

\title{\date[d=17,m=4,y=2024][year:cn-y,年,month:cn,day:cn,日,·,weekday]·三月初九 }
泪水的故乡,泪水之乡也是心愿之乡心愿在河上摆渡,不能说生活是妄想遗忘的摇篮,遗忘的谷仓\footnote{\bi{戈麦的诗} \regular{戈麦 西渡}}

\title{\date[d=18,m=4,y=2024][year:cn-y,年,month:cn,day:cn,日,·,weekday]·三月初十 }
这赋予我的脸一种甚至比我的童年还老的老态,让我的凝视在幸福中流露出一丝焦虑。\footnote{\bi{宇宙重建了自身:佩索阿诗精选} \regular{费尔南多·佩索阿}}

\title{\date[d=19,m=4,y=2024][year:cn-y,年,month:cn,day:cn,日,·,weekday]·三月十一 ·谷雨}
情绪也是记忆的一种,甚至是更深刻的记忆。\footnote{\bi{重影(著名诗人北岛第一部摄影诗文集,摄影与诗歌相遇,感受光影与文字的相映成趣。摄影给了诗人另一双眼睛,也从另一个维度帮助我们抵近诗人的灵魂。)} \regular{北岛}}

\title{\date[d=20,m=4,y=2024][year:cn-y,年,month:cn,day:cn,日,·,weekday]·三月十二 }
世界在下坠,落日高不可问\footnote{\bi{欧阳江河的诗} \regular{欧阳江河}}

\title{\date[d=21,m=4,y=2024][year:cn-y,年,month:cn,day:cn,日,·,weekday]·三月十三 }
但风景在呈现动态时,并无所意欲。\footnote{\bi{里尔克全集 第九卷 沃普斯韦德、奥古斯特·罗丹} \regular{莱纳.马利亚.里克尔 叶廷芳}}

\title{\date[d=22,m=4,y=2024][year:cn-y,年,month:cn,day:cn,日,·,weekday]·三月十四 }
我最早可追溯的记忆是一种感觉\footnote{\bi{沉石与火舌:特朗斯特罗姆诗全集} \regular{托马斯·特朗斯特罗姆}}

\title{\date[d=23,m=4,y=2024][year:cn-y,年,month:cn,day:cn,日,·,weekday]·三月十五 }
风之语,既非回答,也概不提问。\footnote{\bi{欧阳江河的诗} \regular{欧阳江河}}

\title{\date[d=24,m=4,y=2024][year:cn-y,年,month:cn,day:cn,日,·,weekday]·三月十六 }
有的是一种巨大的消除,它抹去一切做成的动作,而不是那种潜在的疲倦,来自从未做过的动作。\footnote{\bi{不安之书} \regular{费尔南多·佩索阿 热罗尼莫·皮萨罗 }}

\title{\date[d=25,m=4,y=2024][year:cn-y,年,month:cn,day:cn,日,·,weekday]·三月十七 }
我去做所有人的声音\footnote{\bi{芒克的诗} \regular{芒克}}

\title{\date[d=26,m=4,y=2024][year:cn-y,年,month:cn,day:cn,日,·,weekday]·三月十八 }
我喜欢沉重的地球从未在我们脚下漂移。\footnote{\bi{茨维塔耶娃诗选} \regular{茨维塔耶娃}}

\title{\date[d=27,m=4,y=2024][year:cn-y,年,month:cn,day:cn,日,·,weekday]·三月十九 }
我心里有种东西断裂了。红色已变成黄昏。我感到了太多东西,以至于无法继续感觉。我的灵魂已经耗尽,只剩下我心中的回声。飞轮正在慢下来。我的梦将它们的手从我的眼睛上抬起来一点。我心里一无所有,除了空虚,沙漠,夜间的大海。\footnote{\bi{宇宙重建了自身:佩索阿诗精选} \regular{费尔南多·佩索阿}}

\title{\date[d=28,m=4,y=2024][year:cn-y,年,month:cn,day:cn,日,·,weekday]·三月二十 }
多余的暮色压到我的帐篷上\footnote{\bi{西川的诗} \regular{西川}}

\title{\date[d=29,m=4,y=2024][year:cn-y,年,month:cn,day:cn,日,·,weekday]·三月廿一 }
千百个太阳从缝隙飞入。被倒置的重量引力主宰着光的游戏\footnote{\bi{沉石与火舌:特朗斯特罗姆诗全集} \regular{托马斯·特朗斯特罗姆}}

\title{\date[d=30,m=4,y=2024][year:cn-y,年,month:cn,day:cn,日,·,weekday]·三月廿二 }
因为最高级、最丰富多彩以及维持最为恒久的乐趣是精神思想上的乐趣\footnote{\bi{人生的智慧} \regular{叔本华}}

\title{\date[d=1,m=5,y=2024][year:cn-y,年,month:cn,day:cn,日,·,weekday]·三月廿三 }
重新全神贯注,迷失在自我之中,在漫漫长夜里忘却自己,这些夜晚尚未被责任和世界玷污,是不知神秘和未来为何物的处子。\footnote{\bi{不安之书} \regular{费尔南多·佩索阿 热罗尼莫·皮萨罗 }}

\title{\date[d=2,m=5,y=2024][year:cn-y,年,month:cn,day:cn,日,·,weekday]·三月廿四 }
欲留在鸡巴之前\footnote{\bi{西川的诗} \regular{西川}}

\title{\date[d=3,m=5,y=2024][year:cn-y,年,month:cn,day:cn,日,·,weekday]·三月廿五 }
我们是自我意识中逐渐暗淡的风景……由于风景分现实与虚幻两种,我们也含混地一分为二,谁也不清楚另一个究竟是不是自己,抑或那个影影绰绰的他是否活着……\footnote{\bi{不安之书} \regular{费尔南多·佩索阿 热罗尼莫·皮萨罗 }}

\title{\date[d=4,m=5,y=2024][year:cn-y,年,month:cn,day:cn,日,·,weekday]·三月廿六 }
与人的每一次相处都是一个岛,并且是一个永远沉没的岛——完全沉没,无踪无影。\footnote{\bi{抒情诗的呼吸:一九二六年书信(帕斯捷尔纳克作品系列)} \regular{鲍·列·帕斯捷尔纳克 玛·伊·茨维塔耶娃 莱·马·里尔克}}

\title{\date[d=5,m=5,y=2024][year:cn-y,年,month:cn,day:cn,日,·,weekday]·三月廿七 ·立夏}
你发现了吗,我是在零星地把自己给你?\footnote{\bi{抒情诗的呼吸:一九二六年书信(帕斯捷尔纳克作品系列)} \regular{鲍·列·帕斯捷尔纳克 玛·伊·茨维塔耶娃 莱·马·里尔克}}

\title{\date[d=6,m=5,y=2024][year:cn-y,年,month:cn,day:cn,日,·,weekday]·三月廿八 }
桥把自己慢慢筑入天空。\footnote{\bi{沉石与火舌:特朗斯特罗姆诗全集} \regular{托马斯·特朗斯特罗姆}}

\title{\date[d=7,m=5,y=2024][year:cn-y,年,month:cn,day:cn,日,·,weekday]·三月廿九 }
生活就是车站,我很快就要离去了,去哪儿——我不说。\footnote{\bi{抒情诗的呼吸:一九二六年书信(帕斯捷尔纳克作品系列)} \regular{鲍·列·帕斯捷尔纳克 玛·伊·茨维塔耶娃 莱·马·里尔克}}

\title{\date[d=8,m=5,y=2024][year:cn-y,年,month:cn,day:cn,日,·,weekday]·四月初一 }
我把白昼一饮而干金酒瓶抛给上帝\footnote{\bi{于坚的诗} \regular{于坚}}

\title{\date[d=9,m=5,y=2024][year:cn-y,年,month:cn,day:cn,日,·,weekday]·四月初二 }
却宁愿期待着渐增的隔阂会在未来和解,而和解愈是遥遥无期,愈是令人着迷。\footnote{\bi{布里格手记} \regular{里尔克}}

\title{\date[d=10,m=5,y=2024][year:cn-y,年,month:cn,day:cn,日,·,weekday]·四月初三 }
它们聚集在体内,成为一种没有生活过、被摈斥、被遗弃的生命,能以使我们死去\footnote{\bi{给青年诗人的信} \regular{莱内·马利亚·里尔克}}

\title{\date[d=11,m=5,y=2024][year:cn-y,年,month:cn,day:cn,日,·,weekday]·四月初四 }
音乐就会在你的上空无垠展开\footnote{\bi{王家新的诗} \regular{王家新}}

\title{\date[d=12,m=5,y=2024][year:cn-y,年,month:cn,day:cn,日,·,weekday]·四月初五 }
太阳朝着没有人的地方走去了\footnote{\bi{芒克的诗} \regular{芒克}}

\title{\date[d=13,m=5,y=2024][year:cn-y,年,month:cn,day:cn,日,·,weekday]·四月初六 }
“星星就是自由。\footnote{\bi{布里格手记} \regular{里尔克}}

\title{\date[d=14,m=5,y=2024][year:cn-y,年,month:cn,day:cn,日,·,weekday]·四月初七 }
而是出于可能\footnote{\bi{西川的诗} \regular{西川}}

\title{\date[d=15,m=5,y=2024][year:cn-y,年,month:cn,day:cn,日,·,weekday]·四月初八 }
半坡的鸟道、\footnote{\bi{昌耀的诗} \regular{昌耀}}

\title{\date[d=16,m=5,y=2024][year:cn-y,年,month:cn,day:cn,日,·,weekday]·四月初九 }
精神上的搏斗和人与人之间的战斗一样激烈残酷;\footnote{\bi{彩画集:兰波散文诗全集(译文经典)} \regular{阿蒂尔·兰波}}

\title{\date[d=17,m=5,y=2024][year:cn-y,年,month:cn,day:cn,日,·,weekday]·四月初十 }
我像日暮的结尾一样漫游在风景发生之间。眼皮重重地压到我拖着的脚上。\footnote{\bi{不安之书} \regular{费尔南多·佩索阿 热罗尼莫·皮萨罗 }}

\title{\date[d=18,m=5,y=2024][year:cn-y,年,month:cn,day:cn,日,·,weekday]·四月十一 }
一种类似说话的哭泣声\footnote{\bi{多多的诗} \regular{多多}}

\title{\date[d=19,m=5,y=2024][year:cn-y,年,month:cn,day:cn,日,·,weekday]·四月十二 }
一个这样的图像不必拘泥于言辞,它活着是靠自身飘忽不定,它由此更新自己,并非它仿佛不确切并意欲始终如此。\footnote{\bi{谁此时孤独:里尔克晚期书信选} \regular{里尔克}}

\title{\date[d=20,m=5,y=2024][year:cn-y,年,month:cn,day:cn,日,·,weekday]·四月十三 ·小满}
把自己身上省略掉的部分,看作人类心灵的终极欠缺。\footnote{\bi{欧阳江河的诗} \regular{欧阳江河}}

\title{\date[d=21,m=5,y=2024][year:cn-y,年,month:cn,day:cn,日,·,weekday]·四月十四 }
在落日中酿造着黄色啤酒\footnote{\bi{于坚的诗} \regular{于坚}}

\title{\date[d=22,m=5,y=2024][year:cn-y,年,month:cn,day:cn,日,·,weekday]·四月十五 }
令人吃惊的是,痛苦却使人康复、使人年轻了。我突然看见了自己的久未察觉的生活。\footnote{\bi{抒情诗的呼吸:一九二六年书信(帕斯捷尔纳克作品系列)} \regular{鲍·列·帕斯捷尔纳克 玛·伊·茨维塔耶娃 莱·马·里尔克}}

\title{\date[d=23,m=5,y=2024][year:cn-y,年,month:cn,day:cn,日,·,weekday]·四月十六 }
网状的雾\footnote{\bi{戈麦的诗} \regular{戈麦 西渡}}

\title{\date[d=24,m=5,y=2024][year:cn-y,年,month:cn,day:cn,日,·,weekday]·四月十七 }
我向你献上这本书,因为我知道它美丽而无用。它什么也不教,不鼓吹信仰,也不叫你感受。它是流向灰烬深渊的小溪,风吹散那些灰,既不会让土地变得肥沃,也没有任何伤害,◊——我把全部的灵魂投入创作,但我书写时并没有思考它,我只想到了忧伤的自己和无名的你。\footnote{\bi{不安之书} \regular{费尔南多·佩索阿 热罗尼莫·皮萨罗 }}

\title{\date[d=25,m=5,y=2024][year:cn-y,年,month:cn,day:cn,日,·,weekday]·四月十八 }
分手,就是分开走,我们,原本连体出生……\footnote{\bi{茨维塔耶娃诗选} \regular{茨维塔耶娃}}

\title{\date[d=26,m=5,y=2024][year:cn-y,年,month:cn,day:cn,日,·,weekday]·四月十九 }
在这种苦命的声调中,我所听到的却恰好是,这样的幸福无法用双手制造出来,也无法通过强求获得。可是我该如何地强求,才能使你成为一个幸福的人呢?才能把你唤出来与我共处一个时辰呢?\footnote{\bi{抒情诗的呼吸:一九二六年书信(帕斯捷尔纳克作品系列)} \regular{鲍·列·帕斯捷尔纳克 玛·伊·茨维塔耶娃 莱·马·里尔克}}

\title{\date[d=27,m=5,y=2024][year:cn-y,年,month:cn,day:cn,日,·,weekday]·四月二十 }
白日消磨肠断句,世间只有情难诉。\footnote{\bi{牡丹亭} \regular{汤显祖}}

\title{\date[d=28,m=5,y=2024][year:cn-y,年,month:cn,day:cn,日,·,weekday]·四月廿一 }
学诗的尽头是火红的窑火\footnote{\bi{骆一禾的诗} \regular{骆一禾 西渡}}

\title{\date[d=29,m=5,y=2024][year:cn-y,年,month:cn,day:cn,日,·,weekday]·四月廿二 }
一场大雪必定闷死了世界\footnote{\bi{彩画集:兰波散文诗全集(译文经典)} \regular{阿蒂尔·兰波}}

\title{\date[d=30,m=5,y=2024][year:cn-y,年,month:cn,day:cn,日,·,weekday]·四月廿三 }
图像操纵着我,不,那是一种现实,一种陌生的、不可捉摸的、怪物般的现实,我被它彻底浸透,违背自己的意志:因为现在它是强者,我是镜子。我盯着面前这个巨大而可怕的陌生者,和它独处令我不寒而栗。但就在我想到这的那一刻,[93]发生了最可怕的事:我丧失了所有知觉,我彻底失灵了。那一秒钟我有一种不可名状的、痛苦而徒劳的渴望,我渴望着我自己,可那里只有它。除了它什么都没有。\footnote{\bi{布里格手记} \regular{里尔克}}

\title{\date[d=31,m=5,y=2024][year:cn-y,年,month:cn,day:cn,日,·,weekday]·四月廿四 }
黄昏的时候出去送信,而它永不到达!\footnote{\bi{王家新的诗} \regular{王家新}}

\title{\date[d=1,m=6,y=2024][year:cn-y,年,month:cn,day:cn,日,·,weekday]·四月廿五 }
只有这些诗行,写于明天。\footnote{\bi{想象一朵未来的玫瑰:佩索阿诗选} \regular{费尔南多·佩索阿}}

\title{\date[d=2,m=6,y=2024][year:cn-y,年,month:cn,day:cn,日,·,weekday]·四月廿六 }
这些费尽心机的伪装老鼠的外衣,乌鸦的皮毛,划掉的诗节\footnote{\bi{荒原:艾略特文集·诗歌} \regular{T.S.艾略特}}

\title{\date[d=3,m=6,y=2024][year:cn-y,年,month:cn,day:cn,日,·,weekday]·四月廿七 }
然后给我你喜欢的任何牢房,我可以回顾人生。\footnote{\bi{宇宙重建了自身:佩索阿诗精选} \regular{费尔南多·佩索阿}}

\title{\date[d=4,m=6,y=2024][year:cn-y,年,month:cn,day:cn,日,·,weekday]·四月廿八 }
生活最重要的因素是病。世界是一座大医院。我看见人类从灵魂到肉体都变了形\footnote{\bi{沉石与火舌:特朗斯特罗姆诗全集} \regular{托马斯·特朗斯特罗姆}}

\title{\date[d=5,m=6,y=2024][year:cn-y,年,month:cn,day:cn,日,·,weekday]·四月廿九 ·芒种}
只有努力之后才需要天赋\footnote{\bi{布里格手记} \regular{里尔克}}

\title{\date[d=6,m=6,y=2024][year:cn-y,年,month:cn,day:cn,日,·,weekday]·五月初一 }
我要审讯所有人,谁曾在严酷的时代,在摇篮里与世界一同摇摆。\footnote{\bi{茨维塔耶娃诗选} \regular{茨维塔耶娃}}

\title{\date[d=7,m=6,y=2024][year:cn-y,年,month:cn,day:cn,日,·,weekday]·五月初二 }
上升的路和下降的路是同一条路。\footnote{\bi{荒原:艾略特文集·诗歌} \regular{T.S.艾略特}}

\title{\date[d=8,m=6,y=2024][year:cn-y,年,month:cn,day:cn,日,·,weekday]·五月初三 }
自我训练在这里按其本义应理解为原创性自我的转换训练。这种训练使诗人依据创造性的原则在语言和现实/文化之间建立起一种互动的、彼此刺激和生发的关系\footnote{\bi{芒克的诗} \regular{芒克}}

\title{\date[d=9,m=6,y=2024][year:cn-y,年,month:cn,day:cn,日,·,weekday]·五月初四 }
完全准确地以感受到的东西来描写感受——如果感受是清晰的,就清晰地写;如果是隐晦的,就隐晦地写;如果是混乱的,就混乱地写;理解语法是一种工具,而非法律。\footnote{\bi{不安之书} \regular{费尔南多·佩索阿 热罗尼莫·皮萨罗 }}

\title{\date[d=10,m=6,y=2024][year:cn-y,年,month:cn,day:cn,日,·,weekday]·五月初五 ·端午节}
不要向世界索求赐予\footnote{\bi{戈麦的诗} \regular{戈麦 西渡}}

\title{\date[d=11,m=6,y=2024][year:cn-y,年,month:cn,day:cn,日,·,weekday]·五月初六 }
与生活和解的唯一方式就是与自己作对。荒诞即为神圣。\footnote{\bi{不安之书} \regular{费尔南多·佩索阿 热罗尼莫·皮萨罗 }}

\title{\date[d=12,m=6,y=2024][year:cn-y,年,month:cn,day:cn,日,·,weekday]·五月初七 }
“我所有的习惯都来自孤独,而不是来自人群。”\footnote{\bi{不安之书} \regular{费尔南多·佩索阿 热罗尼莫·皮萨罗 }}

\title{\date[d=13,m=6,y=2024][year:cn-y,年,month:cn,day:cn,日,·,weekday]·五月初八 }
我不知道这些感情是不是一种极度沮丧的缓慢疯狂,是不是关于我们可能去过的另一个世界的残存记忆——交叉和混合的残存记忆,好像梦里看见的东西一样。\footnote{\bi{不安之书} \regular{费尔南多·佩索阿 热罗尼莫·皮萨罗 }}

\title{\date[d=14,m=6,y=2024][year:cn-y,年,month:cn,day:cn,日,·,weekday]·五月初九 }
这个老地球今天又酩酊大醉了\footnote{\bi{伽利略传} \regular{贝托尔特·布莱希特}}

\title{\date[d=15,m=6,y=2024][year:cn-y,年,month:cn,day:cn,日,·,weekday]·五月初十 }
他野心勃勃又没有安全感,他才智超群又幼稚可笑,他果断坚决又惶恐不安,他坚忍淡泊又充满困惑\footnote{\bi{奥本海默传(诺兰导演电影《奥本海默》 灵感来源、普利策奖获奖传记)} \regular{凯·伯德 马丁·J.舍温}}

\title{\date[d=16,m=6,y=2024][year:cn-y,年,month:cn,day:cn,日,·,weekday]·五月十一 }
整个世界是一本打开的大书用未知的语言向我微笑\footnote{\bi{宇宙重建了自身:佩索阿诗精选} \regular{费尔南多·佩索阿}}

\title{\date[d=17,m=6,y=2024][year:cn-y,年,month:cn,day:cn,日,·,weekday]·五月十二 }
我想,我就是万物,死过了,但还活着。\footnote{\bi{昌耀的诗} \regular{昌耀}}

\title{\date[d=18,m=6,y=2024][year:cn-y,年,month:cn,day:cn,日,·,weekday]·五月十三 }
正是这一波浪,我爱的正是你身上这种莫名的东西,这种东西从内部蚀穿了我的命运,从外部使我的命运变得既黑暗又凄凉了,它既碍手又碍脚。\footnote{\bi{抒情诗的呼吸:一九二六年书信(帕斯捷尔纳克作品系列)} \regular{鲍·列·帕斯捷尔纳克 玛·伊·茨维塔耶娃 莱·马·里尔克}}

\title{\date[d=19,m=6,y=2024][year:cn-y,年,month:cn,day:cn,日,·,weekday]·五月十四 }
镜子把自己流露出去的美再吸回到自己的镜面。毕竟我们在感觉中蒸发四散;啊,我们在呼吸中把自己吐出,远逝;从薪柴的火焰到火焰我们的气味越发薄弱\footnote{\bi{杜英诺悲歌:里尔克诗选(文学馆系列)} \regular{里尔克}}

\title{\date[d=20,m=6,y=2024][year:cn-y,年,month:cn,day:cn,日,·,weekday]·五月十五 }
黄昏,是天空中唯一的发光体\footnote{\bi{戈麦的诗} \regular{戈麦 西渡}}

\title{\date[d=21,m=6,y=2024][year:cn-y,年,month:cn,day:cn,日,·,weekday]·五月十六 ·夏至}
我让我身体里所有的泪水连同这一夜倾泻一尽。——我的衰竭由此永远滞留不去。\footnote{\bi{彩画集:兰波散文诗全集(译文经典)} \regular{阿蒂尔·兰波}}

\title{\date[d=22,m=6,y=2024][year:cn-y,年,month:cn,day:cn,日,·,weekday]·五月十七 }
我们在自己解释的世界里\footnote{\bi{杜英诺悲歌:里尔克诗选(文学馆系列)} \regular{里尔克}}

\title{\date[d=23,m=6,y=2024][year:cn-y,年,month:cn,day:cn,日,·,weekday]·五月十八 }
我有多少不连贯,我就会有多少天分。\footnote{\bi{张枣的诗} \regular{张枣 颜炼军}}

\title{\date[d=24,m=6,y=2024][year:cn-y,年,month:cn,day:cn,日,·,weekday]·五月十九 }
真正的现代诗是诗之外的生活,它是火车本身,而不是歌唱它的诗句,它是铁轨的铁,灼热的铁轨,车轮的铁,它们真实的旋转,而不是我那些谈论铁轨和车轮的诗,诗中并无铁轨与车轮。\footnote{\bi{宇宙重建了自身:佩索阿诗精选} \regular{费尔南多·佩索阿}}

\title{\date[d=25,m=6,y=2024][year:cn-y,年,month:cn,day:cn,日,·,weekday]·五月二十 }
落入那个世界的时光,来自另一个更为骄傲的世界,因为有更多愁苦被消解……\footnote{\bi{不安之书} \regular{费尔南多·佩索阿 热罗尼莫·皮萨罗 }}

\title{\date[d=26,m=6,y=2024][year:cn-y,年,month:cn,day:cn,日,·,weekday]·五月廿一 }
只取长风我直入海洋[插图]我和太平洋衷曲相诉\footnote{\bi{骆一禾的诗} \regular{骆一禾 西渡}}

\title{\date[d=27,m=6,y=2024][year:cn-y,年,month:cn,day:cn,日,·,weekday]·五月廿二 }
爱情是一种受难!\footnote{\bi{秋日:冯至译诗选} \regular{歌德 海涅 尼采 荷尔德林 布莱希特 里尔克 格奥尔格}}

\title{\date[d=28,m=6,y=2024][year:cn-y,年,month:cn,day:cn,日,·,weekday]·五月廿三 }
难中有伴,烦恼减半。\footnote{\bi{堂吉诃德(译文名著精选)} \regular{塞万提斯}}

\title{\date[d=29,m=6,y=2024][year:cn-y,年,month:cn,day:cn,日,·,weekday]·五月廿四 }
你常常让那么多空间渗入这种值得描述的寻常日子,空间和间隙空间、天穹空间、宇宙空间和最敞开的仰望的一切空间,以至于你的形象在它们中间浓缩为最微小的形象:\footnote{\bi{谁此时孤独:里尔克晚期书信选} \regular{里尔克}}

\title{\date[d=30,m=6,y=2024][year:cn-y,年,month:cn,day:cn,日,·,weekday]·五月廿五 }
想用她绰绰有余的东西贿赂她是没有意义的\footnote{\bi{毛毛:时间窃贼和一个小女孩的不可思议的故事} \regular{米切尔·恩德}}

\title{\date[d=1,m=7,y=2024][year:cn-y,年,month:cn,day:cn,日,·,weekday]·五月廿六 }
时间停下脚步让你路过,我却将你记错了,当我试图把你放进生活抑或相似的表象之中。\footnote{\bi{不安之书} \regular{费尔南多·佩索阿 热罗尼莫·皮萨罗 }}

\title{\date[d=2,m=7,y=2024][year:cn-y,年,month:cn,day:cn,日,·,weekday]·五月廿七 }
大爱人\footnote{\bi{骆一禾的诗} \regular{骆一禾 西渡}}

\title{\date[d=3,m=7,y=2024][year:cn-y,年,month:cn,day:cn,日,·,weekday]·五月廿八 }
感受我的诗如何生长。它在生长,它占据我位置。它把我推到一边。它把我扔出巢穴。诗已完成。\footnote{\bi{沉石与火舌:特朗斯特罗姆诗全集} \regular{托马斯·特朗斯特罗姆}}

\title{\date[d=4,m=7,y=2024][year:cn-y,年,month:cn,day:cn,日,·,weekday]·五月廿九 }
我痛恨挪动线条人造的栩栩如生,我永远也不会满面泪痕,永远也不会满面笑容。\footnote{\bi{恶之花} \regular{波德莱尔}}

\title{\date[d=5,m=7,y=2024][year:cn-y,年,month:cn,day:cn,日,·,weekday]·五月三十 }
凡是心感觉不到的时间, 就是已经失去了\footnote{\bi{毛毛:时间窃贼和一个小女孩的不可思议的故事} \regular{米切尔·恩德}}

\title{\date[d=6,m=7,y=2024][year:cn-y,年,month:cn,day:cn,日,·,weekday]·六月初一 ·小暑}
充满醉意的末班车是满满的一天。\footnote{\bi{多多的诗} \regular{多多}}

\title{\date[d=7,m=7,y=2024][year:cn-y,年,month:cn,day:cn,日,·,weekday]·六月初二 }
我将把所有的日子都给你带去\footnote{\bi{芒克的诗} \regular{芒克}}

\title{\date[d=8,m=7,y=2024][year:cn-y,年,month:cn,day:cn,日,·,weekday]·六月初三 }
长久以来,我做过很多梦。我为此感到疲惫,但我不厌倦做梦。没有人会对做梦感到厌倦,因为做梦就是遗忘,而遗忘没有重量,是一种无梦的沉睡,我们在其中保持清醒。在梦里我实现了一切\footnote{\bi{不安之书} \regular{费尔南多·佩索阿 热罗尼莫·皮萨罗 }}

\title{\date[d=9,m=7,y=2024][year:cn-y,年,month:cn,day:cn,日,·,weekday]·六月初四 }
回忆太多,就得忘记[插图],一定要有很大的耐心,等待它们再回来。因为回忆本身还不是它。只有当回忆成为我们的血,成为眼神和表情,只有当它们无以名状、再无法与我们分开,[21]唯有如此,一首诗的第一个字才会在某个特殊的时刻,在回忆的中心出现,从那走出来。\footnote{\bi{布里格手记} \regular{里尔克}}

\title{\date[d=10,m=7,y=2024][year:cn-y,年,month:cn,day:cn,日,·,weekday]·六月初五 }
我这显然微乎其微的声音是否也体现了千万个声音的本质、千万个生命倾诉自我的饥渴以及无数灵魂的忍耐,它们与我的灵魂一样屈从于日常命运、徒劳之梦和无影无踪的希望。\footnote{\bi{不安之书} \regular{费尔南多·佩索阿 热罗尼莫·皮萨罗 }}

\title{\date[d=11,m=7,y=2024][year:cn-y,年,month:cn,day:cn,日,·,weekday]·六月初六 }
但它不也随即是一种从外部观察我们、一种在旁边关注我们的命运,有了它我们大概不再孤单?\footnote{\bi{谁此时孤独:里尔克晚期书信选} \regular{里尔克}}

\title{\date[d=12,m=7,y=2024][year:cn-y,年,month:cn,day:cn,日,·,weekday]·六月初七 }
人应该从当下的事情中汲取力量,战胜敌人。\footnote{\bi{朝圣} \regular{保罗·柯艾略}}

\title{\date[d=13,m=7,y=2024][year:cn-y,年,month:cn,day:cn,日,·,weekday]·六月初八 }
万物组合成一种极为奇怪的面貌,以突然而令人眩晕的暴力,击打着我的灵魂。\footnote{\bi{不安之书} \regular{费尔南多·佩索阿 热罗尼莫·皮萨罗 }}

\title{\date[d=14,m=7,y=2024][year:cn-y,年,month:cn,day:cn,日,·,weekday]·六月初九 }
我只是憧憬着有一台从天而降的大型压榨机,把灾难、大崩溃、惨绝人寰的悲剧、人类和物质、丑物和美物,不加区分统统碾碎\footnote{\bi{金阁寺} \regular{三岛由纪夫}}

\title{\date[d=15,m=7,y=2024][year:cn-y,年,month:cn,day:cn,日,·,weekday]·六月初十 }
大爱人\footnote{\bi{骆一禾的诗} \regular{骆一禾 西渡}}

\title{\date[d=16,m=7,y=2024][year:cn-y,年,month:cn,day:cn,日,·,weekday]·六月十一 }
世界就是凸显的、棱角分明的万物;但是,如果我们近视的话,他就是一团贫乏而连续的雾。\footnote{\bi{不安之书} \regular{费尔南多·佩索阿 热罗尼莫·皮萨罗 }}

\title{\date[d=17,m=7,y=2024][year:cn-y,年,month:cn,day:cn,日,·,weekday]·六月十二 }
当我明白失去它的时候,却同时感到,再不会有什么其他东西能证明自己了。\footnote{\bi{布里格手记} \regular{里尔克}}

\title{\date[d=18,m=7,y=2024][year:cn-y,年,month:cn,day:cn,日,·,weekday]·六月十三 }
人能够自欺,好像并不寂寞,只不过如此而已。\footnote{\bi{给青年诗人的信} \regular{莱内·马利亚·里尔克}}

\title{\date[d=19,m=7,y=2024][year:cn-y,年,month:cn,day:cn,日,·,weekday]·六月十四 }
我曾对一切虚掷我自己,\footnote{\bi{最好的里尔克} \regular{赖纳·马利亚·里尔克}}

\title{\date[d=20,m=7,y=2024][year:cn-y,年,month:cn,day:cn,日,·,weekday]·六月十五 }
诗人[插图]通过长期、广泛和经过推理思考过程,打乱所有的感觉意识,使自己成为通灵者。包括一切形式的爱、痛苦、疯狂;他亲自去寻找自身,他在他自身排尽一切毒素,以求保留精髓。在不可言喻的痛苦的折磨下,他要保持全部信念,全部超越于人的力量,他要成为一切人之中伟大的病人,伟大的罪人,伟大的被诅咒的人,——无比崇高的博学的科学家!——因为他要深入到不可知!他培育他的心灵,使之丰满富足,比任何人都要丰满富足!他进入不可知境界,这时,他在迷狂状态下,失去对他所见景象的理解力,真正有所见,真正看到他的幻象!就让他在这些闻所未闻、无可言状的事物中翻腾跳踉以至死去:另一类可怕的工人将要到来;他们将从这个人沉陷消亡的地平线上开始起步\footnote{\bi{彩画集:兰波散文诗全集(译文经典)} \regular{阿蒂尔·兰波}}

\title{\date[d=21,m=7,y=2024][year:cn-y,年,month:cn,day:cn,日,·,weekday]·六月十六 }
我从内心深处望着你,假想的新娘,我们的关系在你存在前就已破裂。\footnote{\bi{不安之书} \regular{费尔南多·佩索阿 热罗尼莫·皮萨罗 }}

\title{\date[d=22,m=7,y=2024][year:cn-y,年,month:cn,day:cn,日,·,weekday]·六月十七 ·大暑}
曾把堂吉诃德的这类藏书都给烧了,\footnote{\bi{堂吉诃德(译文名著精选)} \regular{塞万提斯}}

\title{\date[d=23,m=7,y=2024][year:cn-y,年,month:cn,day:cn,日,·,weekday]·六月十八 }
这就是生活:充满极其不寻常的东西,它们只对一个人有意义,且不可言说。\footnote{\bi{布里格手记} \regular{里尔克}}

\title{\date[d=24,m=7,y=2024][year:cn-y,年,month:cn,day:cn,日,·,weekday]·六月十九 }
如果我意识到自己在做梦,立刻就会有词语在我心中产生。我的所有情感都是一个图像,所有梦都是配乐的图画。\footnote{\bi{不安之书} \regular{费尔南多·佩索阿 热罗尼莫·皮萨罗 }}

\title{\date[d=25,m=7,y=2024][year:cn-y,年,month:cn,day:cn,日,·,weekday]·六月二十 }
以安宁的生活代替孤独的漂泊的意味深长的日子,\footnote{\bi{黄灿然的诗} \regular{黄灿然}}

\title{\date[d=26,m=7,y=2024][year:cn-y,年,month:cn,day:cn,日,·,weekday]·六月廿一 }
我和过去隔着黑色的土地我和未来隔着无声的空气\footnote{\bi{海子的诗} \regular{海子}}

\title{\date[d=27,m=7,y=2024][year:cn-y,年,month:cn,day:cn,日,·,weekday]·六月廿二 }
在夜的黑暗中碾着虚无\footnote{\bi{沉石与火舌:特朗斯特罗姆诗全集} \regular{托马斯·特朗斯特罗姆}}

\title{\date[d=28,m=7,y=2024][year:cn-y,年,month:cn,day:cn,日,·,weekday]·六月廿三 }
只要想起一生中后悔的事梅花便落了下来\footnote{\bi{张枣的诗} \regular{张枣 颜炼军}}

\title{\date[d=29,m=7,y=2024][year:cn-y,年,month:cn,day:cn,日,·,weekday]·六月廿四 }
成一种它是不可思议的感觉,在我写下这些文字的时候,这种感觉实在难以形诸笔墨。\footnote{\bi{抒情诗的呼吸:一九二六年书信(帕斯捷尔纳克作品系列)} \regular{鲍·列·帕斯捷尔纳克 玛·伊·茨维塔耶娃 莱·马·里尔克}}

\title{\date[d=30,m=7,y=2024][year:cn-y,年,month:cn,day:cn,日,·,weekday]·六月廿五 }
只要摈弃简单的表述,人们就能够将某种精神之化体“发表出来”\footnote{\bi{谁此时孤独:里尔克晚期书信选} \regular{里尔克}}

\title{\date[d=31,m=7,y=2024][year:cn-y,年,month:cn,day:cn,日,·,weekday]·六月廿六 }
给我狭窄的心/一个大的宇宙\footnote{\bi{给青年诗人的信} \regular{莱内·马利亚·里尔克}}

\title{\date[d=1,m=8,y=2024][year:cn-y,年,month:cn,day:cn,日,·,weekday]·六月廿七 }
擦着广播中的锈用砖灰\footnote{\bi{顾城的诗} \regular{顾城}}

\title{\date[d=2,m=8,y=2024][year:cn-y,年,month:cn,day:cn,日,·,weekday]·六月廿八 }
你将眼看着身体里长出一个老人,\footnote{\bi{欧阳江河的诗} \regular{欧阳江河}}

\title{\date[d=3,m=8,y=2024][year:cn-y,年,month:cn,day:cn,日,·,weekday]·六月廿九 }
我盯着卧室的墙,好像那是整个宇宙\footnote{\bi{想象一朵未来的玫瑰:佩索阿诗选} \regular{费尔南多·佩索阿}}

\title{\date[d=4,m=8,y=2024][year:cn-y,年,month:cn,day:cn,日,·,weekday]·七月初一 }
她却一动不动任由月光清洗,仿佛只要稍微转转眼睛、动动嘴巴,被她拒绝的世界就会瞬间如雪崩般涌来。\footnote{\bi{金阁寺} \regular{三岛由纪夫}}

\title{\date[d=5,m=8,y=2024][year:cn-y,年,month:cn,day:cn,日,·,weekday]·七月初二 }
大海广阔如庄子\footnote{\bi{西川的诗} \regular{西川}}

\title{\date[d=6,m=8,y=2024][year:cn-y,年,month:cn,day:cn,日,·,weekday]·七月初三 }
以至于诗里的时间最后成了真,就好像用地球的尺寸制造了一个地球仪\footnote{\bi{布里格手记} \regular{里尔克}}

\title{\date[d=7,m=8,y=2024][year:cn-y,年,month:cn,day:cn,日,·,weekday]·七月初四 ·立秋}
我带着各种感觉睡觉,\footnote{\bi{宇宙重建了自身:佩索阿诗精选} \regular{费尔南多·佩索阿}}

\title{\date[d=8,m=8,y=2024][year:cn-y,年,month:cn,day:cn,日,·,weekday]·七月初五 }
我被影子拎着像提琴被放入黑盒\footnote{\bi{沉石与火舌:特朗斯特罗姆诗全集} \regular{托马斯·特朗斯特罗姆}}

\title{\date[d=9,m=8,y=2024][year:cn-y,年,month:cn,day:cn,日,·,weekday]·七月初六 }
这个拂晓是世界上的第一个\footnote{\bi{不安之书} \regular{费尔南多·佩索阿 热罗尼莫·皮萨罗 }}

\title{\date[d=10,m=8,y=2024][year:cn-y,年,month:cn,day:cn,日,·,weekday]·七月初七 ·七夕}
从未经历过的生活浮出来,[57]混入曾经的真实,挤走人们自以为了解的过去\footnote{\bi{布里格手记} \regular{里尔克}}

\title{\date[d=11,m=8,y=2024][year:cn-y,年,month:cn,day:cn,日,·,weekday]·七月初八 }
我用词语幻觉解释我各种像中了魔法那样的诡论!\footnote{\bi{彩画集:兰波散文诗全集(译文经典)} \regular{阿蒂尔·兰波}}

\title{\date[d=12,m=8,y=2024][year:cn-y,年,month:cn,day:cn,日,·,weekday]·七月初九 }
人们看到的我,和我自己认为的我,哪个更长久呢?\footnote{\bi{金阁寺} \regular{三岛由纪夫}}

\title{\date[d=13,m=8,y=2024][year:cn-y,年,month:cn,day:cn,日,·,weekday]·七月初十 }
她们沉睡时长出很多很多脸庞,\footnote{\bi{雪是谁说的谎:倪湛舸诗集} \regular{倪湛舸}}

\title{\date[d=14,m=8,y=2024][year:cn-y,年,month:cn,day:cn,日,·,weekday]·七月十一 }
所有雨滴都闪耀一下\footnote{\bi{顾城的诗} \regular{顾城}}

\title{\date[d=15,m=8,y=2024][year:cn-y,年,month:cn,day:cn,日,·,weekday]·七月十二 }
事物的充盈贯穿我空气般透澈的心,仅仅观望就足够温情。我从来都只是无形的幻梦,褪去全部的灵魂,空留一缕倏来忽往的漂泊的微风。\footnote{\bi{不安之书} \regular{费尔南多·佩索阿 热罗尼莫·皮萨罗 }}

\title{\date[d=16,m=8,y=2024][year:cn-y,年,month:cn,day:cn,日,·,weekday]·七月十三 }
而未来的时间却包含在过去里\footnote{\bi{荒原:艾略特文集·诗歌} \regular{T.S.艾略特}}

\title{\date[d=17,m=8,y=2024][year:cn-y,年,month:cn,day:cn,日,·,weekday]·七月十四 }
我投入此火 这三者是囚禁我的灯盏 吐出光辉 \footnote{\bi{海子的诗} \regular{海子}}

\title{\date[d=18,m=8,y=2024][year:cn-y,年,month:cn,day:cn,日,·,weekday]·七月十五 ·中元节}
长久以来,我度过的时光间歇,那些用隔绝的甜蜜所雕刻的面对自然的时光,会像勋章一样留存于心。我在这些时刻里忘却了一切生活意图与希冀的方向,享受着虚无与精神的全然宁静,落入以渴望编织成的蓝色怀抱。或许我从未经历过某个不可磨灭的时刻,能够幸免于溃败和沮丧的精神深渊。我全部的自由时间里,有一种疼痛躲在意识之墙后面,在别处的庭院里沉睡着,隐隐绽放;可是忧伤之花的香气与独有色彩凭借本能漫过高墙;墙的另一面,玫瑰盛开,“彼端”深居于我神秘混沌的存在,却也始终是某个“此端”,浮显于我对生活似睡非睡的倦怠。\footnote{\bi{不安之书} \regular{费尔南多·佩索阿 热罗尼莫·皮萨罗 }}

\title{\date[d=19,m=8,y=2024][year:cn-y,年,month:cn,day:cn,日,·,weekday]·七月十六 }
你的名字是洛阳,你的命运是黄昏\footnote{\bi{戈麦的诗} \regular{戈麦 西渡}}

\title{\date[d=20,m=8,y=2024][year:cn-y,年,month:cn,day:cn,日,·,weekday]·七月十七 }
您不是我最喜爱的诗人(“最喜爱的”是一个程度),您是大自然的一个现象,这一现象不可能是我的,也无法去爱它,而只能用全部身心去感觉它,或者(还没完呀!)您就是第五元素的化身:即诗的本身,或者(还没完)您就是诗从其中诞生出来的那种东西,是大于诗歌本身——即您大于自身的那种东西。\footnote{\bi{抒情诗的呼吸:一九二六年书信(帕斯捷尔纳克作品系列)} \regular{鲍·列·帕斯捷尔纳克 玛·伊·茨维塔耶娃 莱·马·里尔克}}

\title{\date[d=21,m=8,y=2024][year:cn-y,年,month:cn,day:cn,日,·,weekday]·七月十八 }
至少去创造一种新的悲观主义吧,一种新的否定,这样就能产生幻觉,以为我们身上的某些东西即使不好,也会留下!\footnote{\bi{不安之书} \regular{费尔南多·佩索阿 热罗尼莫·皮萨罗 }}

\title{\date[d=22,m=8,y=2024][year:cn-y,年,month:cn,day:cn,日,·,weekday]·七月十九 ·处暑}
我就是这最后一个夜晚最后一盏黑暗的灯是最后一个夜晚水面上爱情阴沉的旗帜在黑暗中鞭打着一颗干渴的心沿着先知的梯子上下爬行\footnote{\bi{戈麦的诗} \regular{戈麦 西渡}}

\title{\date[d=23,m=8,y=2024][year:cn-y,年,month:cn,day:cn,日,·,weekday]·七月二十 }
夜从城楼跳将下来踯躅原野。\footnote{\bi{昌耀的诗} \regular{昌耀}}

\title{\date[d=24,m=8,y=2024][year:cn-y,年,month:cn,day:cn,日,·,weekday]·七月廿一 }
一点雨声的幽凉滴到我憔悴的梦,也许会长成一树圆圆的绿阴来覆荫我自己。\footnote{\bi{何其芳散文} \regular{何其芳}}

\title{\date[d=25,m=8,y=2024][year:cn-y,年,month:cn,day:cn,日,·,weekday]·七月廿二 }
有些感受就是睡意,好像雾一样笼罩着精神的全部延展,不允许思考,也不允许行动,甚至不允许明确地存在。\footnote{\bi{不安之书} \regular{费尔南多·佩索阿 热罗尼莫·皮萨罗 }}

\title{\date[d=26,m=8,y=2024][year:cn-y,年,month:cn,day:cn,日,·,weekday]·七月廿三 }
墙是你自身的一部分—\footnote{\bi{沉石与火舌:特朗斯特罗姆诗全集} \regular{托马斯·特朗斯特罗姆}}

\title{\date[d=27,m=8,y=2024][year:cn-y,年,month:cn,day:cn,日,·,weekday]·七月廿四 }
现在我像从四十六楼窗口俯视十几楼窗外飘扬的花格子衬衫那样俯视我的青春。\footnote{\bi{黄灿然的诗} \regular{黄灿然}}

\title{\date[d=28,m=8,y=2024][year:cn-y,年,month:cn,day:cn,日,·,weekday]·七月廿五 }
生命中最黑暗的事件 “写”永远不会抵达 所谓写作就是逃跑的马拉松\footnote{\bi{于坚的诗} \regular{于坚}}

\title{\date[d=29,m=8,y=2024][year:cn-y,年,month:cn,day:cn,日,·,weekday]·七月廿六 }
那么现在,我对自己说的话,你能够听到了吗?\footnote{\bi{多多的诗} \regular{多多}}

\title{\date[d=30,m=8,y=2024][year:cn-y,年,month:cn,day:cn,日,·,weekday]·七月廿七 }
各人有各人的时间。时间只有在属于那个人时, 它才是活的。\footnote{\bi{毛毛:时间窃贼和一个小女孩的不可思议的故事} \regular{米切尔·恩德}}

\title{\date[d=31,m=8,y=2024][year:cn-y,年,month:cn,day:cn,日,·,weekday]·七月廿八 }
以石板为睡床,以熬夜当睡眠’\footnote{\bi{堂吉诃德(译文名著精选)} \regular{塞万提斯}}

\title{\date[d=1,m=9,y=2024][year:cn-y,年,month:cn,day:cn,日,·,weekday]·七月廿九 }
坏诗糟蹋艺术,好诗为诗所误\footnote{\bi{黄灿然的诗} \regular{黄灿然}}

\title{\date[d=2,m=9,y=2024][year:cn-y,年,month:cn,day:cn,日,·,weekday]·七月三十 }
酒回到粮食,秋天,空的杯盏。\footnote{\bi{欧阳江河的诗} \regular{欧阳江河}}

\title{\date[d=3,m=9,y=2024][year:cn-y,年,month:cn,day:cn,日,·,weekday]·八月初一 }
愁容骑士\footnote{\bi{堂吉诃德(译文名著精选)} \regular{塞万提斯}}

\title{\date[d=4,m=9,y=2024][year:cn-y,年,month:cn,day:cn,日,·,weekday]·八月初二 }
穿过世界,一片横跨裂缝的蛛网。\footnote{\bi{想象一朵未来的玫瑰:佩索阿诗选} \regular{费尔南多·佩索阿}}

\title{\date[d=5,m=9,y=2024][year:cn-y,年,month:cn,day:cn,日,·,weekday]·八月初三 }
我知道白天将会重压在我身上,好像我一无所知一样。我知道今天所做的一切,都会分享我有过的失眠,而不是我没有睡觉引起的疲劳。我知道我会经历一种更强烈的梦游症,更浅层,不仅因为我没有睡,更因为我不能睡。\footnote{\bi{不安之书} \regular{费尔南多·佩索阿 热罗尼莫·皮萨罗 }}

\title{\date[d=6,m=9,y=2024][year:cn-y,年,month:cn,day:cn,日,·,weekday]·八月初四 }
看见的却不是自然,而是它在我心中唤起的种种面目…\footnote{\bi{里尔克全集 第九卷 沃普斯韦德、奥古斯特·罗丹} \regular{莱纳.马利亚.里克尔 叶廷芳}}

\title{\date[d=7,m=9,y=2024][year:cn-y,年,month:cn,day:cn,日,·,weekday]·八月初五 ·白露}
无线电中传来的敲击,它的名字叫远方。\footnote{\bi{茨维塔耶娃诗选} \regular{茨维塔耶娃}}

\title{\date[d=8,m=9,y=2024][year:cn-y,年,month:cn,day:cn,日,·,weekday]·八月初六 }
呼吸月光呼吸鸟语呼吸花朵和它的暴力\footnote{\bi{于坚的诗} \regular{于坚}}

\title{\date[d=9,m=9,y=2024][year:cn-y,年,month:cn,day:cn,日,·,weekday]·八月初七 }
我已看惯许多的人生。我已看惯许多的人死。我已经饱经沧桑\footnote{\bi{昌耀的诗} \regular{昌耀}}

\title{\date[d=10,m=9,y=2024][year:cn-y,年,month:cn,day:cn,日,·,weekday]·八月初八 }
它越是在薄纱和透明中让我回想起死去的那些时刻,越是将眼下的时光变得遥远,这些时光环绕我的时候就越是雾气沉沉,当我作为它们的化身时,它们越是缥缈\footnote{\bi{不安之书} \regular{费尔南多·佩索阿 热罗尼莫·皮萨罗 }}

\title{\date[d=11,m=9,y=2024][year:cn-y,年,month:cn,day:cn,日,·,weekday]·八月初九 }
天空从远方照应着这些远近搭配的景物,以一种如此灵性的空气赋予它们魂魄,于是景物之间的特殊配合\footnote{\bi{谁此时孤独:里尔克晚期书信选} \regular{里尔克}}

\title{\date[d=12,m=9,y=2024][year:cn-y,年,month:cn,day:cn,日,·,weekday]·八月初十 }
走着我自己\footnote{\bi{骆一禾的诗} \regular{骆一禾 西渡}}

\title{\date[d=13,m=9,y=2024][year:cn-y,年,month:cn,day:cn,日,·,weekday]·八月十一 }
我写我自己,是为了从生活中散心\footnote{\bi{不安之书} \regular{费尔南多·佩索阿 热罗尼莫·皮萨罗 }}

\title{\date[d=14,m=9,y=2024][year:cn-y,年,month:cn,day:cn,日,·,weekday]·八月十二 }
所有滚动的轮子都在抵抗着死亡!\footnote{\bi{沉石与火舌:特朗斯特罗姆诗全集} \regular{托马斯·特朗斯特罗姆}}

\title{\date[d=15,m=9,y=2024][year:cn-y,年,month:cn,day:cn,日,·,weekday]·八月十三 }
两只眼睛,一只飞在天上,一只掉进洞里\footnote{\bi{戈麦的诗} \regular{戈麦 西渡}}

\title{\date[d=16,m=9,y=2024][year:cn-y,年,month:cn,day:cn,日,·,weekday]·八月十四 }
影子叠着影子使黑暗蠕动起来。\footnote{\bi{张枣的诗} \regular{张枣 颜炼军}}

\title{\date[d=17,m=9,y=2024][year:cn-y,年,month:cn,day:cn,日,·,weekday]·八月十五 ·中秋节}
爬行的阴影\footnote{\bi{顾城的诗} \regular{顾城}}

\title{\date[d=18,m=9,y=2024][year:cn-y,年,month:cn,day:cn,日,·,weekday]·八月十六 }
而是为了给内在世界以全然的外在性,实现不可能实现之事,联结对立面,通过使梦表象化,为其注入纯粹之梦无坚不摧的力量\footnote{\bi{不安之书} \regular{费尔南多·佩索阿 热罗尼莫·皮萨罗 }}

\title{\date[d=19,m=9,y=2024][year:cn-y,年,month:cn,day:cn,日,·,weekday]·八月十七 }
诗句生长,像星星像玫瑰,像家中不需要的美。\footnote{\bi{茨维塔耶娃诗选} \regular{茨维塔耶娃}}

\title{\date[d=20,m=9,y=2024][year:cn-y,年,month:cn,day:cn,日,·,weekday]·八月十八 }
在秋天怀念秋天 如今只有回忆能抵达这个季节\footnote{\bi{于坚的诗} \regular{于坚}}

\title{\date[d=21,m=9,y=2024][year:cn-y,年,month:cn,day:cn,日,·,weekday]·八月十九 }
街在看我\footnote{\bi{沉石与火舌:特朗斯特罗姆诗全集} \regular{托马斯·特朗斯特罗姆}}

\title{\date[d=22,m=9,y=2024][year:cn-y,年,month:cn,day:cn,日,·,weekday]·八月二十 ·秋分}
于是你的苦痛与辛酸终成眷属\footnote{\bi{不安之书} \regular{费尔南多·佩索阿 热罗尼莫·皮萨罗 }}

\title{\date[d=23,m=9,y=2024][year:cn-y,年,month:cn,day:cn,日,·,weekday]·八月廿一 }
空,落地,我俯身拾起无限多的空。每一片具体的碎片里,都有一个抽象\footnote{\bi{欧阳江河的诗} \regular{欧阳江河}}

\title{\date[d=24,m=9,y=2024][year:cn-y,年,month:cn,day:cn,日,·,weekday]·八月廿二 }
每棵树都是自己声音的囚徒。\footnote{\bi{沉石与火舌:特朗斯特罗姆诗全集} \regular{托马斯·特朗斯特罗姆}}

\title{\date[d=25,m=9,y=2024][year:cn-y,年,month:cn,day:cn,日,·,weekday]·八月廿三 }
是的,每次道别都是死亡。在那列我们称作生活的火车上我们都是彼此生活中的偶然,该当离去时,我们都会感到遗憾。 \footnote{\bi{想象一朵未来的玫瑰:佩索阿诗选} \regular{费尔南多·佩索阿}}

\title{\date[d=26,m=9,y=2024][year:cn-y,年,month:cn,day:cn,日,·,weekday]·八月廿四 }
组成世界的无数人与物就是一条看不见尽头的画廊,其内在枯燥乏味\footnote{\bi{不安之书} \regular{费尔南多·佩索阿 热罗尼莫·皮萨罗 }}

\title{\date[d=27,m=9,y=2024][year:cn-y,年,month:cn,day:cn,日,·,weekday]·八月廿五 }
我们想象天开,气流中雨落斗笠、长枪\footnote{\bi{戈麦的诗} \regular{戈麦 西渡}}

\title{\date[d=28,m=9,y=2024][year:cn-y,年,month:cn,day:cn,日,·,weekday]·八月廿六 }
酒借人们的歌喉颂扬自己的功勋,\footnote{\bi{恶之花} \regular{波德莱尔}}

\title{\date[d=29,m=9,y=2024][year:cn-y,年,month:cn,day:cn,日,·,weekday]·八月廿七 }
你要默认自己的诗句:行行重行行\footnote{\bi{骆一禾的诗} \regular{骆一禾 西渡}}

\title{\date[d=30,m=9,y=2024][year:cn-y,年,month:cn,day:cn,日,·,weekday]·八月廿八 }
你在你名字里失踪\footnote{\bi{张枣的诗} \regular{张枣 颜炼军}}

\title{\date[d=1,m=10,y=2024][year:cn-y,年,month:cn,day:cn,日,·,weekday]·八月廿九 }
好像书页在她眼下越来越满,好像她一边读一边添进去某些对她必不可少、却不在书中的词。\footnote{\bi{布里格手记} \regular{里尔克}}

\title{\date[d=2,m=10,y=2024][year:cn-y,年,month:cn,day:cn,日,·,weekday]·八月三十 }
在一间未点灯的房间,夜便孤立起来\footnote{\bi{张枣的诗} \regular{张枣 颜炼军}}

\title{\date[d=3,m=10,y=2024][year:cn-y,年,month:cn,day:cn,日,·,weekday]·九月初一 }
在万物夜间塌陷的凹洞中制造出另一种声音。\footnote{\bi{不安之书} \regular{费尔南多·佩索阿 热罗尼莫·皮萨罗 }}

\title{\date[d=4,m=10,y=2024][year:cn-y,年,month:cn,day:cn,日,·,weekday]·九月初二 }
在生命之前,人是一切和永恒,一旦生活起来,他就变成了某人和此刻。(存在和拥有——并无区别!)\footnote{\bi{抒情诗的呼吸:一九二六年书信(帕斯捷尔纳克作品系列)} \regular{鲍·列·帕斯捷尔纳克 玛·伊·茨维塔耶娃 莱·马·里尔克}}

\title{\date[d=5,m=10,y=2024][year:cn-y,年,month:cn,day:cn,日,·,weekday]·九月初三 }
就像火走在柴中\footnote{\bi{海子的诗} \regular{海子}}

\title{\date[d=6,m=10,y=2024][year:cn-y,年,month:cn,day:cn,日,·,weekday]·九月初四 }
梦是一种惩罚。我从梦境中获得如此清明的理智,以至于我把所有梦见的事物都视为真实。因此,一切曾经入梦的东西也都失去了价值。\footnote{\bi{不安之书} \regular{费尔南多·佩索阿 热罗尼莫·皮萨罗 }}

\title{\date[d=7,m=10,y=2024][year:cn-y,年,month:cn,day:cn,日,·,weekday]·九月初五 }
是另一种东西,只属于我,一点点纠缠在孤立的感觉上,掺着夜晚和寂静,选择那盏灯作为支点,因为它是唯一的支点。好像因为它亮着,夜晚才这么暗。好像因为我醒着,在黑暗中做梦,它才亮着。\footnote{\bi{不安之书} \regular{费尔南多·佩索阿 热罗尼莫·皮萨罗 }}

\title{\date[d=8,m=10,y=2024][year:cn-y,年,month:cn,day:cn,日,·,weekday]·九月初六 ·寒露}
还有夜,我们所期望的,当风满含空宇啮咬我们的脸——她将为任何人滞留。温柔而略带失望的莅临与孤寂的心会合。\footnote{\bi{最好的里尔克} \regular{赖纳·马利亚·里尔克}}

\title{\date[d=9,m=10,y=2024][year:cn-y,年,month:cn,day:cn,日,·,weekday]·九月初七 }
我感到这阵风里还抬着更多的风\footnote{\bi{张枣的诗} \regular{张枣 颜炼军}}

\title{\date[d=10,m=10,y=2024][year:cn-y,年,month:cn,day:cn,日,·,weekday]·九月初八 }
它包含了呼唤和回答;\footnote{\bi{布里格手记} \regular{里尔克}}

\title{\date[d=11,m=10,y=2024][year:cn-y,年,month:cn,day:cn,日,·,weekday]·九月初九 ·重阳节}
我已见过所有我之前还未见过的东西。我已见过所有我还未见过的东西。\footnote{\bi{不安之书} \regular{费尔南多·佩索阿 热罗尼莫·皮萨罗 }}

\title{\date[d=12,m=10,y=2024][year:cn-y,年,month:cn,day:cn,日,·,weekday]·九月初十 }
生活是虚无的螺旋,无限向往着它不能拥有的事物。\footnote{\bi{不安之书} \regular{费尔南多·佩索阿 热罗尼莫·皮萨罗 }}

\title{\date[d=13,m=10,y=2024][year:cn-y,年,month:cn,day:cn,日,·,weekday]·九月十一 }
现在(是的,我应如何描述?),现在一片静寂。静得仿佛疼痛停止了。一种感觉特殊的、发痒的静寂,仿佛伤口在愈合。\footnote{\bi{布里格手记} \regular{里尔克}}

\title{\date[d=14,m=10,y=2024][year:cn-y,年,month:cn,day:cn,日,·,weekday]·九月十二 }
在梦里过虚幻的人生也依然是在生活\footnote{\bi{不安之书} \regular{费尔南多·佩索阿 热罗尼莫·皮萨罗 }}

\title{\date[d=15,m=10,y=2024][year:cn-y,年,month:cn,day:cn,日,·,weekday]·九月十三 }
没有词,但也许有风格…\footnote{\bi{沉石与火舌:特朗斯特罗姆诗全集} \regular{托马斯·特朗斯特罗姆}}

\title{\date[d=16,m=10,y=2024][year:cn-y,年,month:cn,day:cn,日,·,weekday]·九月十四 }
像一笔坚硬的债,我要用全部生命偿还\footnote{\bi{戈麦的诗} \regular{戈麦 西渡}}

\title{\date[d=17,m=10,y=2024][year:cn-y,年,month:cn,day:cn,日,·,weekday]·九月十五 }
在一切不可能的事情中,“不做我”是最难的,\footnote{\bi{不安之书} \regular{费尔南多·佩索阿 热罗尼莫·皮萨罗 }}

\title{\date[d=18,m=10,y=2024][year:cn-y,年,month:cn,day:cn,日,·,weekday]·九月十六 }
体验新感觉的唯一方法就是建立一个新的灵魂\footnote{\bi{不安之书} \regular{费尔南多·佩索阿 热罗尼莫·皮萨罗 }}

\title{\date[d=19,m=10,y=2024][year:cn-y,年,month:cn,day:cn,日,·,weekday]·九月十七 }
木材自认是提琴,那有什么办法\footnote{\bi{彩画集:兰波散文诗全集(译文经典)} \regular{阿蒂尔·兰波}}

\title{\date[d=20,m=10,y=2024][year:cn-y,年,month:cn,day:cn,日,·,weekday]·九月十八 }
无论如何,最好不要出生,因为无论它在每个时刻多么有趣,\footnote{\bi{宇宙重建了自身:佩索阿诗精选} \regular{费尔南多·佩索阿}}

\title{\date[d=21,m=10,y=2024][year:cn-y,年,month:cn,day:cn,日,·,weekday]·九月十九 }
我们喜悦那永续不断的事物,并且站在世界与玩具之间的中间地带,\footnote{\bi{杜英诺悲歌:里尔克诗选(文学馆系列)} \regular{里尔克}}

\title{\date[d=22,m=10,y=2024][year:cn-y,年,month:cn,day:cn,日,·,weekday]·九月二十 }
浆果一样的梦\footnote{\bi{顾城的诗} \regular{顾城}}

\title{\date[d=23,m=10,y=2024][year:cn-y,年,month:cn,day:cn,日,·,weekday]·九月廿一 ·霜降}
全知全能的上帝和极乐永生的彼世是自欺欺人的幻象,受制于空间的刚性和时间的不可逆,个体的人永远无法逃离偶然,其所见所思无非是随机的碎片\footnote{\bi{布里格手记} \regular{里尔克}}

\title{\date[d=24,m=10,y=2024][year:cn-y,年,month:cn,day:cn,日,·,weekday]·九月廿二 }
你去吧,去眼睛张望的地方!所有国度的眼,整个大地的眼,还有你蓝色的眼睛,我在你眼里看见自己,你的眼把罗斯张望。\footnote{\bi{茨维塔耶娃诗选} \regular{茨维塔耶娃}}

\title{\date[d=25,m=10,y=2024][year:cn-y,年,month:cn,day:cn,日,·,weekday]·九月廿三 }
夏日衰老着,一切都汇成伤感的喧嚣\footnote{\bi{沉石与火舌:特朗斯特罗姆诗全集} \regular{托马斯·特朗斯特罗姆}}

\title{\date[d=26,m=10,y=2024][year:cn-y,年,month:cn,day:cn,日,·,weekday]·九月廿四 }
不是向外而是向更深处退避,不要强行抵抗现实的压力,而是利用压力,以便借此潜入自己本性的更密实、更深厚、更独特的层次。\footnote{\bi{谁此时孤独:里尔克晚期书信选} \regular{里尔克}}

\title{\date[d=27,m=10,y=2024][year:cn-y,年,month:cn,day:cn,日,·,weekday]·九月廿五 }
我领悟了大地是活的,有一个无边的巨大的爬行和飞翔的世界,可以完全不理睬我们而过着自己丰富的生活。\footnote{\bi{沉石与火舌:特朗斯特罗姆诗全集} \regular{托马斯·特朗斯特罗姆}}

\title{\date[d=28,m=10,y=2024][year:cn-y,年,month:cn,day:cn,日,·,weekday]·九月廿六 }
现在你额头无光,你的灵魂已经出窍,\footnote{\bi{黄灿然的诗} \regular{黄灿然}}

\title{\date[d=29,m=10,y=2024][year:cn-y,年,month:cn,day:cn,日,·,weekday]·九月廿七 }
你只是一个瞬息,你被无数瞬息牵引\footnote{\bi{张枣的诗} \regular{张枣 颜炼军}}

\title{\date[d=30,m=10,y=2024][year:cn-y,年,month:cn,day:cn,日,·,weekday]·九月廿八 }
唯有无声息的感官为他输入世界,[68]万籁俱寂,一个紧张等待的世界,它尚未完成,在音调被创造之前。\footnote{\bi{布里格手记} \regular{里尔克}}

\title{\date[d=31,m=10,y=2024][year:cn-y,年,month:cn,day:cn,日,·,weekday]·九月廿九 }
爱人,爱人,我们像神:整个世界都为了我们!我们在人间到处是家,一切都属于我们。\footnote{\bi{茨维塔耶娃诗选} \regular{茨维塔耶娃}}

\title{\date[d=1,m=11,y=2024][year:cn-y,年,month:cn,day:cn,日,·,weekday]·十月初一 }
我想成为高温的生物,滚烫的水,波浪的节奏和……\footnote{\bi{宇宙重建了自身:佩索阿诗精选} \regular{费尔南多·佩索阿}}

\title{\date[d=2,m=11,y=2024][year:cn-y,年,month:cn,day:cn,日,·,weekday]·十月初二 }
生与死不是对立的两种状态,而是互相包含、不断转化的\footnote{\bi{三岛由纪夫,或空的幻景} \regular{玛格丽特·尤瑟纳尔}}

\title{\date[d=3,m=11,y=2024][year:cn-y,年,month:cn,day:cn,日,·,weekday]·十月初三 }
它像一首深歌,把我变成水中物,让我学习鱼的品格\footnote{\bi{黄灿然的诗} \regular{黄灿然}}

\title{\date[d=4,m=11,y=2024][year:cn-y,年,month:cn,day:cn,日,·,weekday]·十月初四 }
等待就等于倒行\footnote{\bi{顾城的诗} \regular{顾城}}

\title{\date[d=5,m=11,y=2024][year:cn-y,年,month:cn,day:cn,日,·,weekday]·十月初五 }
还给我最初的记忆吧!\footnote{\bi{顾城的诗} \regular{顾城}}

\title{\date[d=6,m=11,y=2024][year:cn-y,年,month:cn,day:cn,日,·,weekday]·十月初六 }
然则人一接近死亡,就不再凝望那死亡\footnote{\bi{杜英诺悲歌:里尔克诗选(文学馆系列)} \regular{里尔克}}

\title{\date[d=7,m=11,y=2024][year:cn-y,年,month:cn,day:cn,日,·,weekday]·十月初七 ·立冬}
此处在此岸与彼岸之间,\footnote{\bi{荒原:艾略特文集·诗歌} \regular{T.S.艾略特}}

\title{\date[d=8,m=11,y=2024][year:cn-y,年,month:cn,day:cn,日,·,weekday]·十月初八 }
醒,是梦中往外跳伞。摆脱令人窒息的旋涡\footnote{\bi{沉石与火舌:特朗斯特罗姆诗全集} \regular{托马斯·特朗斯特罗姆}}

\title{\date[d=9,m=11,y=2024][year:cn-y,年,month:cn,day:cn,日,·,weekday]·十月初九 }
一种向外部呼吸的空无;一种轻盈的死亡,人们带着怀恋与清新从中醒来;一种顺服,把灵魂的织布交给遗忘抚摩。\footnote{\bi{不安之书} \regular{费尔南多·佩索阿 热罗尼莫·皮萨罗 }}

\title{\date[d=10,m=11,y=2024][year:cn-y,年,month:cn,day:cn,日,·,weekday]·十月初十 }
所谓痛苦不过是过错\footnote{\bi{戈麦的诗} \regular{戈麦 西渡}}

\title{\date[d=11,m=11,y=2024][year:cn-y,年,month:cn,day:cn,日,·,weekday]·十月十一 }
灯光释放黑夜。\footnote{\bi{昌耀的诗} \regular{昌耀}}

\title{\date[d=12,m=11,y=2024][year:cn-y,年,month:cn,day:cn,日,·,weekday]·十月十二 }
生活中没有初学者的班级,它总是要求马上就开始最难的。\footnote{\bi{布里格手记} \regular{里尔克}}

\title{\date[d=13,m=11,y=2024][year:cn-y,年,month:cn,day:cn,日,·,weekday]·十月十三 }
她说她的忧郁是扇形的\footnote{\bi{于坚的诗} \regular{于坚}}

\title{\date[d=14,m=11,y=2024][year:cn-y,年,month:cn,day:cn,日,·,weekday]·十月十四 }
时光在流动,但是不会离去,也不会改变。你在同一时间里分处于不同的地方。\footnote{\bi{抒情诗的呼吸:一九二六年书信(帕斯捷尔纳克作品系列)} \regular{鲍·列·帕斯捷尔纳克 玛·伊·茨维塔耶娃 莱·马·里尔克}}

\title{\date[d=15,m=11,y=2024][year:cn-y,年,month:cn,day:cn,日,·,weekday]·十月十五 }
它本身就是一种箴言\footnote{\bi{于坚的诗} \regular{于坚}}

\title{\date[d=16,m=11,y=2024][year:cn-y,年,month:cn,day:cn,日,·,weekday]·十月十六 }
大多数人自发地过着虚构的他人生活。\footnote{\bi{不安之书} \regular{费尔南多·佩索阿 热罗尼莫·皮萨罗 }}

\title{\date[d=17,m=11,y=2024][year:cn-y,年,month:cn,day:cn,日,·,weekday]·十月十七 }
经历之中什么教你最痛苦,喝的要是太苦就改喝酒吧。无尽的长夜于你感官交汇的奇异处就成为它们奇逢的意义。\footnote{\bi{最好的里尔克} \regular{赖纳·马利亚·里尔克}}

\title{\date[d=18,m=11,y=2024][year:cn-y,年,month:cn,day:cn,日,·,weekday]·十月十八 }
死于红尘滚滚\footnote{\bi{王家新的诗} \regular{王家新}}

\title{\date[d=19,m=11,y=2024][year:cn-y,年,month:cn,day:cn,日,·,weekday]·十月十九 }
爱情活在语言里,却死在行动中。\footnote{\bi{抒情诗的呼吸:一九二六年书信(帕斯捷尔纳克作品系列)} \regular{鲍·列·帕斯捷尔纳克 玛·伊·茨维塔耶娃 莱·马·里尔克}}

\title{\date[d=20,m=11,y=2024][year:cn-y,年,month:cn,day:cn,日,·,weekday]·十月二十 }
她像炊烟忠实于天空\footnote{\bi{于坚的诗} \regular{于坚}}

\title{\date[d=21,m=11,y=2024][year:cn-y,年,month:cn,day:cn,日,·,weekday]·十月廿一 }
那对我们当下的环境来说很不妙。社会一步步变成充满秘密的社会。现代专制者的罪恶之处在于他善于藏身匿迹。他比他的奴仆更默默无闻。比起过去恃强凌弱的专制者,他更像一个懦夫。有钱的出版商对待穷困潦倒的诗人就像旧时的师父对待学徒。不同的是,以前学徒逃了师父追着跑,如今是诗人徒劳无益地追逐并试图明确责任事实,而出版商则逃之夭夭。所罗门先生的员工被炒了鱿鱼;苏莱曼大帝[插图]美丽的希腊女奴也卷了铺盖,或者说铺盖卷了她。虽然她已淹没在博斯普鲁斯海峡黑色的潮水中,但驱逐她的人并没有被淹没——他骑着白象走在胜利的号角声里。那名员工的情况就不同了,解雇者之谜就像他的前途一样难以看清。下令解雇他的可能是所罗门先生,也可能是所罗门先生的主管,还可能是所罗门先生在切尔滕纳姆的有钱姑母,或者是所罗门先生在柏林的有钱债主。曾经为了厘清责任而生的复杂机制如今只被用来推卸责任。\footnote{\bi{事事关心} \regular{吉尔伯特•基思•切斯特顿}}

\title{\date[d=22,m=11,y=2024][year:cn-y,年,month:cn,day:cn,日,·,weekday]·十月廿二 ·小雪}
我在绝望中期待,我的离去就是留下\footnote{\bi{堂吉诃德(译文名著精选)} \regular{塞万提斯}}

\title{\date[d=23,m=11,y=2024][year:cn-y,年,month:cn,day:cn,日,·,weekday]·十月廿三 }
从五根弦普通的和声捕捉一生信号”\footnote{\bi{沉石与火舌:特朗斯特罗姆诗全集} \regular{托马斯·特朗斯特罗姆}}

\title{\date[d=24,m=11,y=2024][year:cn-y,年,month:cn,day:cn,日,·,weekday]·十月廿四 }
向你注入我的毒液,啊,我的妹妹!\footnote{\bi{恶之花} \regular{波德莱尔}}

\title{\date[d=25,m=11,y=2024][year:cn-y,年,month:cn,day:cn,日,·,weekday]·十月廿五 }
我什么都不是……我是一个虚构……我想从世上的一切或自身之中获得什么?\footnote{\bi{想象一朵未来的玫瑰:佩索阿诗选} \regular{费尔南多·佩索阿}}

\title{\date[d=26,m=11,y=2024][year:cn-y,年,month:cn,day:cn,日,·,weekday]·十月廿六 }
仿佛宇宙是一个错误,一切都在沉睡;风,不确定地飘着,好像无形的旗帜在不存在的大厦上飘扬。\footnote{\bi{不安之书} \regular{费尔南多·佩索阿 热罗尼莫·皮萨罗 }}

\title{\date[d=27,m=11,y=2024][year:cn-y,年,month:cn,day:cn,日,·,weekday]·十月廿七 }
疲惫的人是逐步沉向孤独的动物。\footnote{\bi{昌耀的诗} \regular{昌耀}}

\title{\date[d=28,m=11,y=2024][year:cn-y,年,month:cn,day:cn,日,·,weekday]·十月廿八 }
诗人诞生于离世,因为只有在他死后,人们才开始欣赏他的诗歌。\footnote{\bi{不安之书} \regular{费尔南多·佩索阿 热罗尼莫·皮萨罗 }}

\title{\date[d=29,m=11,y=2024][year:cn-y,年,month:cn,day:cn,日,·,weekday]·十月廿九 }
我的心是全人类的汇合点,\footnote{\bi{宇宙重建了自身:佩索阿诗精选} \regular{费尔南多·佩索阿}}

\title{\date[d=30,m=11,y=2024][year:cn-y,年,month:cn,day:cn,日,·,weekday]·十月三十 }
你以为自己将腐烂得更快一些吗?\footnote{\bi{昌耀的诗} \regular{昌耀}}

\title{\date[d=1,m=12,y=2024][year:cn-y,年,month:cn,day:cn,日,·,weekday]·十一月初一 }
我们的使命就是把这个短暂而羸弱的大地深深地、悲悯地、痴情地铭刻在心,好让它的本质在我们心中“不可见地”复活。\footnote{\bi{谁此时孤独:里尔克晚期书信选} \regular{里尔克}}

\title{\date[d=2,m=12,y=2024][year:cn-y,年,month:cn,day:cn,日,·,weekday]·十一月初二 }
我公然不模糊眼睛。我透过大雨凝望。\footnote{\bi{茨维塔耶娃诗选} \regular{茨维塔耶娃}}

\title{\date[d=3,m=12,y=2024][year:cn-y,年,month:cn,day:cn,日,·,weekday]·十一月初三 }
大海在自己加工自己\footnote{\bi{抒情诗的呼吸:一九二六年书信(帕斯捷尔纳克作品系列)} \regular{鲍·列·帕斯捷尔纳克 玛·伊·茨维塔耶娃 莱·马·里尔克}}

\title{\date[d=4,m=12,y=2024][year:cn-y,年,month:cn,day:cn,日,·,weekday]·十一月初四 }
这个世界应当改变吗?怎样改变?谁来改变?不,它一切都很正常\footnote{\bi{四川好人} \regular{贝托尔特·布莱希特}}

\title{\date[d=5,m=12,y=2024][year:cn-y,年,month:cn,day:cn,日,·,weekday]·十一月初五 }
我想要暗淡、明朗、改观。想要别人心灵和自己心灵的远角。想要你永远也听不到、永远也不会说的那些话。想要不同寻常的东西。想要神奇的东西。想要奇迹。\footnote{\bi{抒情诗的呼吸:一九二六年书信(帕斯捷尔纳克作品系列)} \regular{鲍·列·帕斯捷尔纳克 玛·伊·茨维塔耶娃 莱·马·里尔克}}

\title{\date[d=6,m=12,y=2024][year:cn-y,年,month:cn,day:cn,日,·,weekday]·十一月初六 ·大雪}
一口祖先们向后代挖掘的井。\footnote{\bi{海子的诗} \regular{海子}}

\title{\date[d=7,m=12,y=2024][year:cn-y,年,month:cn,day:cn,日,·,weekday]·十一月初七 }
我的注意力被永不间断的幻境取代。在看见的事物乃至梦中所见之上叠加我带来的其他梦。我表现得足够心不在焉,以便践行所谓的“在梦中观看事物”\footnote{\bi{不安之书} \regular{费尔南多·佩索阿 热罗尼莫·皮萨罗 }}

\title{\date[d=8,m=12,y=2024][year:cn-y,年,month:cn,day:cn,日,·,weekday]·十一月初八 }
荒野没有词空白之页向四方展开!\footnote{\bi{沉石与火舌:特朗斯特罗姆诗全集} \regular{托马斯·特朗斯特罗姆}}

\title{\date[d=9,m=12,y=2024][year:cn-y,年,month:cn,day:cn,日,·,weekday]·十一月初九 }
即使将这块玻璃画上彩色的梦,也无法逃避从窗外获得的外部生活的杂音。\footnote{\bi{不安之书} \regular{费尔南多·佩索阿 热罗尼莫·皮萨罗 }}

\title{\date[d=10,m=12,y=2024][year:cn-y,年,month:cn,day:cn,日,·,weekday]·十一月初十 }
因为人的内心快乐抑或内心痛苦首先就是人的感情、意欲和思想的产物。\footnote{\bi{人生的智慧} \regular{叔本华}}

\title{\date[d=11,m=12,y=2024][year:cn-y,年,month:cn,day:cn,日,·,weekday]·十一月十一 }
幕的后面,神在打牌\footnote{\bi{戈麦的诗} \regular{戈麦 西渡}}

\title{\date[d=12,m=12,y=2024][year:cn-y,年,month:cn,day:cn,日,·,weekday]·十一月十二 }
我正死着自己的死和后人的死。\footnote{\bi{荒原:艾略特文集·诗歌} \regular{T.S.艾略特}}

\title{\date[d=13,m=12,y=2024][year:cn-y,年,month:cn,day:cn,日,·,weekday]·十一月十三 }
我走出森林的底部。 树之间浮出光亮。\footnote{\bi{沉石与火舌:特朗斯特罗姆诗全集} \regular{托马斯·特朗斯特罗姆}}

\title{\date[d=14,m=12,y=2024][year:cn-y,年,month:cn,day:cn,日,·,weekday]·十一月十四 }
我生来就有一颗阴暗的心。我的心从未懂得自在的开朗。\footnote{\bi{金阁寺} \regular{三岛由纪夫}}

\title{\date[d=15,m=12,y=2024][year:cn-y,年,month:cn,day:cn,日,·,weekday]·十一月十五 }
与你相遇的时候,我就是在与自己相遇,与所有锋芒都转过来针对我的那个自我相遇。\footnote{\bi{抒情诗的呼吸:一九二六年书信(帕斯捷尔纳克作品系列)} \regular{鲍·列·帕斯捷尔纳克 玛·伊·茨维塔耶娃 莱·马·里尔克}}

\title{\date[d=16,m=12,y=2024][year:cn-y,年,month:cn,day:cn,日,·,weekday]·十一月十六 }
我只不过是以飞行为生\footnote{\bi{骆一禾的诗} \regular{骆一禾 西渡}}

\title{\date[d=17,m=12,y=2024][year:cn-y,年,month:cn,day:cn,日,·,weekday]·十一月十七 }
这建筑是用金泥写在无明长夜上的建筑\footnote{\bi{金阁寺} \regular{三岛由纪夫}}

\title{\date[d=18,m=12,y=2024][year:cn-y,年,month:cn,day:cn,日,·,weekday]·十一月十八 }
直到光追上我叠起时间\footnote{\bi{沉石与火舌:特朗斯特罗姆诗全集} \regular{托马斯·特朗斯特罗姆}}

\title{\date[d=19,m=12,y=2024][year:cn-y,年,month:cn,day:cn,日,·,weekday]·十一月十九 }
死亡如眼泪,把诗人腌制得更加诗人\footnote{\bi{张枣的诗} \regular{张枣 颜炼军}}

\title{\date[d=20,m=12,y=2024][year:cn-y,年,month:cn,day:cn,日,·,weekday]·十一月二十 }
用静止计算时间\footnote{\bi{戈麦的诗} \regular{戈麦 西渡}}

\title{\date[d=21,m=12,y=2024][year:cn-y,年,month:cn,day:cn,日,·,weekday]·十一月廿一 ·冬至}
我是一个王子心是我的王国\footnote{\bi{顾城的诗} \regular{顾城}}

\title{\date[d=22,m=12,y=2024][year:cn-y,年,month:cn,day:cn,日,·,weekday]·十一月廿二 }
自然也变坏了吗,\footnote{\bi{秋日:冯至译诗选} \regular{歌德 海涅 尼采 荷尔德林 布莱希特 里尔克 格奥尔格}}

\title{\date[d=23,m=12,y=2024][year:cn-y,年,month:cn,day:cn,日,·,weekday]·十一月廿三 }
我为爱你而心痛过,那么不再爱你的痛也含有亲密……\footnote{\bi{我的心迟到了:佩索阿情诗} \regular{费尔南多·佩索阿}}

\title{\date[d=24,m=12,y=2024][year:cn-y,年,month:cn,day:cn,日,·,weekday]·十一月廿四 }
这世界,凭自己的力量,变成了你的世界\footnote{\bi{谁此时孤独:里尔克晚期书信选} \regular{里尔克}}

\title{\date[d=25,m=12,y=2024][year:cn-y,年,month:cn,day:cn,日,·,weekday]·十一月廿五 }
疯狂的愤怒,因为无人有足够的生命成为所有人,\footnote{\bi{宇宙重建了自身:佩索阿诗精选} \regular{费尔南多·佩索阿}}

\title{\date[d=26,m=12,y=2024][year:cn-y,年,month:cn,day:cn,日,·,weekday]·十一月廿六 }
我心中所逝,早已超越了陈年过往。\footnote{\bi{不安之书} \regular{费尔南多·佩索阿 热罗尼莫·皮萨罗 }}

\title{\date[d=27,m=12,y=2024][year:cn-y,年,month:cn,day:cn,日,·,weekday]·十一月廿七 }
那个夏天还在拖延那个声音已经停止\footnote{\bi{顾城的诗} \regular{顾城}}

\title{\date[d=28,m=12,y=2024][year:cn-y,年,month:cn,day:cn,日,·,weekday]·十一月廿八 }
你要迎着黄昏歌唱迎着黄昏歌唱你便走到黑夜的那边\footnote{\bi{骆一禾的诗} \regular{骆一禾 西渡}}

\title{\date[d=29,m=12,y=2024][year:cn-y,年,month:cn,day:cn,日,·,weekday]·十一月廿九 }
该如何回答这大梦沉沉的天问?\footnote{\bi{欧阳江河的诗} \regular{欧阳江河}}

\title{\date[d=30,m=12,y=2024][year:cn-y,年,month:cn,day:cn,日,·,weekday]·十一月三十 }
是为了一些瞬间而活\footnote{\bi{抒情诗的呼吸:一九二六年书信(帕斯捷尔纳克作品系列)} \regular{鲍·列·帕斯捷尔纳克 玛·伊·茨维塔耶娃 莱·马·里尔克}}

\title{\date[d=31,m=12,y=2024][year:cn-y,年,month:cn,day:cn,日,·,weekday]·腊月初一 }
我们会一辈子玩得高兴我们的玩具是整个世界\footnote{\bi{于坚的诗} \regular{于坚}}

